%%\documentstyle[cmu-titlepage]{report} % -*- Dictionary: design -*-
%\documentstyle{report} % -*- Dictionary: design -*-

\documentclass{report}
\usepackage{ifthen}
\usepackage{calc}
\usepackage{palatino}
\usepackage[hyperindex=false,colorlinks=false,urlcolor=blue]{hyperref}

% define a new conditional statement which allows us to include
% stuff conditionally when compiling to PDF. 
\newif\ifpdf
\ifx\pdfoutput\undefined
   \pdffalse
\else
   \pdfoutput=1
   \pdftrue
\fi



\title{Design of CMU Common Lisp}
\date{January 15, 2003}
\author{Robert A. MacLachlan (ed)}

\ifpdf
\pdfinfo{
/Author (Robert A. MacLachlan, ed)
/Title (Design of CMU Common Lisp)
}
% Add section numbers to the bookmarks, and open 2 levels by default.
\hypersetup{bookmarksnumbered=true,
            bookmarksopen=true,
            bookmarksopenlevel=2}
\fi
%%\trnumber{CMU-CS-91-???}

%% This code taken from the LaTeX companion.  It's meant as a
%% replacement for the description environment.  We want one that
%% prints description items in a fixed size box and puts the
%% description itself on the same line or the next depending on the
%% size of the item.
\newcommand{\entrylabel}[1]{\mbox{#1}\hfil}
\newenvironment{entry}{%
  \begin{list}{}%
    {\renewcommand{\makelabel}{\entrylabel}%
      \setlength{\labelwidth}{45pt}%
      \setlength{\leftmargin}{\labelwidth+\labelsep}}}%
  {\end{list}}

\newlength{\Mylen}
\newcommand{\Lentrylabel}[1]{%
  \settowidth{\Mylen}{#1}%
  \ifthenelse{\lengthtest{\Mylen > \labelwidth}}%
  {\parbox[b]{\labelwidth}%  term > labelwidth
    {\makebox[0pt][l]{#1}\\}}%
  {#1}%
  \hfil\relax}
\newenvironment{Lentry}{%
  \renewcommand{\entrylabel}{\Lentrylabel}
  \begin{entry}}%
  {\end{entry}}

\setcounter{tocdepth}{2}
\setcounter{secnumdepth}{3}
\def\textfraction{.1}
\def\bottomfraction{.9}         % was .3
\def\topfraction{.9}

\newcommand{\code}[1]{\textnormal{{\sffamily #1}}}
%% Some common abbreviations
\newcommand{\cmucl}{\textsc{cmucl}}

%% Set up margins
\setlength{\oddsidemargin}{-10pt}
\setlength{\evensidemargin}{-10pt}
\setlength{\topmargin}{-40pt}
\setlength{\headheight}{12pt}
\setlength{\headsep}{25pt}
\setlength{\footskip}{30pt}
\setlength{\textheight}{9.25in}
\setlength{\textwidth}{6.75in}
\setlength{\columnsep}{0.375in}
\setlength{\columnseprule}{0pt}


\begin{document}
\maketitle
\abstract{This report documents internal details of the CMU Common Lisp
compiler and run-time system.  CMU Common Lisp is a public domain
implementation of Common Lisp that runs on various Unix workstations.
This document is a work in progress: neither the contents nor the
presentation are completed. Nevertheless, it provides some useful
background information, in particular regarding the \cmucl{} compiler.}

\tableofcontents
\part{System Architecture}% -*- Dictionary: int:design -*-

\chapter{Package and File Structure}

\section{Source Tree Structure}

The \cmucl{} source tree has subdirectories for each major subsystem:

\begin{description}
\item[{\tt assembly/}] Holds the CMU CL source-file assembler, and has machine
specific subdirectories holding assembly code for that architecture.

\item[{\tt clx/}] The CLX interface to the X11 window system.

\item[{\tt code/}] The Lisp code for the runtime system and standard CL
utilities.

\item[{\tt compiler/}] The Python compiler.  Has architecture-specific
subdirectories which hold backends for different machines.  The {\tt generic}
subdirectory holds code that is shared across most backends.

\item[{\tt hemlock/}] The Hemlock editor.

\item[{\tt lisp/}] The C runtime system code and low-level Lisp debugger.

\item[{\tt pcl/}] \cmucl{} version of the PCL implementation of CLOS.

\item[{\tt tools/}] System building command files and source management tools.
\end{description}


\section{Package structure}

Goals: with the single exception of LISP, we want to be able to export from the
package that the code lives in.

\begin{description}
\item[Mach, CLX...] --- These Implementation-dependent system-interface
packages provide direct access to specific features available in the operating
system environment, but hide details of how OS communication is done.

\item[system] contains code that must know about the operating system
environment: I/O, etc.  Hides the operating system environment.  Provides OS
interface extensions such as {\tt print-directory}, etc.

\item[kernel] hides state and types used for system integration: package
system, error system, streams (?), reader, printer.  Also, hides the VM, in
that we don't export anything that reveals the VM interface.  Contains code
that needs to use the VM and SYSTEM interface, but is independent of OS and VM
details.  This code shouldn't need to be changed in any port of CMU CL, but
won't work when plopped into an arbitrary CL.  Uses SYSTEM, VM, EXTENSIONS.  We
export "hidden" symbols related to implementation of CL: setf-inverses,
possibly some global variables.

The boundary between KERNEL and VM is fuzzy, but this fuzziness reflects the
fuzziness in the definition of the VM.  We can make the VM large, and bring
everything inside, or we can make it small.  Obviously, we want the VM to be
as small as possible, subject to efficiency constraints.  Pretty much all of
the code in KERNEL could be put in VM.  The issue is more what VM hides from
KERNEL: VM knows about everything.

\item[lisp]  Originally, this package had all the system code in it.  The
current ideal is that this package should have {\it no} code in it, and only
exist to export the standard interface.  Note that the name has been changed by
x3j13 to common-lisp.

\item[extensions] contains code that any random user could have written: list
operations, syntactic sugar macros.  Uses only LISP, so code in EXTENSIONS is
pure CL.  Exports everything defined within that is useful elsewhere.  This
package doesn't hide much, so it is relatively safe for users to use
EXTENSIONS, since they aren't getting anything they couldn't have written
themselves.  Contrast this to KERNEL, which exports additional operations on
CL's primitive data structures: PACKAGE-INTERNAL-SYMBOL-COUNT, etc.  Although
some of the functionality exported from KERNEL could have been defined in CL,
the kernel implementation is much more efficient because it knows about
implementation internals.  Currently this package contains only extensions to
CL, but in the ideal scheme of things, it should contain the implementations of
all CL functions that are in KERNEL (the library.)

\item[VM] hides information about the hardware and data structure
representations.  Contains all code that knows about this sort of thing: parts
of the compiler, GC, etc.  The bulk of the code is the compiler back-end.
Exports useful things that are meaningful across all implementations, such as
operations for examining compiled functions, system constants.  Uses COMPILER
and whatever else it wants.  Actually, there are different {\it machine}{\tt
-VM} packages for each target implementation.  VM is a nickname for whatever
implementation we are currently targeting for.


\item[compiler] hides the algorithms used to map Lisp semantics onto the
operations supplied by the VM.  Exports the mechanisms used for defining the
VM.  All the VM-independent code in the compiler, partially hiding the compiler
intermediate representations.  Uses KERNEL.

\item[eval] holds code that does direct execution of the compiler's ICR.  Uses
KERNEL, COMPILER.  Exports debugger interface to interpreted code.

\item[debug-internals] presents a reasonable, unified interface to
manipulation of the state of both compiled and interpreted code.  (could be in
KERNEL) Uses VM, INTERPRETER, EVAL, KERNEL.

\item[debug] holds the standard debugger, and exports the debugger 
\end{description}

\chapter{System Building}

It's actually rather easy to build a CMU CL core with exactly what you want in
it.  But to do this you need two things: the source and a working CMU CL.

Basically, you use the working copy of CMU CL to compile the sources,
then run a process call ``genesis'' which builds a ``kernel'' core.
You then load whatever you want into this kernel core, and save it.

In the \verb|tools/| directory in the sources there are several files that
compile everything, and build cores, etc.  The first step is to compile the C
startup code.

{\bf Note:} {\it the various scripts mentioned below have hard-wired paths in
them set up for our directory layout here at CMU.  Anyone anywhere else will
have to edit them before they will work.}

\section{Compiling the C Startup Code}

There is a circular dependancy between lisp/internals.h and lisp/lisp.map that
causes bootstrapping problems.  The easiest way to get around this problem
is to make a fake lisp.nm file that has nothing in it but a version number:

\begin{verbatim}
        % echo "Map file for lisp version 0" > lisp.nm
\end{verbatim}
and then run genesis with NIL for the list of files:
\begin{verbatim}
        * (load ".../compiler/generic/new-genesis") ; compile before loading
        * (lisp::genesis nil ".../lisp/lisp.nm" "/dev/null"
                ".../lisp/lisp.map" ".../lisp/lisp.h")
\end{verbatim}
It will generate
a whole bunch of warnings about things being undefined, but ignore
that, because it will also generate a correct lisp.h.  You can then
compile lisp producing a correct lisp.map:
\begin{verbatim}
        % make
\end{verbatim}
and then use \verb|tools/do-worldbuild| and \verb|tools/mk-lisp| to build
\verb|kernel.core| and \verb|lisp.core| (see section \ref{building-cores}.)

\section{Compiling the Lisp Code}

The \verb|tools| directory contains various lisp and C-shell utilities for
building CMU CL:
\begin{description}
\item[compile-all*] Will compile lisp files and build a kernel core.  It has
numerous command-line options to control what to compile and how.  Try -help to
see a description.  It runs a separate Lisp process to compile each
subsystem.  Error output is generated in files with ``{\tt .log}'' extension in
the root of the build area.

\item[setup.lisp] Some lisp utilities used for compiling changed files in batch
mode and collecting the error output. Sort of a crude defsystem.  Loads into the
``user'' package.  See {\tt with-compiler-log-file} and {\tt comf}.

\item[{\it foo}com.lisp] Each system has a ``\verb|.lisp|'' file in
\verb|tools/| which compiles that system.
\end{description}

\section{Building Core Images}
\label{building-cores}
Both the kernel and final core build are normally done using shell script
drivers:
\begin{description}
\item[do-worldbuild*] Builds a kernel core for the current machine.  The
version to build is indicated by an optional argument, which defaults to
``alpha''.  The \verb|kernel.core| file is written either in the \verb|lisp/|
directory in the build area, or in \verb|/usr/tmp/|.  The directory which
already contains \verb|kernel.core| is chosen.  You can create a dummy version
with e.g. ``touch'' to select the initial build location.

\item[mk-lisp*] Builds a full core, with conditional loading of subsystems.
The version is the first argument, which defaults to ``alpha''.  Any additional
arguments are added to the \verb|*features*| list, which controls system
loading (among other things.)  The \verb|lisp.core| file is written in the
current working directory.
\end{description}

These scripts load Lisp command files.  When \verb|tools/worldbuild.lisp| is
loaded, it calls genesis with the correct arguments to build a kernel core.
Similarly, \verb|worldload.lisp|
builds a full core.  Adding certain symbols to \verb|*features*| before
loading worldload.lisp suppresses loading of different parts of the
system.  These symbols are:
\begin{description}
\item[:no-compiler] don't load the compiler.
\item[:no-clx] don't load CLX.
\item[:no-clm] don't load CLM.
\item[:no-hemlock] don't load Hemlock.
\item[:no-pcl] don't load PCL.
\item[:runtime] build a runtime code, implies all of the above, and then some.
\end{description}

Note: if you don't load the compiler, you can't (successfully) load the
pretty-printer or pcl.  And if you compiled hemlock with CLX loaded, you can't
load it without CLX also being loaded.

These features are only used during the worldload process; they are
not propagated to the generated \verb|lisp.core| file. 

\part{Compiler Organization}
\chapter{Compiler Overview} % -*- Dictionary: design -*-

The structure of the compiler may be broadly characterized by describing the
compilation phases and the data structures that they manipulate.  The steps in
the compilation are called phases rather than passes since they don't
necessarily involve a full pass over the code.  The data structure used to
represent the code at some point is called an {\it intermediate
representation.}

Two major intermediate representations are used in the compiler:
\begin{itemize}

\item The Implicit Continuation Representation (ICR) represents the lisp-level
semantics of the source code during the initial phases.  Partial evaluation and
semantic analysis are done on this representation.  ICR is roughly equivalent
to a subset of Common Lisp, but is represented as a flow-graph rather than a
syntax tree.  Phases which only manipulate ICR comprise the ``front end''.  It
would be possible to use a different back end such as one that directly
generated code for a stack machine.

\item The Virtual Machine Representation (VMR) represents the implementation of
the source code on a virtual machine.  The virtual machine may vary depending
on the the target hardware, but VMR is sufficiently stylized that most of the
phases which manipulate it are portable.
\end{itemize}

Each phase is briefly described here.  The phases from ``local call analysis''
to ``constraint propagation'' all interact; for maximum optimization, they
are generally repeated until nothing new is discovered.  The source files which
primarily contain each phase are listed after ``Files: ''.
\begin{description}

\item[ICR conversion]
Convert the source into ICR, doing macroexpansion and simple source-to-source
transformation.  All names are resolved at this time, so we don't have to worry
about name conflicts later on.  Files: {\tt ir1tran, srctran, typetran}

\item[Local call analysis] Find calls to local functions and convert them to
local calls to the correct entry point, doing keyword parsing, etc.  Recognize
once-called functions as lets.  Create {\it external entry points} for
entry-point functions.  Files: {\tt locall}

\item[Find components]
Find flow graph components and compute depth-first ordering.  Separate
top-level code from run-time code, and determine which components are top-level
components.  Files: {\tt dfo}

\item[ICR optimize] A grab-bag of all the non-flow ICR optimizations.  Fold
constant functions, propagate types and eliminate code that computes unused
values.  Special-case calls to some known global functions by replacing them
with a computed function.  Merge blocks and eliminate IF-IFs.  Substitute let
variables.  Files: {\tt ir1opt, ir1tran, typetran, seqtran, vm/vm-tran}

\item[Type constraint propagation]
Use global flow analysis to propagate information about lexical variable
types.   Eliminate unnecessary type checks and tests.  Files: {\tt constraint}

\item[Type check generation]
Emit explicit ICR code for any necessary type checks that are too complex to be
easily generated on the fly by the back end.  Files: {\tt checkgen}

\item[Event driven operations]
Various parts of ICR are incrementally recomputed, either eagerly on
modification of the ICR, or lazily, when the relevant information is needed.
\begin{itemize}
\item Check that type assertions are satisfied, marking places where type
checks need to be done.

\item Locate let calls.

\item Delete functions and variables with no references
\end{itemize}
Files: {\tt ir1util}, {\tt ir1opt}

\item[ICR finalize]
This phase is run after all components have been compiled.  It scans the
global variable references, looking for references to undefined variables
and incompatible function redefinitions.  Files: {\tt ir1final}, {\tt main}.

\item[Environment analysis]
Determine which distinct environments need to be allocated, and what
context needed to be closed over by each environment.  We detect non-local
exits and set closure variables.  We also emit cleanup code as funny
function calls.  This is the last pure ICR pass.  Files: {\tt envanal}

\item[Global TN allocation (GTN)]
Iterate over all defined functions, determining calling conventions
and assigning TNs to local variables.  Files: {\tt gtn}

\item[Local TN allocation (LTN)]
Use type and policy information to determine which VMR translation to use
for known functions, and then create TNs for expression evaluation
temporaries.  We also accumulate some random information needed by VMR
conversion.  Files: {\tt ltn}

\item[Control analysis]
Linearize the flow graph in a way that minimizes the number of branches.  The
block-level structure of the flow graph is basically frozen at this point.
Files: {\tt control}

\item[Stack analysis]
Maintain stack discipline for unknown-values continuation in the presence
of local exits.  Files: {\tt stack}

\item[Entry analysis]
Collect some back-end information for each externally callable function.

\item[VMR conversion] Convert ICR into VMR by translating nodes into VOPs.
Emit type checks.  Files: {\tt ir2tran, vmdef}

\item[Copy propagation] Use flow analysis to eliminate unnecessary copying of
TN values.  Files: {\tt copyprop}

\item[Representation selection]
Look at all references to each TN to determine which representation has the
lowest cost.  Emit appropriate move and coerce VOPS for that representation.

\item[Lifetime analysis]
Do flow analysis to find the set of TNs whose lifetimes 
overlap with the lifetimes of each TN being packed.  Annotate call VOPs with
the TNs that need to be saved.  Files: {\tt life}

\item[Pack]
Find a legal register allocation, attempting to minimize unnecessary moves.
Files: {\tt pack}

\item[Code generation]
Call the VOP generators to emit assembly code.  Files: {\tt codegen}

\item[Pipeline reorganization] On some machines, move memory references
backward in the code so that they can overlap with computation.  On machines
with delayed branch instructions, locate instructions that can be moved into
delay slots.  Files: {\tt assem-opt}

\item[Assembly]
Resolve branches and convert into object code and fixup information.
Files: {\tt assembler}

\item[Dumping] Convert the compiled code into an object file or in-core
function.  Files: {\tt debug-dump}, {\tt dump}, {\tt vm/core}

\end{description}

\chapter{The Implicit Continuation Representation}

The set of special forms recognized is exactly that specified in the Common
Lisp manual.  Everything that is described as a macro in CLTL is a macro.

Large amounts of syntactic information are thrown away by the conversion to an
anonymous flow graph representation.  The elimination of names eliminates the
need to represent most environment manipulation special forms.  The explicit
representation of control eliminates the need to represent BLOCK and GO, and
makes flow analysis easy.  The full Common Lisp LAMBDA is implemented with a
simple fixed-arg lambda, which greatly simplifies later code.
      
The elimination of syntactic information eliminates the need for most of the
``beta transformation'' optimizations in Rabbit.  There are no progns, no
tagbodys and no returns.  There are no ``close parens'' which get in the way of
determining which node receives a given value.

In ICR, computation is represented by Nodes.  These are the node types:
\begin{description}
\item[if]  Represents all conditionals.

\item[set] Represents a {\tt setq}.

\item[ref] Represents a constant or variable reference.

\item[combination] Represents a normal function call.

\item[MV-combination] Represents a {\tt multiple-value-call}.  This is used to
implement all multiple value receiving forms except for {\tt
multiple-value-prog1}, which is implicit.

\item[bind]
This represents the allocation and initialization of the variables in
a lambda.

\item[return]
This collects the return value from a lambda and represents the
control transfer on return.

\item[entry] Marks the start of a dynamic extent that can have non-local exits
to it.  Dynamic state can be saved at this point for restoration on re-entry.

\item[exit] Marks a potentially non-local exit.  This node is interposed
between the non-local uses of a continuation and the {\tt dest} so that code to
do a non-local exit can be inserted if necessary.
\end{description}

Some slots are shared between all node types (via defstruct inheritance.)  This
information held in common between all nodes often makes it possible to avoid
special-casing nodes on the basis of type.  This shared information is
primarily concerned with the order of evaluation and destinations and
properties of results.  This control and value flow is indicated in the node
primarily by pointing to continuations.

The {\tt continuation} structure represents information sufficiently related
to the normal notion of a continuation that naming it so seems sensible.
Basically, a continuation represents a place in the code, or alternatively the
destination of an expression result and a transfer of control.  These two
notions are bound together for the same reasons that they are related in the
standard functional continuation interpretation.

A continuation may be deprived of either or both of its value or control
significance.  If the value of a continuation is unused due to evaluation for
effect, then the continuation will have a null {\tt dest}.  If the {\tt next}
node for a continuation is deleted by some optimization, then {\tt next} will
be {\tt :none}.

  [\#\#\# Continuation kinds...]

The {\tt block} structure represents a basic block, in the the normal sense.
Control transfers other than simple sequencing are represented by information
in the block structure.  The continuation for the last node in a block
represents only the destination for the result.

It is very difficult to reconstruct anything resembling the original source
from ICR, so we record the original source form in each node.  The location of
the source form within the input is also recorded, allowing for interfaces such
as ``Edit Compiler Warnings''.  See section \ref{source-paths}.

Forms such as special-bind and catch need to have cleanup code executed at all
exit points from the form.  We represent this constraint in ICR by annotating
the code syntactically within the form with a Cleanup structure describing what
needs to be cleaned up.  Environment analysis determines the cleanup locations
by watching for a change in the cleanup between two continuations.  We can't
emit cleanup code during ICR conversion, since we don't know which exits will
be local until after ICR optimizations are done.

Special binding is represented by a call to the funny function \%Special-Bind.
The first argument is the Global-Var structure for the variable bound and the
second argument is the value to bind it to.

Some subprimitives are implemented using a macro-like mechanism for translating
\%PRIMITIVE forms into arbitrary lisp code.  Subprimitives special-cased by VMR
conversion are represented by a call to the funny function \%\%Primitive.  The
corresponding Template structure is passed as the first argument.

We check global function calls for syntactic legality with respect to any
defined function type function.  If the call is illegal or we are unable to
tell if it is legal due to non-constant keywords, then we give a warning and
mark the function reference as :notinline to force a full call and cause
subsequent phases to ignore the call.  If the call is legal and is to a known
function, then we annotate the Combination node with the Function-Info
structure that contains the compiler information for the function.


\section{Tail sets}
\#|
Probably want to have a GTN-like function result equivalence class mechanism
for ICR type inference.  This would be like the return value propagation being
done by Propagate-From-Calls, but more powerful, less hackish, and known to
terminate.  The ICR equivalence classes could probably be used by GTN, as well.

What we do is have local call analysis eagerly maintain the equivalence classes
of functions that return the same way by annotating functions with a Tail-Info
structure shared between all functions whose value could be the value of this
function.  We don't require that the calls actually be tail-recursive, only
that the call deliver its value to the result continuation.  [\#\#\# Actually
now done by ICR-OPTIMIZE-RETURN, which is currently making ICR optimize
mandatory.]

We can then use the Tail-Set during ICR type inference.  It would have a type
that is the union across all equivalent functions of the types of all the uses
other than in local calls.  This type would be recomputed during optimization
of return nodes.  When the type changes, we would propagate it to all calls to
any of the equivalent functions.  How do we know when and how to recompute the
type for a tail-set?  Recomputation is driven by type propagation on the result
continuation.

This is really special-casing of RETURN nodes.  The return node has the type
which is the union of all the non-call uses of the result.  The tail-set is
found though the lambda.  We can then recompute the overall union by taking the
union of the type per return node, rather than per-use.


How do result type assertions work?  We can't intersect the assertions across
all functions in the equivalence class, since some of the call combinations may
not happen (or even be possible).  We can intersect the assertion of the result
with the derived types for non-call uses.

When we do a tail call, we obviously can't check that the returned value
matches our assertion.  Although in principle, we would like to be able to
check all assertions, to preserve system integrity, we only need to check
assertions that we depend on.  We can afford to lose some assertion information
as long as we entirely lose it, ignoring it for type inference as well as for
type checking.

Things will work out, since the caller will see the tail-info type as the
derived type for the call, and will emit a type check if it needs a stronger
result.

A remaining question is whether we should intersect the assertion with
per-RETURN derived types from the very beginning (i.e. before the type check
pass).  I think the answer is yes.  We delay the type check pass so that we can
get our best guess for the derived type before we decide whether a check is
necessary.  But with the function return type, we aren't committing to doing
any type check when we intersect with the type assertion; the need to type
check is still determined in the type check pass by examination of the result
continuation.

What is the relationship between the per-RETURN types and the types in the
result continuation?  The assertion is exactly the Continuation-Asserted-Type
(note that the asserted type of result continuations will never change after
ICR conversion).  The per-RETURN derived type is different than the
Continuation-Derived-Type, since it is intersected with the asserted type even
before Type Check runs.  Ignoring the Continuation-Derived-Type probably makes
life simpler anyway, since this breaks the potential circularity of the
Tail-Info-Type will affecting the Continuation-Derived-Type, which affects...

When a given return has no non-call uses, we represent this by using
*empty-type*.  This is consistent with the interpretation that a return type of
NIL means the function can't return.


\section{Hairy function representation}

Non-fixed-arg functions are represented using Optional-Dispatch.  An
Optional-Dispatch has an entry-point function for each legal number of
optionals, and one for when extra args are present.  Each entry point function
is a simple lambda.  The entry point function for an optional is passed the
arguments which were actually supplied; the entry point function is expected to
default any remaining parameters and evaluate the actual function body.

If no supplied-p arg is present, then we can do this fairly easily by having
each entry point supply its default and call the next entry point, with the
last entry point containing the body.  If there are supplied-p args, then entry
point function is replaced with a function that calls the original entry
function with T's inserted at the position of all the supplied args with
supplied-p parameters.

We want to be a bit clever about how we handle arguments declared special when
doing optional defaulting, or we will emit really gross code for special
optionals.  If we bound the arg specially over the entire entry-point function,
then the entry point function would be caused to be non-tail-recursive.  What
we can do is only bind the variable specially around the evaluation of the
default, and then read the special and store the final value of the special
into a lexical variable which we then pass as the argument.  In the common case
where the default is a constant, we don't have to special-bind at all, since
the computation of the default is not affected by and cannot affect any special
bindings.

Keyword and rest args are both implemented using a LEXPR-like ``more
args'' convention.  The More-Entry takes two arguments in addition to
the fixed and optional arguments: the argument context and count.
\verb+(ARG <context> <n>)+ accesses the N'th additional argument.  Keyword
args are implemented directly using this mechanism.  Rest args are
created by calling \%Listify-Rest-Args with the context and count.

The More-Entry parses the keyword arguments and passes the values to the main
function as positional arguments.  If a keyword default is not constant, then
we pass a supplied-p parameter into the main entry and let it worry about
defaulting the argument.  Since the main entry accepts keywords in parsed form,
we can parse keywords at compile time for calls to known functions.  We keep
around the original parsed lambda-list and related information so that people
can figure out how to call the main entry.


\section{ICR representation of non-local exits}

All exits are initially represented by EXIT nodes:
How about an Exit node:
\begin{verbatim}
    (defstruct (exit (:include node))
      value)
\end{verbatim}
The Exit node uses the continuation that is to receive the thrown Value.
During optimization, if we discover that the Cont's home-lambda is the same as
the exit node's, then we can delete the Exit node, substituting the Cont for
all of the Value's uses.

The successor block of an EXIT is the entry block in the entered environment.
So we use the Exit node to mark the place where exit code is inserted.  During
environment analysis, we need only insert a single block containing the entry
point stub.

We ensure that all Exits that aren't for a NLX don't have any Value, so that
local exits never require any value massaging.

The Entry node marks the beginning of a block or tagbody:
\begin{verbatim} 
    (defstruct (entry (:include node))
      (continuations nil :type list)) 
\end{verbatim}
It contains a list of all the continuations that the body could exit to.  The
Entry node is used as a marker for the place to snapshot state, including
the control stack pointer.  Each lambda has a list of its Entries so
that environment analysis can figure out which continuations are really being
closed over.  There is no reason for optimization to delete Entry nodes,
since they are harmless in the degenerate case: we just emit no code (like a
no-var let).


We represent CATCH using the lexical exit mechanism.  We do a transformation
like this:
\begin{verbatim}
   (catch 'foo xxx)  ==>
   (block #:foo
     (%catch #'(lambda () (return-from #:foo (%unknown-values))) 'foo)
     (%within-cleanup :catch
       xxx))
\end{verbatim}

\%CATCH just sets up the catch frame which points to the exit function.  \%Catch
is an ordinary function as far as ICR is concerned.  The fact that the catcher
needs to be cleaned up is expressed by the Cleanup slots in the continuations
in the body.  \%UNKNOWN-VALUES is a dummy function call which represents the
fact that we don't know what values will be thrown.  

\%WITHIN-CLEANUP is a special special form that instantiates its first argument
as the current cleanup when converting the body.  In reality, the lambda is
also created by the special special form \%ESCAPE-FUNCTION, which gives the
lambda a special :ESCAPE kind so that the back end knows not to generate any
code for it.


We use a similar hack in Unwind-Protect to represent the fact that the cleanup
forms can be invoked at arbitrarily random times.

\begin{verbatim}
    (unwind-protect p c)  ==>
    (flet ((#:cleanup () c))
      (block #:return
	(multiple-value-bind
	    (#:next #:start #:count)
	    (block #:unwind
              (%unwind-protect #'(lambda (x) (return-from #:unwind x)))
              (%within-cleanup :unwind-protect
		(return-from #:return p)))
	  (#:cleanup)
          (%continue-unwind #:next #:start #:count))))
\end{verbatim}

We use the block \#:unwind to represent the entry to cleanup code in the case
where we are non-locally unwound.  Calling of the cleanup function in the
drop-through case (or any local exit) is handled by cleanup generation.  We
make the cleanup a function so that cleanup generation can add calls at local
exits from the protected form.  \#:next, \#:start and \#:count are state used in
the case where we are unwound.  They indicate where to go after doing the
cleanup and what values are being thrown.  The cleanup encloses only the
protected form.  As in CATCH, the escape function is specially tagged as
:ESCAPE.  The cleanup function is tagged as :CLEANUP to inhibit let conversion
(since references are added in environment analysis.)

Notice that implementing these forms using closures over continuations
eliminates any need to special-case ICR flow analysis.  Obviously we don't
really want to make heap-closures here.  In reality these functions are
special-cased by the back-end according to their KIND.


\section{Block compilation}

One of the properties of ICR is that it supports ``block compilation'' by allowing
arbitrarily large amounts of code to be converted at once, with actual
compilation of the code being done at will.


In order to preserve the normal semantics we must recognize that proclamations
(possibly implicit) are scoped.  A proclamation is in effect only from the time
of appearance of the proclamation to the time it is contradicted.  The current
global environment at the end of a block is not necessarily the correct global
environment for compilation of all the code within the block.  We solve this
problem by closing over the relevant information in the ICR at the time it is
converted.  For example, each functional variable reference is marked as
inline, notinline or don't care.  Similarly, each node contains a structure
known as a Cookie which contains the appropriate settings of the compiler
policy switches.

We actually convert each form in the file separately, creating a separate
``initial component'' for each one.  Later on, these components are merged as
needed.  The main reason for doing this is to cause EVAL-WHEN processing to be
interleaved with reading. 


\section{Entry points}

\#|

Since we need to evaluate potentially arbitrary code in the XEP argument forms
(for type checking), we can't leave the arguments in the wired passing
locations.  Instead, it seems better to give the XEP max-args fixed arguments,
with the passing locations being the true passing locations.  Instead of using
\%XEP-ARG, we reference the appropriate variable.

Also, it might be a good idea to do argument count checking and dispatching
with explicit conditional code in the XEP.  This would simplify both the code
that creates the XEP and the VMR conversion of XEPs.  Also, argument count
dispatching would automatically benefit from any cleverness in compilation of
case-like forms (jump tables, etc).  On the downside, this would push some
assumptions about how arg dispatching is done into ICR.  But then we are
currently violating abstraction at least as badly in VMR conversion, which is
also supposed to be implementation independent.
|\#

As a side-effect of finding which references to known functions can be
converted to local calls, we find any references that cannot be converted.
References that cannot be converted to a local call must evaluate to a
``function object'' (or function-entry) that can be called using the full call
convention.  A function that can be called from outside the component is called
an ``entry-point''.

Lots of stuff that happens at compile-time with local function calls must be
done at run-time when an entry-point is called.

It is desirable for optimization and other purposes if all the calls to every
function were directly present in ICR as local calls.  We cannot directly do
this with entry-point functions, since we don't know where and how the
entry-point will be called until run-time.

What we do is represent all the calls possible from outside the component by
local calls within the component.  For each entry-point function, we create a
corresponding lambda called the external entry point or XEP.  This is a
function which takes the number of arguments passed as the first argument,
followed by arguments corresponding to each required or optional argument.

If an optional argument is unsupplied, the value passed into the XEP is
undefined.  The XEP is responsible for doing argument count checking and
dispatching.  

In the case of a fixed-arg lambda, we emit a call to the \%VERIFY-ARGUMENT-COUNT
funny function (conditional on policy), then call the real function on the
passed arguments.  Even in this simple case, we benefit several ways from
having a separate XEP:
\begin{itemize}
\item The argument count checking is factored out, and only needs to
  be done in full calls.
\item Argument type checking happens automatically as a consequence of
  passing the XEP arguments in a local call to the real function.
  This type checking is also only done in full calls.
\item The real function may use a non-standard calling convention for
  the benefit of recursive or block-compiled calls.  The XEP converts
  arguments/return values to/from the standard convention.  This also
  requires little special-casing of XEPs.
\end{itemize}

If the function has variable argument count (represented by an
OPTIONAL-DISPATCH), then the XEP contains a COND which dispatches off of the
argument count, calling the appropriate entry-point function (which then does
defaulting).  If there is a more entry (for keyword or rest args), then the XEP
obtains the more arg context and count by calling the \%MORE-ARG-CONTEXT funny
function.

All non-local-call references to functions are replaced with references to the
corresponding XEP.  ICR optimization may discover a local call that was
previously a non-local reference.  When we delete the reference to the XEP, we
may find that it has no references.  In this case, we can delete the XEP,
causing the function to no longer be an entry-point.



\chapter{ICR conversion} % -*- Dictionary: design -*-


\section{Canonical forms}

\#|

Would be useful to have a Freeze-Type proclamation.  Its primary use would be
to say that the indicated type won't acquire any new subtypes in the future.
This allows better open-coding of structure type predicates, since the possible
types that would satisfy the predicate will be constant at compile time, and
thus can be compiled as a skip-chain of EQ tests.  

Of course, this is only a big win when the subtypes are few: the most important
case is when there are none.  If the closure of the subtypes is much larger
than the average number of supertypes of an inferior, then it is better to grab
the list of superiors out of the object's type, and test for membership in that
list.

Should type-specific numeric equality be done by EQL rather than =?  i.e.
should = on two fixnums become EQL and then convert to EQL/FIXNUM?
Currently we transform EQL into =, which is complicated, since we have to prove
the operands are the class of numeric type before we do it.  Also, when EQL
sees one operand is a FIXNUM, it transforms to EQ, but the generator for EQ
isn't expecting numbers, so it doesn't use an immediate compare.


\subsection{Array hackery}

Array type tests are transformed to \verb|%array-typep|, separation of the
implementation-dependent array-type handling.  This way we can transform
STRINGP to:

\begin{verbatim}
     (or (simple-string-p x)
	 (and (complex-array-p x)
	      (= (array-rank x) 1)
	      (simple-string-p (%array-data x))))  
\end{verbatim}

In addition to the similar bit-vector-p, we also handle vectorp and any type
tests on which the a dimension isn't wild.
[Note that we will want to expand into frobs compatible with those that
array references expand into so that the same optimizations will work on both.]

These changes combine to convert hairy type checks into hairy typep's, and then
convert hairyp typeps into simple typeps.


Do we really need non-VOP templates? It seems that we could get the
desired effect through implementation-dependent ICR transforms. The
main risk would be of obscuring the type semantics of the code. We
could fairly easily retain all the type information present at the
time the tranform is run, but if we discover new type information,
then it won't be propagated unless the VM also supplies type inference
methods for its internal frobs (precluding the use of
\verb|%PRIMITIVE|, since primitives don't have derive-type methods.)

I guess one possibility would be to have the call still considered ``known'' even
though it has been transformed.  But this doesn't work, since we start doing
LET optimizations that trash the arglist once the call has been transformed
(and indeed we want to.)

Actually, I guess the overhead for providing type inference methods for the
internal frobs isn't that great, since we can usually borrow the inference
method for a Common Lisp function.  For example, in our AREF case:

\begin{verbatim}
    (aref x y)
==>
    (let ((#:len (array-dimension x 0)))
      (%unchecked-aref x (%check-in-bounds y #:len)))  
\end{verbatim}

Now in this case, if we made \verb|%UNCHECKED-AREF| have the same
derive-type method as AREF, then if we discovered something new about
X's element type, we could derive a new type for the entire
expression.

Actually, it seems that baring this detail at the ICR level is
beneficial, since it admits the possibility of optimizing away the bounds
check using type information. If we discover X's dimensions, then
\verb|#:LEN| becomes a constant that can be substituted. Then
\verb|%CHECK-IN-BOUNDS| can notice that the bound is constant and
check it against the type for Y. If Y is known to be in range, then we
can optimize away the bounds check.

Actually in this particular case, the best thing to do would be if we
discovered the bound is constant, then replace the bounds check with an
implicit type check.  This way all the type check optimization mechanisms would
be brought into the act.

So we actually want to do the bounds-check expansion as soon as possible,
rather than later than possible: it should be a source-transform, enabled by
the fast-safe policy.

With multi-dimensional arrays we probably want to explicitly do the index
computation: this way portions of the index computation can become loop
invariants.  In a scan in row-major order, the inner loop wouldn't have to do
any multiplication: it would only do an addition.  We would use normal
fixnum arithmetic, counting on * to cleverly handle multiplication by a
constant, and appropriate inline expansion.

Note that in a source transform, we can't make any assumptions the type of the
array.  If it turns out to be a complex array without declared dimensions, then
the calls to ARRAY-DIMENSION will have to turn into a VOP that can be affected.
But if it is simple, then the VOP is unaffected, and if we know the bounds, it
is constant.  Similarly, we would have %ARRAY-DATA and %ARRAY-DISPLACEMENT
operations.  %ARRAY-DISPLACEMENT would optimize to 0 if we discover the array
is simple.  [This is somewhat inefficient when the array isn't eventually
discovered to be simple, since finding the data and finding the displacement
duplicate each other.  We could make %ARRAY-DATA return both as MVs, and then
optimize to (VALUES (%SIMPLE-ARRAY-DATA x) 0), but this would require
optimization of trivial VALUES uses.]

Also need (THE (ARRAY * * * ...) x) to assert correct rank.

|\#

A bunch of functions have source transforms that convert them into the
canonical form that later parts of the compiler want to see.  It is not legal
to rely on the canonical form since source transforms can be inhibited by a
Notinline declaration.  This shouldn't be a problem, since everyone should keep
their hands off of Notinline calls.

Some transformations:
\begin{verbatim}
Endp  ==>  (NULL (THE LIST ...))
(NOT xxx) or (NULL xxx) => (IF xxx NIL T)

(typep x '<simple type>) => (<simple predicate> x)
(typep x '<complex type>) => ...composition of simpler operations...
\end{verbatim}

TYPEP of AND, OR and NOT types turned into conditionals over multiple TYPEP
calls.  This makes hairy TYPEP calls more digestible to type constraint
propagation, and also means that the TYPEP code generators don't have to deal
with these cases.  [\#\#\# In the case of union types we may want to do something
to preserve information for type constraint propagation.]


\begin{verbatim}
    (apply \#'foo a b c)
==>
    (multiple-value-call \#'foo (values a) (values b) (values-list c))
\end{verbatim}

This way only MV-CALL needs to know how to do calls with unknown numbers of
arguments.  It should be nearly as efficient as a special-case VMR-Convert
method could be.


\begin{verbatim}
Make-String => Make-Array
N-arg predicates associated into two-arg versions.
Associate N-arg arithmetic ops.
Expand CxxxR and FIRST...nTH
Zerop, Plusp, Minusp, 1+, 1-, Min, Max, Rem, Mod
(Values x), (Identity x) => (Prog1 x)

All specialized aref functions => (aref (the xxx) ...)
\end{verbatim}

Convert (ldb (byte ...) ...) into internal frob that takes size and position as
separate args.  Other byte functions also...

Change for-value primitive predicates into \verb+(if <pred> t nil)+.  This isn't
particularly useful during ICR phases, but makes life easy for VMR conversion.


This last can't be a source transformation, since a source transform can't tell
where the form appears.  Instead, ICR conversion special-cases calls to known
functions with the Predicate attribute by doing the conversion when the
destination of the result isn't an IF.  It isn't critical that this never be
done for predicates that we ultimately discover to deliver their value to an
IF, since IF optimizations will flush unnecessary IFs in a predicate.


\section{Inline functions}

[\#\#\# Inline expansion is especially powerful in the presence of good lisp-level
optimization (``partial evaluation'').  Many ``optimizations'' usually done in Lisp
compilers by special-case source-to-source transforms can be had simply by
making the source of the general case function available for inline expansion.
This is especially helpful in Common Lisp, which has many commonly used
functions with simple special cases but bad general cases (list and sequence
functions, for example.)

Inline expansion of recursive functions is allowed, and is not as silly as it
sounds.  When expanded in a specific context, much of the overhead of the
recursive calls may be eliminated (especially if there are many keyword
arguments, etc.)

[Also have MAYBE-INLINE]
]

We only record a function's inline expansion in the global environment when the
function is in the null lexical environment, since the expansion must be
represented as source.

We do inline expansion of functions locally defined by FLET or LABELS even when
the environment is not null.  Since the appearances of the local function must
be nested within the desired environment, it is possible to expand local
functions inline even when they use the environment.  We just stash the source
form and environments in the Functional for the local function.  When we
convert a call to it, we just reconvert the source in the saved environment.

An interesting alternative to the inline/full-call dichotomy is ``semi-inline''
coding.  Whenever we have an inline expansion for a function, we can expand it
only once per block compilation, and then use local call to call this copied
version.  This should get most of the speed advantage of real inline coding
with much less code bloat.  This is especially attractive for simple system
functions such as Read-Char.

The main place where true inline expansion would still be worth doing is where
large amounts of the function could be optimized away by constant folding or
other optimizations that depend on the exact arguments to the call.



\section{Compilation policy}

We want more sophisticated control of compilation safety than is offered in CL,
so that we can emit only those type checks that are likely to discover
something (i.e. external interfaces.)



\section{Notes}

Generalized back-end notion provides dynamic retargeting?  (for byte code)

The current node type annotations seem to be somewhat unsatisfactory, since we
lose information when we do a THE on a continuation that already has uses, or
when we convert a let where the actual result continuation has other uses.  

But the case with THE isn't really all that bad, since the test of whether
there are any uses happens before conversion of the argument, thus THE loses
information only when there are uses outside of the declared form.  The LET
case may not be a big deal either.

Note also that losing user assertions isn't really all that bad, since it won't
damage system integrity.  At worst, it will cause a bug to go undetected.  More
likely, it will just cause the error to be signaled in a different place (and
possibly in a less informative way).  Of course, there is an efficiency hit for
losing type information, but if it only happens in strange cases, then this
isn't a big deal.


\chapter{Local call analysis}

All calls to local functions (known named functions and LETs) are resolved to
the exact LAMBDA node which is to be called.  If the call is syntactically
illegal, then we emit a warning and mark the reference as :notinline, forcing
the call to be a full call.  We don't even think about converting APPLY calls;
APPLY is not special-cased at all in ICR.  We also take care not to convert
calls in the top-level component, which would join it to normal code.  Calls to
functions with rest args and calls with non-constant keywords are also not
converted.

We also convert MV-Calls that look like MULTIPLE-VALUE-BIND to local calls,
since we know that they can be open-coded.  We replace the optional dispatch
with a call to the last optional entry point, letting MV-Call magically default
the unsupplied values to NIL.

When ICR optimizations discover a possible new local call, they explicitly
invoke local call analysis on the code that needs to be reanalyzed. 

[\#\#\# Let conversion.  What it means to be a let.  Argument type checking done
by caller.  Significance of local call is that all callers are known, so
special call conventions may be used.]
A lambda called in only one place is called a ``let'' call, since a Let would
turn into one.

In addition to enabling various ICR optimizations, the let/non-let distinction
has important environment significance.  We treat the code in function and all
of the lets called by that function as being in the same environment.  This
allows exits from lets to be treated as local exits, and makes life easy for
environment analysis.  

Since we will let-convert any function with only one call, we must be careful
about cleanups.  It is possible that a lexical exit from the let function may
have to clean up dynamic bindings not lexically apparent at the exit point.  We
handle this by annotating lets with any cleanup in effect at the call site.
The cleanup for continuations with no immediately enclosing cleanup is the
lambda that the continuation is in.  In this case, we look at the lambda to see
if any cleanups need to be done.

Let conversion is disabled for entry-point functions, since otherwise we might
convert the call from the XEP to the entry point into a let.  Then later on, we
might want to convert a non-local reference into a local call, and not be able
to, since once a function has been converted to a let, we can't convert it
back.


A function's return node may also be deleted if it is unreachable, which can
happen if the function never returns normally.  Such functions are not lets.


\chapter{Find components}

This is a post-pass to ICR conversion that massages the flow graph into the
shape subsequent phases expect.  Things done:
  Compute the depth-first ordering for the flow graph.
  Find the components (disconnected parts) of the flow graph.

This pass need only be redone when newly converted code has been added to the
flow graph.  The reanalyze flag in the component structure should be set by
people who mess things up.

We create the initial DFO using a variant of the basic algorithm.  The initial
DFO computation breaks the ICR up into components, which are parts that can be
compiled independently.  This is done to increase the efficiency of large block
compilations.  In addition to improving locality of reference and reducing the
size of flow analysis problems, this allows back-end data structures to be
reclaimed after the compilation of each component.

ICR optimization can change the connectivity of the flow graph by discovering
new calls or eliminating dead code.  Initial DFO determination splits up the
flow graph into separate components, but does so conservatively, ensuring that
parts that might become joined (due to local call conversion) are joined from
the start.  Initial DFO computation also guarantees that all code which shares
a lexical environment is in the same component so that environment analysis
needs to operate only on a single component at a time.

[This can get a bit hairy, since code seemingly reachable from the
environment entry may be reachable from a NLX into that environment.  Also,
function references must be considered as links joining components even though
the flow graph doesn't represent these.]

After initial DFO determination, components are neither split nor joined.  The
standard DFO computation doesn't attempt to split components that have been
disconnected.


\chapter{ICR optimize}

{\bf Somewhere describe basic ICR utilities: continuation-type,
constant-continuation-p, etc.  Perhaps group by type in ICR description?}

We are conservative about doing variable-for-variable substitution in ICR
optimization, since if we substitute a variable with a less restrictive type,
then we may prevent use of a ``good'' representation within the scope of the
inner binding.

Note that variable-variable substitutions aren't really crucial in ICR, since
they don't create opportunities for new optimizations (unlike substitution of
constants and functions).  A spurious variable-variable binding will show up as
a Move operation in VMR.  This can be optimized away by reaching-definitions
and also by targeting.  [\#\#\# But actually, some optimizers do see if operands
are the same variable.]

\#|

The IF-IF optimization can be modeled as a value driven optimization, since
adding a use definitely is cause for marking the continuation for
reoptimization.  [When do we add uses?  Let conversion is the only obvious
time.]  I guess IF-IF conversion could also be triggered by a non-immediate use
of the test continuation becoming immediate, but to allow this to happen would
require Delete-Block (or somebody) to mark block-starts as needing to be
reoptimized when a predecessor changes.  It's not clear how important it is
that IF-IF conversion happen under all possible circumstances, as long as it
happens to the obvious cases.

[\#\#\# It isn't totally true that code flushing never enables other worthwhile
optimizations.  Deleting a functional reference can cause a function to cease
being an XEP, or even trigger let conversion.  It seems we still want to flush
code during ICR optimize, but maybe we want to interleave it more intimately
with the optimization pass.  

Ref-flushing works just as well forward as backward, so it could be done in the
forward pass.  Call flushing doesn't work so well, but we could scan the block
backward looking for any new flushable stuff if we flushed a call on the
forward pass.

When we delete a variable due to lack of references, we leave the variable
in the lambda-list so that positional references still work.  The initial value
continuation is flushed, though (replaced with NIL) allowing the initial value
for to be deleted (modulo side-effects.)

Note that we can delete vars with no refs even when they have sets.  I guess
when there are no refs, we should also flush all sets, allowing the value
expressions to be flushed as well.

Squeeze out single-reference unset let variables by changing the dest of the
initial value continuation to be the node that receives the ref.  This can be
done regardless of what the initial value form is, since we aren't actually
moving the evaluation.  Instead, we are in effect using the continuation's
locations in place of the temporary variable.  

Doing this is of course, a wild violation of stack discipline, since the ref
might be inside a loop, etc.  But with the VMR back-end, we only need to
preserve stack discipline for unknown-value continuations; this ICR
transformation must be already inhibited when the DEST of the REF is a
multiple-values receiver (EXIT, RETURN or MV-COMBINATION), since we must
preserve the single-value semantics of the let-binding in this case.

The REF and variable must be deleted as part of this operation, since the ICR
would otherwise be left in an inconsistent state; we can't wait for the REF to
be deleted due to being unused, since we have grabbed the arg continuation and
substituted it into the old DEST.

The big reason for doing this transformation is that in macros such as INCF and
PSETQ, temporaries are squeezed out, and the new value expression is evaluated
directly to the setter, allowing any result type assertion to be applied to the
expression evaluation.  Unlike in the case of substitution, there is no point
in inhibiting this transformation when the initial value type is weaker than
the variable type.  Instead, we intersect the asserted type for the old REF's
CONT with the type assertion on the initial value continuation.  Note that the
variable's type has already been asserted on the initial-value continuation.

Of course, this transformation also simplifies the ICR even when it doesn't
discover interesting type assertions, so it makes sense to do it whenever
possible.  This reduces the demands placed on register allocation, etc.


There are three dead-code flushing rules:

\begin{enumerate}
\item Refs with no DEST may be flushed.

\item Known calls with no dest that are flushable may be flushed.  We null the
DEST in all the args.

\item If a lambda-var has no refs, then it may be deleted. The flushed
    argument continuations have their DEST nulled.
\end{enumerate}

These optimizations all enable one another.  We scan blocks backward, looking
for nodes whose CONT has no DEST, then type-dispatching off of the node.  If we
delete a ref, then we check to see if it is a lambda-var with no refs.  When we
flush an argument, we mark the blocks for all uses of the CONT as needing to be
reoptimized.


\section{Goals for ICR optimizations}

\#|

When an optimization is disabled, code should still be correct and not
ridiculously inefficient.  Phases shouldn't be made mandatory when they have
lots of non-required stuff jammed into them.

|\#

This pass is optional, but is desirable if anything is more important than
compilation speed.

This phase is a grab-bag of optimizations that concern themselves with the flow
of values through the code representation.  The main things done are type
inference, constant folding and dead expression elimination.  This phase can be
understood as a walk of the expression tree that propagates assertions down the
tree and propagates derived information up the tree.  The main complication is
that there isn't any expression tree, since ICR is flow-graph based.

We repeat this pass until we don't discover anything new.  This is a bit of
feat, since we dispatch to arbitrary functions which may do arbitrary things,
making it hard to tell if anything really happened.  Even if we solve this
problem by requiring people to flag when they changed or by checking to see if
they changed something, there are serious efficiency problems due to massive
redundant computation, since in many cases the only way to tell if anything
changed is to recompute the value and see if it is different from the old one.

We solve this problem by requiring that optimizations for a node only depend on
the properties of the CONT and the continuations that have the node as their
DEST.  If the continuations haven't changed since the last pass, then we don't
attempt to re-optimize the node, since we know nothing interesting will happen.

We keep track of which continuations have changed by a REOPTIMIZE flag that is
set whenever something about the continuation's value changes.

When doing the bottom up pass, we dispatch to type specific code that knows how
to tell when a node needs to be reoptimized and does the optimization.  These
node types are special-cased: COMBINATION, IF, RETURN, EXIT, SET.

The REOPTIMIZE flag in the COMBINATION-FUN is used to detect when the function
information might have changed, so that we know when there are new assertions
that could be propagated from the function type to the arguments.

When we discover something about a leaf, or substitute for leaf, we reoptimize
the CONT for all the REF and SET nodes. 

We have flags in each block that indicate when any nodes or continuations in
the block need to be re-optimized, so we don't have to scan blocks where there
is no chance of anything happening.

It is important for efficiency purposes that optimizers never say that they did
something when they didn't, but this by itself doesn't guarantee timely
termination.  I believe that with the type system implemented, type inference
will converge in finite time, but as a practical matter, it can take far too
long to discover not much.  For this reason, ICR optimization is terminated
after three consecutive passes that don't add or delete code.  This premature
termination only happens 2\% of the time.


\section{Flow graph simplification}

Things done:

\begin{itemize}
\item Delete blocks with no predecessors.
\item Merge blocks that can be merged.
\item Convert local calls to Let calls.
\item Eliminate degenerate IFs.
\end{itemize}

We take care not to merge blocks that are in different functions or have
different cleanups.  This guarantees that non-local exits are always at block
ends and that cleanup code never needs to be inserted within a block.

We eliminate IFs with identical consequent and alternative.  This would most
likely happen if both the consequent and alternative were optimized away.

[Could also be done if the consequent and alternative were different blocks,
but computed the same value.  This could be done by a sort of cross-jumping
optimization that looked at the predecessors for a block and merged code shared
between predecessors.  IFs with identical branches would eventually be left
with nothing in their branches.]

We eliminate IF-IF constructs:

\begin{verbatim}
    (IF (IF A B C) D E) ==>
    (IF A (IF B D E) (IF C D E))
\end{verbatim}

In reality, what we do is replicate blocks containing only an IF node where the
predicate continuation is the block start.  We make one copy of the IF node for
each use, leaving the consequent and alternative the same.  If you look at the
flow graph representation, you will see that this is really the same thing as
the above source to source transformation.


\section{Forward ICR optimizations}

In the forward pass, we scan the code in forward depth-first order.  We
examine each call to a known function, and:

\begin{itemize}
\item Eliminate any bindings for unused variables.

\item Do top-down type assertion propagation.  In local calls, we propagate
asserted and derived types between the call and the called lambda.

\item
    Replace calls of foldable functions with constant arguments with the
    result.  We don't have to actually delete the call node, since Top-Down
    optimize will delete it now that its value is unused.
 
\item
   Run any Optimizer for the current function.  The optimizer does arbitrary
    transformations by hacking directly on the IR.  This is useful primarily
    for arithmetic simplification and similar things that may need to examine
    and modify calls other than the current call.  The optimizer is responsible
    for recording any changes that it makes.  An optimizer can inhibit further
    optimization of the node during the current pass by returning true.  This
    is useful when deleting the node.

\item
   Do ICR transformations, replacing a global function call with equivalent
    inline lisp code.

\item
    Do bottom-up type propagation/inferencing.  For some functions such as
    Coerce we will dispatch to a function to find the result type.  The
    Derive-Type function just returns a type structure, and we check if it is
    different from the old type in order to see if there was a change.

\item
    Eliminate IFs with predicates known to be true or false.

\item
    Substitute the value for unset let variables that are bound to constants,
    unset lambda variables or functionals.

\item
    Propagate types from local call args to var refs.
\end{itemize}

We use type info from the function continuation to find result types for
functions that don't have a derive-type method.


\subsection{ICR transformation}

ICR transformation does ``source to source'' transformations on known global
functions, taking advantage of semantic information such as argument types and
constant arguments.  Transformation is optional, but should be done if speed or
space is more important than compilation speed.  Transformations which increase
space should pass when space is more important than speed.

A transform is actually an inline function call where the function is computed
at compile time.  The transform gets to peek at the continuations for the
arguments, and computes a function using the information gained.  Transforms
should be cautious about directly using the values of constant continuations,
since the compiler must preserve eqlness of named constants, and it will have a
hard time if transforms go around randomly copying constants.

The lambda that the transform computes replaces the original function variable
reference as the function for the call.  This lets the compiler worry about
evaluating each argument once in the right order.  We want to be careful to
preserve type information when we do a transform, since it may be less than
obvious what the transformed code does.

There can be any number of transforms for a function.  Each transform is
associated with a function type that the call must be compatible with.  A
transform is only invoked if the call has the right type.  This provides a way
to deal with the common case of a transform that only applies when the
arguments are of certain types and some arguments are not specified.  We always
use the derived type when determining whether a transform is applicable.  Type
check is responsible for setting the derived type to the intersection of the
asserted and derived types.

If the code in the expansion has insufficient explicit or implicit argument
type checking, then it should cause checks to be generated by making
declarations.

A transformation may decide to pass if it doesn't like what it sees when it
looks at the args.  The Give-Up function unwinds out of the transform and deals
with complaining about inefficiency if speed is more important than brevity.
The format args for the message are arguments to Give-Up.  If a transform can't
be done, we just record the message where ICR finalize can find it.  note.  We
can't complain immediately, since it might get transformed later on.


\section{Backward ICR optimizations}

In the backward pass, we scan each block in reverse order, and
eliminate any effectless nodes with unused values.  In ICR this is the
only way that code is deleted other than the elimination of unreachable blocks.


\chapter{Type checking}

%  Somehow split this section up into three parts:
%  -- Conceptual: how we know a check is necessary, and who is responsible for
%     doing checks.
%  -- Incremental: intersection of derived and asserted types, checking for
%     non-subtype relationship.
%  -- Check generation phase.

We need to do a pretty good job of guessing when a type check will ultimately
need to be done.  Generic arithmetic, for example: In the absence of
declarations, we will use the safe variant, but if we don't know this, we
will generate a check for NUMBER anyway.  We need to look at the fast-safe
templates and guess if any of them could apply.

We compute a function type from the VOP arguments
and assertions on those arguments.  This can be used with Valid-Function-Use
to see which templates do or might apply to a particular call.  If we guess
that a safe implementation will be used, then we mark the continuation so as to
force a safe implementation to be chosen.  [This will happen if ICR optimize
doesn't run to completion, so the ICR optimization after type check generation
can discover new type information.  Since we won't redo type check at that
point, there could be a call that has applicable unsafe templates, but isn't
type checkable.]

[\#\#\# A better and more general optimization of structure type checks: in type
check conversion, we look at the *original derived* type of the continuation:
if the difference between the proven type and the asserted type is a simple
type check, then check for the negation of the difference.  e.g. if we want a
FOO and we know we've got (OR FOO NULL), then test for (NOT NULL).  This is a
very important optimization for linked lists of structures, but can also apply
in other situations.]

If after ICR phases, we have a continuation with check-type set in a context
where it seems likely a check will be emitted, and the type is too 
hairy to be easily checked (i.e. no CHECK-xxx VOP), then we do a transformation
on the ICR equivalent to:

\begin{verbatim}
  (... (the hair <foo>) ...)
==>
  (... (funcall \#'(lambda (\#:val)
		    (if (typep \#:val 'hair)
			\#:val
			(%type-check-error \#:val 'hair)))
		<foo>)
       ...)
\end{verbatim}
This way, we guarantee that VMR conversion never has to emit type checks for
hairy types.

[Actually, we need to do a MV-bind and several type checks when there is a MV
continuation.  And some values types are just too hairy to check.  We really
can't check any assertion for a non-fixed number of values, since there isn't
any efficient way to bind arbitrary numbers of values.  (could be done with
MV-call of a more-arg function, I guess...)
]

[Perhaps only use CHECK-xxx VOPs for types equivalent to a ptype?  Exceptions
for CONS and SYMBOL?  Anyway, no point in going to trouble to implement and
emit rarely used CHECK-xxx vops.]

One potential lose in converting a type check to explicit conditionals rather
than to a CHECK-xxx VOP is that VMR code motion optimizations won't be able to
do anything.  This shouldn't be much of an issue, though, since type constraint
propagation has already done global optimization of type checks.


This phase is optional, but should be done if anything is more important than
compile speed.  

Type check is responsible for reconciling the continuation asserted and derived
types, emitting type checks if appropriate.  If the derived type is a subtype
of the asserted type, then we don't need to do anything.

If there is no intersection between the asserted and derived types, then there
is a manifest type error.  We print a warning message, indicating that
something is almost surely wrong.  This will inhibit any transforms or
generators that care about their argument types, yet also inhibits further
error messages, since NIL is a subtype of every type.

If the intersection is not null, then we set the derived type to the
intersection of the asserted and derived types and set the Type-Check flag in
the continuation.  We always set the flag when we can't prove that the type
assertion is satisfied, regardless of whether we will ultimately actually emit
a type check or not.  This is so other phases such as type constraint
propagation can use the Type-Check flag to detect an interesting type
assertion, instead of having to duplicate much of the work in this phase.  
[\#\#\# 7 extremely random values for CONTINUATION-TYPE-CHECK.]

Type checks are generated on the fly during VMR conversion.  When VMR
conversion generates the check, it prints an efficiency note if speed is
important.  We don't flame now since type constraint progpagation may decide
that the check is unnecessary.  [\#\#\# Not done now, maybe never.]

In local function call, it is the caller that is in effect responsible for
checking argument types.  This happens in the same way as any other type check,
since ICR optimize propagates the declared argument types to the type
assertions for the argument continuations in all the calls.

Since the types of arguments to entry points are unknown at compile time, we
want to do runtime checks to ensure that the incoming arguments are of the
correct type.  This happens without any special effort on the part of type
check, since the XEP is represented as a local call with unknown type
arguments.  These arguments will be marked as needing to be checked.


\chapter{Constraint propagation}

New lambda-var-slot:

constraints: a list of all the constraints on this var for either X or Y.

How to maintain consistency?  Does it really matter if there are constraints
with deleted vars lying around?  Note that whatever mechanism we use for
getting the constraints in the first place should tend to keep them up to date.
Probably we would define optimizers for the interesting relations that look at
their CONT's dest and annotate it if it is an IF.

But maybe it is more trouble then it is worth trying to build up the set of
constraints during ICR optimize (maintaining consistency in the process).
Since ICR optimize iterates a bunch of times before it converges, we would be
wasting time recomputing the constraints, when nobody uses them till constraint
propagation runs.  

It seems that the only possible win is if we re-ran constraint propagation
(which we might want to do.)  In that case, we wouldn't have to recompute all
the constraints from scratch.  But it seems that we could do this just as well
by having ICR optimize invalidate the affected parts of the constraint
annotation, rather than trying to keep them up to date.  This also fits better
with the optional nature of constraint propagation, since we don't want ICR
optimize to commit to doing a lot of the work of constraint propagation.  

For example, we might have a per-block flag indicating that something happened
in that block since the last time constraint propagation ran.  We might have
different flags to represent the distinction between discovering a new type
assertion inside the block and discovering something new about an if
predicate, since the latter would be cheaper to update and probably is more
common.

It's fairly easy to see how we can build these sets of restrictions and
propagate them using flow analysis, but actually using this information seems
a bit more ad-hoc.  

Probably the biggest thing we do is look at all the refs.  If we have proven that
the value is EQ (EQL for a number) to some other leaf (constant or lambda-var),
then we can substitute for that reference.  In some cases, we will want to do
special stuff depending on the DEST.  If the dest is an IF and we proved (not
null), then we can substitute T.  And if the dest is some relation on the same
two lambda-vars, then we want to see if we can show that relation is definitely
true or false.

Otherwise, we can do our best to invert the set of restrictions into a type.
Since types hold only constant info, we have to ignore any constraints between
two vars.  We can make some use of negated type restrictions by using
TYPE-DIFFERENCE to remove the type from the ref types.  If our inferred type is
as good as the type assertion, then the continuation's type-check flag will be
cleared.

It really isn't much of a problem that we don't infer union types on joins,
since union types are relatively easy to derive without using flow information.
The normal bottom-up type inference done by ICR optimize does this for us: it
annotates everything with the union of all of the things it might possibly be.
Then constraint propagation subtracts out those types that can't be in effect
because of predicates or checks.



This phase is optional, but is desirable if anything is more important than
compilation speed.  We use an algorithm similar to available expressions to
propagate variable type information that has been discovered by implicit or
explicit type tests, or by type inference.

We must do a pre-pass which locates set closure variables, since we cannot do
flow analysis on such variables.  We set a flag in each set closure variable so
that we can quickly tell that it is losing when we see it again.  Although this
may seem to be wastefully redundant with environment analysis, the overlap
isn't really that great, and the cost should be small compared to that of the
flow analysis that we are preparing to do.  [Or we could punt on set
variables...]

A type constraint is a structure that includes sset-element and has
the type and variable. [Also a not-p flag indicating whether the sense
is negated.]

Each variable has a list of its type constraints. We create a type
constraint when we see a type test or check. If there is already a
constraint for the same variable and type, then we just re-use it. If
there is already a weaker constraint, then we generate both the weak
constraints and the strong constraint so that the weak constraints
won't be lost even if the strong one is unavailable.

We find all the distinct type constraints for each variable during the pre-pass
over the lambda nesting.  Each constraint has a list of the weaker constraints
so that we can easily generate them.

Every block generates all the type constraints in it, but a constraint is
available in a successor only if it is available in all predecessors.  We
determine the actual type constraint for a variable at a block by intersecting
all the available type constraints for that variable.

This isn't maximally tense when there are constraints that are not
hierarchically related, e.g. (or a b) (or b c).  If these constraints were
available from two predecessors, then we could infer that we have an (or a b c)
constraint, but the above algorithm would come up with none.  This probably
isn't a big problem.

[\#\#\# Do we want to deal with \verb+(if (eq <var> '<foo>) ...)+ indicating singleton
member type?]

We detect explicit type tests by looking at type test annotation in the IF
node.  If there is a type check, the OUT sets are stored in the node, with
different sets for the consequent and alternative.  Implicit type checks are
located by finding Ref nodes whose Cont has the Type-Check flag set.  We don't
actually represent the GEN sets, we just initialize OUT to it, and then form
the union in place.

When we do the post-pass, we clear the Type-Check flags in the continuations
for Refs when we discover that the available constraints satisfy the asserted
type.  Any explicit uses of typep should be cleaned up by the ICR optimizer for
typep.  We can also set the derived type for Refs to the intersection of the
available type assertions.  If we discover anything, we should consider redoing
ICR optimization, since better type information might enable more
optimizations.


\chapter{ICR finalize} % -*- Dictionary: design -*-

This pass looks for interesting things in the ICR so that we can forget about
them.  Used and not defined things are flamed about.

We postpone these checks until now because the ICR optimizations may discover
errors that are not initially obvious.  We also emit efficiency notes about
optimizations that we were unable to do.  We can't emit the notes immediately,
since we don't know for sure whether a repeated attempt at optimization will
succeed.

We examine all references to unknown global function variables and update the
approximate type accordingly.  We also record the names of the unknown
functions so that they can be flamed about if they are never defined.  Unknown
normal variables are flamed about on the fly during ICR conversion, so we
ignore them here.

We check each newly defined global function for compatibility with previously
recorded type information.  If there is no :defined or :declared type, then we
check for compatibility with any approximate function type inferred from
previous uses.



\chapter{Environment analysis}

A related change would be to annotate ICR with information about tail-recursion
relations.  What we would do is add a slot to the node structure that points to
the corresponding Tail-Info when a node is in a TR position.  This annotation
would be made in a final ICR pass that runs after cleanup code is generated
(part of environment analysis).  When true, the node is in a true TR position
(modulo return-convention incompatibility).  When we determine return
conventions, we null out the tail-p slots in XEP calls or known calls where we
decided not to preserve tail-recursion. 


In this phase, we also check for changes in the dynamic binding environment
that require cleanup code to be generated.  We just check for changes in the
Continuation-Cleanup on local control transfers.  If it changes from
an inner dynamic context to an outer one that is in the same environment, then
we emit code to clean up the dynamic bindings between the old and new
continuation.  We represent the result of cleanup detection to the back end by
interposing a new block containing a call to a funny function.  Local exits
from CATCH or UNWIND-PROTECT are detected in the same way.


|\#

The primary activity in environment analysis is the annotation of ICR with
environment structures describing where variables are allocated and what values
the environment closes over.

Each lambda points to the environment where its variables are allocated, and
the environments point back.  We always allocate the environment at the Bind
node for the sole non-let lambda in the environment, so there is a close
relationship between environments and functions.  Each ``real function'' (i.e.
not a LET) has a corresponding environment.

We attempt to share the same environment among as many lambdas as possible so
that unnecessary environment manipulation is not done.  During environment
analysis the only optimization of this sort is realizing that a Let (a lambda
with no Return node) cannot need its own environment, since there is no way
that it can return and discover that its old values have been clobbered.

When the function is called, values from other environments may need to be made
available in the function's environment.  These values are said to be ``closed
over''.

Even if a value is not referenced in a given environment, it may need to be
closed over in that environment so that it can be passed to a called function
that does reference the value.  When we discover that a value must be closed
over by a function, we must close over the value in all the environments where
that function is referenced.  This applies to all references, not just local
calls, since at other references we must have the values on hand so that we can
build a closure.  This propagation must be applied recursively, since the value
must also be available in *those* functions' callers.

If a closure reference is known to be ``safe'' (not an upward funarg), then the
closure structure may be allocated on the stack.

Closure analysis deals only with closures over values, while Common Lisp
requires closures over variables.  The difference only becomes significant when
variables are set.  If a variable is not set, then we can freely make copies of
it without keeping track of where they are.  When a variable is set, we must
maintain a single value cell, or at least the illusion thereof.  We achieve
this by creating a heap-allocated ``value cell'' structure for each set variable
that is closed over.  The pointer to this value cell is passed around as the
``value'' corresponding to that variable.  References to the variable must
explicitly indirect through the value cell.

When we are scanning over the lambdas in the component, we also check for bound
but not referenced variables.

Environment analysis emits cleanup code for local exits and markers for
non-local exits.

A non-local exit is a control transfer from one environment to another.  In a
non-local exit, we must close over the continuation that we transfer to so that
the exiting function can find its way back.  We indicate the need to close a
continuation by placing the continuation structure in the closure and also
pushing it on a list in the environment structure for the target of the exit.
[\#\#\# To be safe, we would treat the continuation as a set closure variable so
that we could invalidate it when we leave the dynamic extent of the exit point.
Transferring control to a meaningless stack pointer would be apt to cause
horrible death.]

Each local control transfer may require dynamic state such as special bindings
to be undone.  We represent cleanup actions by funny function calls in a new
block linked in as an implicit MV-PROG1.


% -*- Dictionary: design -*-


\chapter{Virtual Machine Representation Introduction}


\chapter{Global TN assignment}

% Rename this phase so as not to be confused with the local/global TN
% representation.

The basic mechanism for closing over values is to pass the values as additional
implicit arguments in the function call.  This technique is only applicable
when:

\begin{itemize}
\item the calling function knows which values the called function wants to close
    over, and
\item the values to be closed over are available in the calling
  environment.
\end{itemize}

The first condition is always true of local function calls.  Environment
analysis can guarantee that the second condition holds by closing over any
needed values in the calling environment.

If the function that closes over values may be called in an environment where
the closed over values are not available, then we must store the values in a
``closure'' so that they are always accessible.  Closures are called using the
``full call'' convention.  When a closure is called, control is transferred to
the ``external entry point'', which fetches the values out of the closure and
then does a local call to the real function, passing the closure values as
implicit arguments.

In this scheme there is no such thing as a ``heap closure variable'' in code,
since the closure values are moved into TNs by the external entry point.  There
is some potential for pessimization here, since we may end up moving the values
from the closure into a stack memory location, but the advantages are also
substantial.  Simplicity is gained by always representing closure values the
same way, and functions with closure references may still be called locally
without allocating a closure.  All the TN based VMR optimizations will apply
to closure variables, since closure variables are represented in the same way
as all other variables in VMR.  Closure values will be allocated in registers
where appropriate.

Closures are created at the point where the function is referenced, eliminating
the need to be able to close over closures.  This lazy creation of closures has
the additional advantage that when a closure reference is conditionally not
done, then the closure consing will never be done at all.  The corresponding
disadvantage is that a closure over the same values may be created multiple
times if there are multiple references.  Note however, that VMR loop and common
subexpression optimizations can eliminate redundant closure consing.  In any
case, multiple closures over the same variables doesn't seem to be that common.

\#|
Having the Tail-Info would also make return convention determination trivial.
We could just look at the type, checking to see if it represents a fixed number
of values.  To determine if the standard return convention is necessary to
preserve tail-recursion, we just iterate over the equivalent functions, looking
for XEPs and uses in full calls.
|\#

The Global TN Assignment pass (GTN) can be considered a post-pass to
environment analysis.  This phase assigns the TNs used to hold local lexical
variables and pass arguments and return values and determines the value-passing
strategy used in local calls.

To assign return locations, we look at the function's tail-set.

If the result continuation for an entry point is used as the continuation for a
full call, then we may need to constrain the continuation's values passing
convention to the standard one.  This is not necessary when the call is known
not to be part of a tail-recursive loop (due to being a known function).

Once we have figured out where we must use the standard value passing strategy,
we can use a more flexible strategy to determine the return locations for local
functions.  We determine the possible numbers of return values from each
function by examining the uses of all the result continuations in the
equivalence class of the result continuation.

If the tail-set type is for a fixed number of
values, then we return that fixed number of values from all the functions whose
result continuations are equated.  If the number of values is not fixed, then
we must use the unknown-values convention, although we are not forced to use
the standard locations.  We assign the result TNs at this time.

We also use the tail-sets to see what convention we want to use.  What we do is
use the full convention for any function that has a XEP its tail-set, even if
we aren't required to do so by a tail-recursive full call, as long as there are
no non-tail-recursive local calls in the set.  This prevents us from
gratuitously using a non-standard convention when there is no reason to.


\chapter{Local TN assignment}

[Want a different name for this so as not to be confused with the different
local/global TN representations.  The really interesting stuff in this phase is
operation selection, values representation selection, return strategy, etc.
Maybe this phase should be conceptually lumped with GTN as ``implementation
selection'', since GTN determines call strategies and locations.]

\#|

[\#\#\# I guess I believe that it is OK for VMR conversion to dick the ICR flow
graph.  An alternative would be to give VMR its very own flow graph, but that
seems like overkill.

In particular, it would be very nice if a TR local call looked exactly like a
jump in VMR.  This would allow loop optimizations to be done on loops written
as recursions.  In addition to making the call block transfer to the head of
the function rather than to the return, we would also have to do something
about skipping the part of the function prolog that moves arguments from the
passing locations, since in a TR call they are already in the right frame.


In addition to directly indicating whether a call should be coded with a TR
variant, the Tail-P annotation flags non-call nodes that can directly return
the value (an ``advanced return''), rather than moving the value to the result
continuation and jumping to the return code.  Then (according to policy), we
can decide to advance all possible returns.  If all uses of the result are
Tail-P, then LTN can annotate the result continuation as :Unused, inhibiting
emission of the default return code.

[\#\#\# But not really.  Now there is a single list of templates, and a given
template has only one policy.]

In LTN, we use the :Safe template as a last resort even when the policy is
unsafe.  Note that we don't try :Fast-Safe; if this is also a good unsafe
template, then it should have the unsafe policies explicitly specified.

With a :Fast-Safe template, the result type must be proven to satisfy the
output type assertion.  This means that a fast-safe template with a fixnum
output type doesn't need to do fixnum overflow checking.  [\#\#\# Not right to
just check against the Node-Derived-Type, since type-check intersects with
this.]

It seems that it would be useful to have a kind of template where the args must
be checked to be fixnum, but the template checks for overflow and signals an
error.  In the case where an output assertion is present, this would generate
better code than conditionally branching off to make a bignum, and then doing a
type check on the result.

    How do we deal with deciding whether to do a fixnum overflow check?  This
    is perhaps a more general problem with the interpretation of result type
    restrictions in templates.  It would be useful to be able to discriminate
    between the case where the result has been proven to be a fixnum and where
    it has simply been asserted to be so.

    The semantics of result type restriction is that the result must be proven
    to be of that type *except* for safe generators, which are assumed to
    verify the assertion.  That way ``is-fixnum'' case can be a fast-safe
    generator and the ``should-be-fixnum'' case is a safe generator.  We could
    choose not to have a safe ``should-be-fixnum'' generator, and let the
    unrestricted safe generator handle it.  We would then have to do an
    explicit type check on the result.

    In other words, for all template except Safe, a type restriction on either
    an argument or result means ``this must be true; if it is not the system may
    break.''  In contrast, in a Safe template, the restriction means ``If this is
    not true, I will signal an error.''

    Since the node-derived-type only takes into consideration stuff that can be
    proved from the arguments, we can use the node-derived-type to select
    fast-safe templates.  With unsafe policies, we don't care, since the code
    is supposed to be unsafe.

|\#

Local TN assignment (LTN) assigns all the TNs needed to represent the values of
continuations.  This pass scans over the code for the component, examining each
continuation and its destination.  A number of somewhat unrelated things are
also done at the same time so that multiple passes aren't necessary.
 -- Determine the Primitive-Type for each continuation value and assigns TNs
    to hold the values.
 -- Use policy information to determine the implementation strategy for each
    call to a known function.
 -- Clear the type-check flags in continuations whose destinations have safe
    implementations.
 -- Determine the value-passing strategy for each continuation: known or
    unknown.
 -- Note usage of unknown-values continuations so that stack analysis can tell
    when stack values must be discarded.
 
If safety is more important than speed and space, then we consider generating
type checks on the values of nodes whose CONT has the Type-Check flag set.  If
the destination for the continuation value is safe, then we don't need to do
a check.  We assume that all full calls are safe, and use the template
information to determine whether inline operations are safe.

This phase is where compiler policy switches have most of their effect.  The
speed/space/safety tradeoff can determine which of a number of coding
strategies are used.  It is important to make the policy choice in VMR
conversion rather than in code generation because the cost and storage
requirement information which drives TNBIND will depend strongly on what actual
VOP is chosen.  In the case of +/FIXNUM, there might be three or more
implementations, some optimized for speed, some for space, etc.  Some of these
VOPS might be open-coded and some not.

We represent the implementation strategy for a call by either marking it as a
full call or annotating it with a ``template'' representing the open-coding
strategy.  Templates are selected using a two-way dispatch off of operand
primitive-types and policy.  The general case of LTN is handled by the
LTN-Annotate function in the function-info, but most functions are handled by a
table-driven mechanism.  There are four different translation policies that a
template may have:
\begin{description}
\item[Safe]
        The safest implementation; must do argument type checking.

\item[Small]
        The (unsafe) smallest implementation.

\item[Fast]
        The (unsafe) fastest implementation.

\item[Fast-Safe]
        An implementation optimized for speed, but which does any necessary
        checks exclusive of argument type checking.  Examples are array bounds
        checks and fixnum overflow checks.
\end{description}

Usually a function will have only one or two distinct templates.  Either or
both of the safe and fast-safe templates may be omitted; if both are specified,
then they should be distinct.  If there is no safe template and our policy is
safe, then we do a full call.

We use four different coding strategies, depending on the policy:
\begin{description}
\item[Safe:]  safety $>$ space $>$ speed, or
we want to use the fast-safe template, but there isn't one.

\item[Small:] space $>$ (max speed safety)

\item[Fast:] speed $>$ (max space safety)

\item[Fast-Safe (and type check):] safety $>$ speed $>$ space, or we want to use
the safe template, but there isn't one.
\end{description}

``Space'' above is actually the maximum of space and cspeed, under the theory
that less code will take less time to generate and assemble.  [\#\#\# This could
lose if the smallest case is out-of-line, and must allocate many linkage
registers.]


\chapter{Control optimization}

In this phase we annotate blocks with drop-throughs.  This controls how code
generation linearizes code so that drop-throughs are used most effectively.  We
totally linearize the code here, allowing code generation to scan the blocks
in the emit order.

There are basically two aspects to this optimization:

\begin{enumerate}
\item
Dynamically reducing the number of branches taken v.s. branches not
taken under the assumption that branches not taken are cheaper.
\item
Statically minimizing the number of unconditional branches, saving
space and presumably time.
\end{enumerate}

These two goals can conflict, but if they do it seems pretty clear that the
dynamic optimization should get preference.  The main dynamic optimization is
changing the sense of a conditional test so that the more commonly taken branch
is the fall-through case.  The problem is determining which branch is more
commonly taken.

The most clear-cut case is where one branch leads out of a loop and the other
is within.  In this case, clearly the branch within the loop should be
preferred.  The only added complication is that at some point in the loop there
has to be a backward branch, and it is preferable for this branch to be
conditional, since an unconditional branch is just a waste of time.

In the absence of such good information, we can attempt to guess which branch
is more popular on the basis of difference in the cost between the two cases.
Min-max strategy suggests that we should choose the cheaper alternative, since
the percentagewise improvement is greater when the branch overhead is
significant with respect to the cost of the code branched to.  A tractable
approximation of this is to compare only the costs of the two blocks
immediately branched to, since this would avoid having to do any hairy graph
walking to find all the code for the consequent and the alternative.  It might
be worthwhile discriminating against ultra-expensive functions such as ERROR.

For this to work, we have to detect when one of the options is empty.  In this
case, the next for one branch is a successor of the other branch, making the
comparison meaningless.  We use dominator information to detect this situation.
When a branch is empty, one of the predecessors of the first block in the empty
branch will be dominated by the first block in the other branch.  In such a
case we favor the empty branch, since that's about as cheap as you can get.

Statically minimizing branches is really a much more tractable problem, but
what literature there is makes it look hard.  Clearly the thing to do is to use
a non-optimal heuristic algorithm.

A good possibility is to use an algorithm based on the depth first ordering.
We can modify the basic DFO algorithm so that it chooses an ordering which
favors any drop-thrus that we may choose for dynamic reasons.  When we are
walking the graph, we walk the desired drop-thru arc last, which will place it
immediately after us in the DFO unless the arc is a retreating arc.

We scan through the DFO and whenever we find a block that hasn't been done yet,
we build a straight-line segment by setting the drop-thru to the unreached
successor block which has the lowest DFN greater than that for the block.  We
move to the drop-thru block and repeat the process until there is no such
block.  We then go back to our original scan through the DFO, looking for the
head of another straight-line segment.

This process will automagically implement all of the dynamic optimizations
described above as long as we favor the appropriate IF branch when creating the
DFO.  Using the DFO will prevent us from making the back branch in a loop the
drop-thru, but we need to be clever about favoring IF branches within loops
while computing the DFO.  The IF join will be favored without any special
effort, since we follow through the most favored path until we reach the end.

This needs some knowledge about the target machine, since on most machines
non-tail-recursive calls will use some sort of call instruction.  In this case,
the call actually wants to drop through to the return point, rather than
dropping through to the beginning of the called function.


\chapter{VMR conversion}

\#|
Single-use let var continuation substitution not really correct, since it can
cause a spurious type error.  Maybe we do want stuff to prove that an NLX can't
happen after all.  Or go back to the idea of moving a combination arg to the
ref location, and having that use the ref cont (with its output assertion.)
This lossage doesn't seem very likely to actually happen, though.
[\#\#\# must-reach stuff wouldn't work quite as well as combination substitute in
psetq, etc., since it would fail when one of the new values is random code
(might unwind.)]

Is this really a general problem with eager type checking?  It seems you could
argue that there was no type error in this code:

\begin{verbatim}
      (+ :foo (throw 'up nil))
\end{verbatim}

But we would signal an error.


Emit explicit you-lose operation when we do a move between two non-T ptypes,
even when type checking isn't on.  Can this really happen?  Seems we should
treat continuations like this as though type-check was true.  Maybe LTN should
leave type-check true in this case, even when the policy is unsafe.  (Do a type
check against NIL?)

At continuation use time, we may in general have to do both a coerce-to-t and a
type check, allocating two temporary TNs to hold the intermediate results.


\section{VMR Control representation}

We represent all control transfer explicitly.  In particular, :Conditional VOPs
take a single Target continuation and a Not-P flag indicating whether the sense
of the test is negated.  Then an unconditional Branch VOP will be emitted
afterward if the other path isn't a drop-through.

So we linearize the code before VMR-conversion.  This isn't a problem,
since there isn't much change in control flow after VMR conversion (none until
loop optimization requires introduction of header blocks.)  It does make
cost-based branch prediction a bit ucky, though, since we don't have any cost
information in ICR.  Actually, I guess we do have pretty good cost information
after LTN even before VMR conversion, since the most important thing to know is
which functions are open-coded.

|\#

VMR preserves the block structure of ICR, but replaces the nodes with a target
dependent virtual machine (VM) representation.  Different implementations may
use different VMs without making major changes in the back end.  The two main
components of VMR are Temporary Names (TNs) and Virtual OPerations (VOPs).  TNs
represent the locations that hold values, and VOPs represent the operations
performed on the values.

A ``primitive type'' is a type meaningful at the VM level.  Examples are Fixnum,
String-Char, Short-Float.  During VMR conversion we use the primitive type of
an expression to determine both where we can store the result of the expression
and which type-specific implementations of an operation can be applied to the
value.  [Ptype is a set of SCs == representation choices and representation
specific operations]

The VM specific definitions provide functions that do stuff like find the
primitive type corresponding to a type and test for primitive type subtypep.
Usually primitive types will be disjoint except for T, which represents all
types.

The primitive type T is special-cased.  Not only does it overlap with all the
other types, but it implies a descriptor (``boxed'' or ``pointer'') representation.
For efficiency reasons, we sometimes want to use
alternate representations for some objects such as numbers.  The majority of
operations cannot exploit alternate representations, and would only be
complicated if they had to be able to convert alternate representations into
descriptors.  A template can require an operand to be a descriptor by
constraining the operand to be of type T.

A TN can only represent a single value, so we bare the implementation of MVs at
this point.  When we know the number of multiple values being handled, we use
multiple TNs to hold them.  When the number of values is actually unknown, we
use a convention that is compatible with full function call.

Everything that is done is done by a VOP in VMR.  Calls to simple primitive
functions such as + and CAR are translated to VOP equivalents by a table-driven
mechanism.  This translation is specified by the particular VM definition; VMR
conversion makes no assumptions about which operations are primitive or what
operand types are worth special-casing.  The default calling mechanisms and
other miscellaneous builtin features are implemented using standard VOPs that
must be implemented by each VM.

Type information can be forgotten after VMR conversion, since all type-specific
operation selections have been made.

Simple type checking is explicitly done using CHECK-xxx VOPs.  They act like
innocuous effectless/unaffected VOPs which return the checked thing as a
result.  This allows loop-invariant optimization and common subexpression
elimination to remove redundant checks.  All type checking is done at the time
the continuation is used.

Note that we need only check asserted types, since if type inference works, the
derived types will also be satisfied.  We can check whichever is more
convenient, since both should be true.

Constants are turned into special Constant TNs, which are wired down in a SC
that is determined by their type.  The VM definition provides a function that
returns a constant TN to represent a Constant Leaf. 

Each component has a constant pool.  There is a register dedicated to holding
the constant pool for the current component.  The back end allocates
non-immediate constants in the constant pool when it discovers them during
translation from ICR.

[\#\#\# Check that we are describing what is actually implemented.  But this
really isn't very good in the presence of interesting unboxed
representations...] 
Since LTN only deals with values from the viewpoint of the receiver, we must be
prepared during the translation pass to do stuff to the continuation at the
time it is used.
 -- If a VOP yields more values than are desired, then we must create TNs to
    hold the discarded results.  An important special-case is continuations
    whose value is discarded.  These continuations won't be annotated at all.
    In the case of a Ref, we can simply skip evaluation of the reference when
    the continuation hasn't been annotated.  Although this will eliminate
    bogus references that for some reason weren't optimized away, the real
    purpose is to handle deferred references.
 -- If a VOP yields fewer values than desired, then we must default the extra
    values to NIL.
 -- If a continuation has its type-check flag set, then we must check the type
    of the value before moving it into the result location.  In general, this
    requires computing the result in a temporary, and having the type-check
    operation deliver it in the actual result location.
 -- If the template's result type is T, then we must generate a boxed
    temporary to compute the result in when the continuation's type isn't T.


We may also need to do stuff to the arguments when we generate code for a
template.  If an argument continuation isn't annotated, then it must be a
deferred reference.  We use the leaf's TN instead.  We may have to do any of
the above use-time actions also.  Alternatively, we could avoid hair by not
deferring references that must be type-checked or may need to be boxed.


\section{Stack analysis}

Think of this as a lifetime problem: a values generator is a write and a values
receiver is a read.  We want to annotate each VMR-Block with the unknown-values
continuations that are live at that point.  If we do a control transfer to a
place where fewer continuations are live, then we must deallocate the newly
dead continuations.

We want to convince ourselves that values deallocation based on lifetime
analysis actually works.  In particular, we need to be sure that it doesn't
violate the required stack discipline.  It is clear that it is impossible to
deallocate the values before they become dead, since later code may decide to
use them.  So the only thing we need to ensure is that the ``right'' time isn't
later than the time that the continuation becomes dead.

The only reason why we couldn't deallocate continuation A as soon as it becomes
dead would be that there is another continuation B on top of it that isn't dead
(since we can only deallocate the topmost continuation).

The key to understanding why this can't happen is that each continuation has
only one read (receiver).  If B is on top of A, then it must be the case that A
is live at the receiver for B.  This means that it is impossible for B to be
live without A being live.


The reason that we don't solve this problem using a normal iterative flow
analysis is that we also need to know the ordering of the continuations on the
stack so that we can do deallocation.  When it comes time to discard values, we
want to know which discarded continuation is on the bottom so that we can reset
SP to its start.  

[I suppose we could also decrement SP by the aggregate size of the discarded
continuations.]  Another advantage of knowing the order in which we expect
continuations to be on the stack is that it allows us to do some consistency
checking.  Also doing a localized graph walk around the values-receiver is
likely to be much more efficient than doing an iterative flow analysis problem
over all the code in the component (not that big a consideration.)



\#|
Actually, what we do is a backward graph walk from each unknown-values
receiver.   As we go, we mark each walked block with the ordered list of
continuations we believe are on the stack.  Starting with an empty stack, we:
 -- When we encounter another unknown-values receiver, we push that
    continuation on our simulated stack.
 -- When we encounter a receiver (which had better be for the topmost
    continuation), we pop that continuation.
 -- When we pop all continuations, we terminate our walk.

[\#\#\# not quite right...  It seems we may run into ``dead values'' during the
graph walk too.  It seems that we have to check if the pushed continuation is
on stack top, and if not, add it to the ending stack so that the post-pass will
discard it.]



[\#\#\# Also, we can't terminate our walk just because we hit a block previously
walked.  We have to compare the End-Stack with the values received along
the current path: if we have more values on our current walk than on the walk
that last touched the block, then we need to re-walk the subgraph reachable
from that block, using our larger set of continuations.  It seems that our
actual termination condition is reaching a block whose End-Stack is already EQ
to our current stack.]





If at the start, the block containing the values receiver has already been
walked, we skip the walk for that continuation, since it has already been
handled by an enclosing values receiver.  Once a walk has started, we
ignore any signs of a previous walk, clobbering the old result with our own,
since we enclose that continuation, and the previous walk doesn't take into
consideration the fact that our values block underlies its own.

When we are done, we have annotated each block with the stack current both at
the beginning and at the end of that block.  Blocks that aren't walked don't
have anything on the stack either place (although they may hack MVs
internally).  

We then scan all the blocks in the component, looking for blocks that have
predecessors with a different ending stack than that block's starting stack.
(The starting stack had better be a tail of the predecessor's ending stack.)
We insert a block intervening between all of these predecessors that sets SP to
the end of the values for the continuation that should be on stack top.  Of
course, this pass needn't be done if there aren't any global unknown MVs.

Also, if we find any block that wasn't reached during the walk, but that USEs
an outside unknown-values continuation, then we know that the DEST can't be
reached from this point, so the values are unused.  We either insert code to
pop the values, or somehow mark the code to prevent the values from ever being
pushed.  (We could cause the popping to be done by the normal pass if we
iterated over the pushes beforehand, assigning a correct END-STACK.)

[\#\#\# But I think that we have to be a bit clever within blocks, given the
possibility of blocks being joined.  We could collect some unknown MVs in a
block, then do a control transfer out of the receiver, and this control
transfer could be squeezed out by merging blocks.  How about:

\begin{verbatim}
    (tagbody
      (return
       (multiple-value-prog1 (foo)
	 (when bar
	   (go UNWIND))))

     UNWIND
      (return
       (multiple-value-prog1 (baz)
	 bletch)))
\end{verbatim}

But the problem doesn't happen here (can't happen in general?) since a node
buried within a block can't use a continuation outside of the block.  In fact,
no block can have more then one PUSH continuation, and this must always be the
last continuation.  So it is trivially (structurally) true that all pops come
before any push.

[\#\#\# But not really: the DEST of an embedded continuation may be outside the
block.  There can be multiple pushes, and we must find them by iterating over
the uses of MV receivers in LTN.  But it would be hard to get the order right
this way.  We could easily get the order right if we added the generators as we
saw the uses, except that we can't guarantee that the continuations will be
annotated at that point.  (Actually, I think we only need the order for
consistency checks, but that is probably worthwhile).  I guess the thing to do
is when we process the receiver, add the generator blocks to the
Values-Generators, then do a post-pass that re-scans the blocks adding the
pushes.]

I believe that above concern with a dead use getting mashed inside a block
can't happen, since the use inside the block must be the only use, and if the
use isn't reachable from the push, then the use is totally unreachable, and
should have been deleted, which would prevent it from ever being
annotated.
]
]
|\#

We find the partial ordering of the values globs for unknown values
continuations in each environment.  We don't have to scan the code looking for
unknown values continuations since LTN annotates each block with the
continuations that were popped and not pushed or pushed and not popped.  This
is all we need to do the inter-block analysis.

After we have found out what stuff is on the stack at each block boundary, we
look for blocks with predecessors that have junk on the stack.  For each such
block, we introduce a new block containing code to restore the stack pointer.
Since unknown-values continuations are represented as \verb+<start, count>+, we can
easily pop a continuation using the Start TN.

Note that there is only doubt about how much stuff is on the control stack,
since only it is used for unknown values.  Any special stacks such as number
stacks will always have a fixed allocation.


\section{Non-local exit}


If the starting and ending continuations are not in the same environment, then
the control transfer is a non-local exit.  In this case just call Unwind with
the appropriate stack pointer, and let the code at the re-entry point worry
about fixing things up.

It seems like maybe a good way to organize VMR conversion of NLX would be to
have environment analysis insert funny functions in new interposed cleanup
blocks.  The thing is that we need some way for VMR conversion to:
 1] Get its hands on the returned values.
 2] Do weird control shit.
 3] Deliver the values to the original continuation destination.
I.e. we need some way to interpose arbitrary code in the path of value
delivery.

What we do is replace the NLX uses of the continuation with another
continuation that is received by a MV-Call to \%NLX-VALUES in a cleanup block
that is interposed between the NLX uses and the old continuation's block.  The
MV-Call uses the original continuation to deliver its values to.  

[Actually, it's not really important that this be an MV-Call, since it has to
be special-cased by LTN anyway.  Or maybe we would want it to be an MV call.
If we did normal LTN analysis of an MV call, it would force the returned values
into the unknown values convention, which is probably pretty convenient for use
in NLX.

Then the entry code would have to use some special VOPs to receive the unknown
values.  But we probably need special VOPs for NLX entry anyway, and the code
can share with the call VOPs.  Also we probably need the technology anyway,
since THROW will use truly unknown values.]


On entry to a dynamic extent that has non-local-exists into it (always at an
ENTRY node), we take a complete snapshot of the dynamic state:

\begin{itemize}
\item the top pointers for all stacks
\item current Catch and Unwind-Protect
\item current special binding (binding stack pointer in shallow binding)
\end{itemize}

We insert code at the re-entry point which restores the saved dynamic state.
All TNs live at an NLX EP are forced onto the stack, so we don't have to restore
them, and we don't have to worry about getting them saved.


% -*- Dictionary: design -*-

\chapter{Copy propagation}

File: {\tt copyprop}

This phase is optional, but should be done whenever speed or space is more
important than compile speed.  We use global flow analysis to find the reaching
definitions for each TN.  This information is used here to eliminate
unnecessary TNs, and is also used later on by loop invariant optimization.

In some cases, VMR conversion will unnecessarily copy the value of a TN into
another TN, since it may not be able to tell that the initial TN has the same
value at the time the second TN is referenced.  This can happen when ICR
optimize is unable to eliminate a trivial variable binding, or when the user
does a setq, or may also result from creation of expression evaluation
temporaries during VMR conversion.  Whatever the cause, we would like to avoid
the unnecessary creation and assignment of these TNs.

What we do is replace TN references whose only reaching definition is a Move
VOP with a reference to the TN moved from, and then delete the Move VOP if the
copy TN has no remaining references.  There are several restrictions on copy
propagation:
\begin{itemize}
\item The TNs must be ``ordinary'' TNs, not restricted or otherwise
unusual.  Extending the life of restricted (or wired) TNs can make register
allocation impossible.  Some other TN kinds have hidden references.

\item We don't want to defeat source-level debugging by replacing named
variables with anonymous temporaries.

\item We can't delete moves that representation selected might want to change
into a representation conversion, since we need the primitive types of both TNs
to select a conversion.
\end{itemize}

Some cleverness reduces the cost of flow analysis.  As for lifetime analysis,
we only need to do flow analysis on global packed TNs.  We can't do the real
local TN assignment pass before this, since we allocate TNs afterward, so we do
a pre-pass that marks the TNs that are local for our purposes.  We don't care
if block splitting eventually causes some of them to be considered global.

Note also that we are really only interested in knowing if there is a
unique reaching definition, which we can mash into our flow analysis rules by
doing an intersection.  Then a definition only appears in the set when it is
unique.  We then propagate only definitions of TNs with only one write, which
allows the TN to stand for the definition.


\chapter{Representation selection}

File: {\tt represent}

Some types of object (such as {\tt single-float}) have multiple possible
representations.  Multiple representations are useful mainly when there is a
particularly efficient non-descriptor representation.  In this case, there is
the normal descriptor representation, and an alternate non-descriptor
representation.

This possibility brings up two major issues:
\begin{itemize}
\item The compiler must decide which representation will be most efficient for
any given value, and

\item Representation conversion code must be inserted where the representation
of a value is changed.
\end{itemize}
First, the representations for TNs are selected by examining all the TN
references and attempting to minimize reference costs.  Then representation
conversion code is introduced.

This phase is in effect a pre-pass to register allocation.  The main reason for
its existence is that representation conversions may be farily complex (e.g.
involving memory allocation), and thus must be discovered before register
allocation.


VMR conversion leaves stubs for representation specific move operations.
Representation selection recognizes {\tt move} by name.  Argument and return
value passing for call VOPs is controlled by the {\tt :move-arguments} option
to {\tt define-vop}.

Representation selection is also responsible for determining what functions use
the number stack.  If any representation is chosen which could involve packing
into the {\tt non-descriptor-stack} SB, then we allocate the NFP register
throughout the component.  As an optimization, permit the decision of whether a
number stack frame needs to be allocated to be made on a per-function basis.
If a function doesn't use the number stack, and isn't in the same tail-set as
any function that uses the number stack, then it doesn't need a number stack
frame, even if other functions in the component do.


\chapter{Lifetime analysis}

File: {\tt life}

This phase is a preliminary to Pack.  It involves three passes:
 -- A pre-pass that computes the DEF and USE sets for live TN analysis, while
    also assigning local TN numbers, splitting blocks if necessary.  \#\#\# But
not really...
 -- A flow analysis pass that does backward flow analysis on the
    component to find the live TNs at each block boundary.
 -- A post-pass that finds the conflict set for each TN.

\#|
Exploit the fact that a single VOP can only exhaust LTN numbers when there are
large more operands.  Since more operand reference cannot be interleaved with
temporary reference, the references all effectively occur at the same time.
This means that we can assign all the more args and all the more results the
same LTN number and the same lifetime info.
|\#


\section{Flow analysis}

It seems we could use the global-conflicts structures during compute the
inter-block lifetime information.  The pre-pass creates all the
global-conflicts for blocks that global TNs are referenced in.  The flow
analysis pass just adds always-live global-conflicts for the other blocks the
TNs are live in.  In addition to possibly being more efficient than SSets, this
would directly result in the desired global-conflicts information, rather than
having to create it from another representation.

The DFO sorted per-TN global-conflicts thread suggests some kind of algorithm
based on the manipulation of the sets of blocks each TN is live in (which is
what we really want), rather than the set of TNs live in each block.

If we sorted the per-TN global-conflicts in reverse DFO (which is just as good
for determining conflicts between TNs), then it seems we could scan though the
conflicts simultaneously with our flow-analysis scan through the blocks.

The flow analysis step is the following:
    If a TN is always-live or read-before-written in a successor block, then we
    make it always-live in the current block unless there are already
    global-conflicts recorded for that TN in this block.

The iteration terminates when we don't add any new global-conflicts during a
pass.

We may also want to promote TNs only read within a block to always-live when
the TN is live in a successor.  This should be easy enough as long as the
global-conflicts structure contains this kind of info.

The critical operation here is determining whether a given global TN has global
conflicts in a given block.  Note that since we scan the blocks in DFO, and the
global-conflicts are sorted in DFO, if we give each global TN a pointer to the
global-conflicts for the last block we checked the TN was in, then we can
guarantee that the global-conflicts we are looking for are always at or after
that pointer.  If we need to insert a new structure, then the pointer will help
us rapidly find the place to do the insertion.]


\section{Conflict detection}

[\#\#\# Environment, :more TNs.]

This phase makes use of the results of lifetime analysis to find the set of TNs
that have lifetimes overlapping with those of each TN.  We also annotate call
VOPs with information about the live TNs so that code generation knows which
registers need to be saved.

The basic action is a backward scan of each block, looking at each TN-Ref and
maintaining a set of the currently live TNs.  When we see a read, we check if
the TN is in the live set.  If not, we:
 -- Add the TN to the conflict set for every currently live TN,
 -- Union the set of currently live TNs with the conflict set for the TN, and
 -- Add the TN to the set of live TNs.

When we see a write for a live TN, we just remove it from the live set.  If we
see a write to a dead TN, then we update the conflicts sets as for a read, but
don't add the TN to the live set.  We have to do this so that the bogus write
doesn't clobber anything.

[We don't consider always-live TNs at all in this process, since the conflict
of always-live TNs with other TNs in the block is implicit in the
global-conflicts structures.

Before we do the scan on a block, we go through the global-conflicts structures
of TNs that change liveness in the block, assigning the recorded LTN number to
the TN's LTN number for the duration of processing of that block.]
 

Efficiently computing and representing this information calls for some
cleverness.  It would be prohibitively expensive to represent the full conflict
set for every TN with sparse sets, as is done at the block-level.  Although it
wouldn't cause non-linear behavior, it would require a complex linked structure
containing tens of elements to be created for every TN.  Fortunately we can
improve on this if we take into account the fact that most TNs are ``local'' TNs:
TNs which have all their uses in one block.

First, many global TNs will be either live or dead for the entire duration of a
given block.  We can represent the conflict between global TNs live throughout
the block and TNs local to the block by storing the set of always-live global
TNs in the block.  This reduces the number of global TNs that must be
represented in the conflicts for local TNs.

Second, we can represent conflicts within a block using bit-vectors.  Each TN
that changes liveness within a block is assigned a local TN number.  Local
conflicts are represented using a fixed-size bit-vector of 64 elements or so
which has a 1 for the local TN number of every TN live at that time.  The block
has a simple-vector which maps from local TN numbers to TNs.  Fixed-size
vectors reduce the hassle of doing allocations and allow operations to be
open-coded in a maximally tense fashion.

We can represent the conflicts for a local TN by a single bit-vector indexed by
the local TN numbers for that block, but in the global TN case, we need to be
able to represent conflicts with arbitrary TNs.  We could use a list-like
sparse set representation, but then we would have to either special-case global
TNs by using the sparse representation within the block, or convert the local
conflicts bit-vector to the sparse representation at the block end.  Instead,
we give each global TN a list of the local conflicts bit-vectors for each block
that the TN is live in.  If the TN is always-live in a block, then we record
that fact instead.  This gives us a major reduction in the amount of work we
have to do in lifetime analysis at the cost of some increase in the time to
iterate over the set during Pack.

Since we build the lists of local conflict vectors a block at a time, the
blocks in the lists for each TN will be sorted by the block number.  The
structure also contains the local TN number for the TN in that block.  These
features allow pack to efficiently determine whether two arbitrary TNs
conflict.  You just scan the lists in order, skipping blocks that are in only
one list by using the block numbers.  When we find a block that both TNs are
live in, we just check the local TN number of one TN in the local conflicts
vector of the other.

In order to do these optimizations, we must do a pre-pass that finds the
always-live TNs and breaks blocks up into small enough pieces so that we don't
run out of local TN numbers.  If we can make a block arbitrarily small, then we
can guarantee that an arbitrarily small number of TNs change liveness within
the block.  We must be prepared to make the arguments to unbounded arg count
VOPs (such as function call) always-live even when they really aren't.  This is
enabled by a panic mode in the block splitter: if we discover that the block
only contains one VOP and there are still too many TNs that aren't always-live,
then we promote the arguments (which we'd better be able to do...).

This is done during the pre-scan in lifetime analysis.  We can do this because
all TNs that change liveness within a block can be found by examining that
block: the flow analysis only adds always-live TNs.


When we are doing the conflict detection pass, we set the LTN number of global
TNs.  We can easily detect global TNs that have not been locally mapped because
this slot is initially null for global TNs and we null it out after processing
each block.  We assign all Always-Live TNs to the same local number so that we
don't need to treat references to them specially when making the scan.

We also annotate call VOPs that do register saving with the TNs that are live
during the call, and thus would need to be saved if they are packed in
registers.

We adjust the costs for TNs that need to be saved so that TNs costing more to
save and restore than to reference get packed on the stack.  We would also like
more often saved TNs to get higher costs so that they are packed in more
savable locations.


\chapter{Packing}

File: {\tt pack}

\#|

Add lifetime/pack support for pre-packed save TNs.

Fix GTN/VMR conversion to use pre-packed save TNs for old-cont and return-PC.
(Will prevent preference from passing location to save location from ever being
honored?)

We will need to make packing of passing locations smarter before we will be
able to target the passing location on the stack in a tail call (when that is
where the callee wants it.)  Currently, we will almost always pack the passing
location in a register without considering whether that is really a good idea.
Maybe we should consider schemes that explicitly understand the parallel
assignment semantics, and try to do the assignment with a minimum number of
temporaries.  We only need assignment temps for TNs that appear both as an
actual argument value and as a formal parameter of the called function.  This
only happens in self-recursive functions.

Could be a problem with lifetime analysis, though.  The write by a move-arg VOP
would look like a write in the current env, when it really isn't.  If this is a
problem, then we might want to make the result TN be an info arg rather than a
real operand.  But this would only be a problem in recursive calls, anyway.
[This would prevent targeting, but targeting across passing locations rarely
seems to work anyway.]  [\#\#\# But the :ENVIRONMENT TN mechanism would get
confused.  Maybe put env explicitly in TN, and have it only always-live in that
env, and normal in other envs (or blocks it is written in.)  This would allow
targeting into environment TNs.  

I guess we would also want the env/PC save TNs normal in the return block so
that we can target them.  We could do this by considering env TNs normal in
read blocks with no successors.  

ENV TNs would be treated totally normally in non-env blocks, so we don't have
to worry about lifetime analysis getting confused by variable initializations.
Do some kind of TN costing to determine when it is more trouble than it is
worth to allocate TNs in registers.

Change pack ordering to be less pessimal.  Pack TNs as they are seen in the LTN
map in DFO, which at least in non-block compilations has an effect something
like packing main trace TNs first, since control analysis tries to put the good
code first.  This could also reduce spilling, since it makes it less likely we
will clog all registers with global TNs.

If we pack a TN with a specified save location on the stack, pack in the
specified location.

Allow old-cont and return-pc to be kept in registers by adding a new ``keep
around'' kind of TN.  These are kind of like environment live, but are only
always-live in blocks that they weren't referenced in.  Lifetime analysis does
a post-pass adding always-live conflicts for each ``keep around'' TN to those
blocks with no conflict for that TN.  The distinction between always-live and
keep-around allows us to successfully target old-cont and return-pc to passing
locations.  MAKE-KEEP-AROUND-TN (ptype), PRE-PACK-SAVE-TN (tn scn offset).
Environment needs a KEEP-AROUND-TNS slot so that conflict analysis can find
them (no special casing is needed after then, they can be made with :NORMAL
kind).  VMR-component needs PRE-PACKED-SAVE-TNS so that conflict analysis or
somebody can copy conflict info from the saved TN.



Note that having block granularity in the conflict information doesn't mean
that a localized packing scheme would have to do all moves at block boundaries
(which would clash with the desire to have saving done as part of this
mechanism.)  All that it means is that if we want to do a move within the
block, we would need to allocate both locations throughout that block (or
something).





Load TN pack:

A location is out for load TN packing if: 

The location has TN live in it after the VOP for a result, or before the VOP
for an argument, or

The location is used earlier in the TN-ref list (after) the saved results ref
or later in the TN-Ref list (before) the loaded argument's ref.

To pack load TNs, we advance the live-tns to the interesting VOP, then
repeatedly scan the vop-refs to find vop-local conflicts for each needed load
TN.  We insert move VOPs and change over the TN-Ref-TNs as we go so the TN-Refs
will reflect conflicts with already packed load-TNs.

If we fail to pack a load-TN in the desired SC, then we scan the Live-TNs for
the SB, looking for a TN that can be packed in an unbounded SB.  This TN must
then be repacked in the unbounded SB.  It is important the load-TNs are never
packed in unbounded SBs, since that would invalidate the conflicts info,
preventing us from repacking TNs in unbounded SBs.  We can't repack in a finite
SB, since there might have been load TNs packed in that SB which aren't
represented in the original conflict structures.

Is it permissible to ``restrict'' an operand to an unbounded SC?  Not impossible
to satisfy as long as a finite SC is also allowed.  But in practice, no
restriction would probably be as good.

We assume all locations can be used when an sc is based on an unbounded sb.

]


TN-Refs are convenient structures to build the target graph out of.  If we
allocated space in every TN-Ref, then there would certainly be enough to
represent arbitrary target graphs.  Would it be enough to allocate a single
Target slot?  If there is a target path through a given VOP, then the Target of
the write ref would be the read, and vice-versa.  To find all the TNs that
target us, we look at the TN for the target of all our write refs.

We separately chain together the read refs and the write refs for a TN,
allowing easy determination of things such as whether a TN has only a single
definition or has no reads.  It would also allow easier traversal of the target
graph.
 
Represent per-location conflicts as vectors indexed by block number of
per-block conflict info.  To test whether a TN conflicts on a location, we
would then have to iterate over the TNs global-conflicts, using the block
number and LTN number to check for a conflict in that block.  But since most
TNs are local, this test actually isn't much more expensive than indexing into
a bit-vector by GTN numbers.

The big win of this scheme is that it is much cheaper to add conflicts into the
conflict set for a location, since we never need to actually compute the
conflict set in a list-like representation (which requires iterating over the
LTN conflicts vectors and unioning in the always-live TNs).  Instead, we just
iterate over the global-conflicts for the TN, using BIT-IOR to combine the
conflict set with the bit-vector for that block in that location, or marking
that block/location combination as being always-live if the conflict is
always-live.

Generating the conflict set is inherently more costly, since although we
believe the conflict set size to be roughly constant, it can easily contain
tens of elements.  We would have to generate these moderately large lists for
all TNs, including local TNs.  In contrast, the proposed scheme does work
proportional to the number of blocks the TN is live in, which is small on
average (1 for local TNs).  This win exists independently from the win of not
having to iterate over LTN conflict vectors.


[\#\#\# Note that since we never do bitwise iteration over the LTN conflict
vectors, part of the motivation for keeping these a small fixed size has been
removed.  But it would still be useful to keep the size fixed so that we can
easily recycle the bit-vectors, and so that we could potentially have maximally
tense special primitives for doing clear and bit-ior on these vectors.]

This scheme is somewhat more space-intensive than having a per-location
bit-vector.  Each vector entry would be something like 150 bits rather than one
bit, but this is mitigated by the number of blocks being 5-10x smaller than the
number of TNs.  This seems like an acceptable overhead, a small fraction of the
total VMR representation.

The space overhead could also be reduced by using something equivalent to a
two-dimensional bit array, indexed first by LTN numbers, and then block numbers
(instead of using a simple-vector of separate bit-vectors.)  This would
eliminate space wastage due to bit-vector overheads, which might be 50% or
more, and would also make efficient zeroing of the vectors more
straightforward.  We would then want efficient operations for OR'ing LTN
conflict vectors with rows in the array.

This representation also opens a whole new range of allocation algorithms: ones
that store allocate TNs in different locations within different portions of the
program.  This is because we can now represent a location being used to hold a
certain TN within an arbitrary subset of the blocks the TN is referenced in.









Pack goals:

Pack should:

Subject to resource constraints:
 -- Minimize use costs
     -- ``Register allocation''
         Allocate as many values as possible in scarce ``good'' locations,
         attempting to minimize the aggregate use cost for the entire program.
     -- ``Save optimization''
         Don't allocate values in registers when the save/restore costs exceed
         the expected gain for keeping the value in a register.  (Similar to
         ``opening costs'' in RAOC.)  [Really just a case of representation
         selection.]

 -- Minimize preference costs
    Eliminate as many moves as possible.


``Register allocation'' is basically an attempt to eliminate moves between
registers and memory.  ``Save optimization'' counterbalances ``register
allocation'' to prevent it from becoming a pessimization, since saves can
introduce register/memory moves.

Preference optimization reduces the number of moves within an SC.  Doing a good
job of honoring preferences is important to the success of the compiler, since
we have assumed in many places that moves will usually be optimized away.

The scarcity-oriented aspect of ``register allocation'' is handled by a greedy
algorithm in pack.  We try to pack the ``most important'' TNs first, under the
theory that earlier packing is more likely to succeed due to fewer constraints.

The drawback of greedy algorithms is their inability to look ahead.  Packing a
TN may mess up later ``register allocation'' by precluding packing of TNs that
are individually ``less important,'' but more important in aggregate.  Packing a
TN may also prevent preferences from being honored.



Initial packing:


Pack all TNs restricted to a finite SC first, before packing any other TNs.

One might suppose that Pack would have to treat TNs in different environments
differently, but this is not the case.  Pack simply assigns TNs to locations so
that no two conflicting TNs are in the same location.  In the process of
implementing call semantics in conflict analysis, we cause TNs in different
environments not to conflict.  In the case of passing TNs, cross environment
conflicts do exist, but this reflects reality, since the passing TNs are
live in both the caller and the callee.  Environment semantics has already been
implemented at this point.

This means that Pack can pack all TNs simultaneously, using one data structure
to represent the conflicts for each location.  So we have only one conflict set
per SB location, rather than separating this information by environment.


Load TN packing:

We create load TNs as needed in a post-pass to the initial packing.  After TNs
are packed, it may be that some references to a TN will require it to be in a
SC other than the one it was packed in.  We create load-TNs and pack them on
the fly during this post-pass.  

What we do is have an optional SC restriction associated with TN-refs.  If we
pack the TN in an SC which is different from the required SC for the reference,
then we create a TN for each such reference, and pack it into the required SC.

In many cases we will be able to pack the load TN with no hassle, but in
general we may need to spill a TN that has already been packed.  We choose a
TN that isn't in use by the offending VOP, and then spill that TN onto the
stack for the duration of that VOP.  If the VOP is a conditional, then we must
insert a new block interposed before the branch target so that the TN
value is restored regardless of which branch is taken.

Instead of remembering lifetime information from conflict analysis, we rederive
it.  We scan each block backward while keeping track of which locations have
live TNs in them.  When we find a reference that needs a load TN packed, we try
to pack it in an unused location.  If we can't, we unpack the currently live TN
with the lowest cost and force it into an unbounded SC.

The per-location and per-TN conflict information used by pack doesn't
need to be updated when we pack a load TN, since we are done using those data
structures.

We also don't need to create any TN-Refs for load TNs.  [??? How do we keep
track of load-tn lifetimes?  It isn't really that hard, I guess.  We just
remember which load TNs we created at each VOP, killing them when we pass the
loading (or saving) step.  This suggests we could flush the Refs thread if we
were willing to sacrifice some flexibility in explicit temporary lifetimes.
Flushing the Refs would make creating the VMR representation easier.]

The lifetime analysis done during load-TN packing doubles as a consistency
check.  If we see a read of a TN packed in a location which has a different TN
currently live, then there is a packing bug.  If any of the TNs recorded as
being live at the block beginning are packed in a scarce SB, but aren't current
in that location, then we also have a problem.

The conflict structure for load TNs is fairly simple, the load TNs for
arguments and results all conflict with each other, and don't conflict with
much else.  We just try packing in targeted locations before trying at random.



\chapter{Code generation}

This is fairly straightforward.  We translate VOPs into instruction sequences
on a per-block basis.

After code generation, the VMR representation is gone.  Everything is
represented by the assembler data structures.


\chapter{Assembly}

In effect, we do much of the work of assembly when the compiler is compiled.

The assembler makes one pass fixing up branch offsets, then squeezes out the
space left by branch shortening and dumps out the code along with the load-time
fixup information.  The assembler also deals with dumping unboxed non-immediate
constants and symbols.  Boxed constants are created by explicit constructor
code in the top-level form, while immediate constants are generated using
inline code.

[\#\#\# The basic output of the assembler is:
    A code vector
    A representation of the fixups along with indices into the code vector for
      the fixup locations
    A PC map translating PCs into source paths

This information can then be used to build an output file or an in-core
function object.
]

The assembler is table-driven and supports arbitrary instruction formats.  As
far as the assembler is concerned, an instruction is a bit sequence that is
broken down into subsequences.  Some of the subsequences are constant in value,
while others can be determined at assemble or load time.

\begin{verbatim}
Assemble Node Form*
    Allow instructions to be emitted during the evaluation of the Forms by
    defining Inst as a local macro.  This macro caches various global
    information in local variables.  Node tells the assembler what node
    ultimately caused this code to be generated.  This is used to create the
    pc=>source map for the debugger.

Assemble-Elsewhere Node Form*
    Similar to Assemble, but the current assembler location is changed to
    somewhere else.  This is useful for generating error code and similar
    things.  Assemble-Elsewhere may not be nested.

Inst Name Arg*
    Emit the instruction Name with the specified arguments.

Gen-Label
Emit-Label (Label)
    Gen-Label returns a Label object, which describes a place in the code.
    Emit-Label marks the current position as being the location of Label.
\end{verbatim}



\chapter{Dumping}

So far as input to the dumper/loader, how about having a list of Entry-Info
structures in the VMR-Component?  These structures contain all information
needed to dump the associated function objects, and are only implicitly
associated with the functional/XEP data structures.  Load-time constants that
reference these function objects should specify the Entry-Info, rather than the
functional (or something).  We would then need to maintain some sort of
association so VMR conversion can find the appropriate Entry-Info.
Alternatively, we could initially reference the functional, and then later
clobber the reference to the Entry-Info.

We have some kind of post-pass that runs after assembly, going through the
functions and constants, annotating the VMR-Component for the benefit of the
dumper:
    Resolve :Label load-time constants.
    Make the debug info.
    Make the entry-info structures.

Fasl dumper and in-core loader are implementation (but not instruction set)
dependent, so we want to give them a clear interface.

\begin{verbatim}
open-fasl-file name => fasl-file
    Returns a ``fasl-file'' object representing all state needed by the dumper.
    We objectify the state, since the fasdumper should be reentrant.  (but
    could fail to be at first.)

close-fasl-file fasl-file abort-p
    Close the specified fasl-file.

fasl-dump-component component code-vector length fixups fasl-file
    Dump the code, constants, etc. for component.  Code-Vector is a vector
    holding the assembled code.  Length is the number of elements of Vector
    that are actually in use.  Fixups is a list of conses (offset . fixup)
    describing the locations and things that need to be fixed up at load time.
    If the component is a top-level component, then the top-level lambda will
    be called after the component is loaded.

load-component component code-vector length fixups
    Like Fasl-Dump-Component, but directly installs the code in core, running
    any top-level code immediately.  (???) but we need some way to glue
    together the componenents, since we don't have a fasl table.
\end{verbatim}



Dumping:

Dump code for each component after compiling that component, but defer dumping
of other stuff.  We do the fixups on the code vectors, and accumulate them in
the table.

We have to grovel the constants for each component after compiling that
component so that we can fix up load-time constants.  Load-time constants are
values needed by the code that are computed after code generation/assembly
time.  Since the code is fixed at this point, load-time constants are always
represented as non-immediate constants in the constant pool.  A load-time
constant is distinguished by being a cons (Kind . What), instead of a Constant
leaf.  Kind is a keyword indicating how the constant is computed, and What is
some context.

Some interesting load-time constants:
\begin{verbatim}
    (:label . <label>)
        Is replaced with the byte offset of the label within the code-vector.

    (:code-vector . <component>)
        Is replaced by the component's code-vector.

    (:entry . <function>)
    (:closure-entry . <function>)
	Is replaced by the function-entry structure for the specified function.
	:Entry is how the top-level component gets a handle on the function
	definitions so that it can set them up.
\end{verbatim}
We also need to remember the starting offset for each entry, although these
don't in general appear as explicit constants.

We then dump out all the :Entry and :Closure-Entry objects, leaving any
constant-pool pointers uninitialized.  After dumping each :Entry, we dump some
stuff to let genesis know that this is a function definition.  Then we dump all
the constant pools, fixing up any constant-pool pointers in the already-dumped
function entry structures.

The debug-info *is* a constant: the first constant in every constant pool.  But
the creation of this constant must be deferred until after the component is
compiled, so we leave a (:debug-info) placeholder.  [Or maybe this is
implicitly added in by the dumper, being supplied in a VMR-component slot.]


    Work out details of the interface between the back-end and the
    assembler/dumper.

    Support for multiple assemblers concurrently loaded?  (for byte code)
    
    We need various mechanisms for getting information out of the assembler.

    We can get entry PCs and similar things into function objects by making a
    Constant leaf, specifying that it goes in the closure, and then
    setting the value after assembly.

    We have an operation Label-Value which can be used to get the value of a
    label after assembly and before the assembler data structures are
    deallocated.

    The function map can be constructed without any special help from the
    assembler.  Codegen just has to note the current label when the function
    changes from one block to the next, and then use the final value of these
    labels to make the function map.

    Probably we want to do the source map this way too.  Although this will
    make zillions of spurious labels, we would have to effectively do that
    anyway.

    With both the function map and the source map, getting the locations right
    for uses of Elsewhere will be a bit tricky.  Users of Elsewhere will need
    to know about how these maps are being built, since they must record the
    labels and corresponding information for the elsewhere range.  It would be
    nice to have some cooperation from Elsewhere so that this isn't necessary,
    otherwise some VOP writer will break the rules, resulting in code that is
    nowhere.

    The Debug-Info and related structures are dumped by consing up the
    structure and making it be the value of a constant.

    Getting the code vector and fixups dumped may be a bit more interesting.  I
    guess we want a Dump-Code-Vector function which dumps the code and fixups
    accumulated by the current assembly, returning a magic object that will
    become the code vector when it is dumped as a constant.
]

\chapter{User Interface of the Compiler}

\section{Error Message Utilities}

\section{Source Paths}



\part{Compiler Retargeting}

\chapter{Retargeting the Compiler}
[\#\#\#

In general, it is a danger sign if a generator references a TN that isn't an
operand or temporary, since lifetime analysis hasn't been done for that use.
We are doing weird stuff for the old-cont and return-pc passing locations,
hoping that the conflicts at the called function have the desired effect.
Other stuff?  When a function returns unknown values, we don't reference the
values locations when a single-value return is done.  But nothing is live at a
return point anyway.



Have a way for template conversion to special-case constant arguments?  
How about:
    If an arg restriction is (:satisfies [$<$predicate function$>$]), and the
    corresponding argument is constant, with the constant value satisfying the
    predicate, then (if any other restrictions are satisfied), the template
    will be emitted with the literal value passed as an info argument.  If the
    predicate is omitted, then any constant will do.

    We could sugar this up a bit by allowing (:member $<$object$>$*) for
    (:satisfies (lambda (x) (member x '($<$object$>$*))))

We could allow this to be translated into a Lisp type by adding a new Constant
type specifier.  This could only appear as an argument to a function type.
To satisfy (Constant $<$type$>$), the argument must be a compile-time constant of
the specified type.  Just Constant means any constant (i.e. (Constant *)).
This would be useful for the type constraints on ICR transforms.


Constant TNs: we count on being able to indirect to the leaf, and don't try to
wedge the information into the offset.  We set the FSC to an appropriate
immediate SC.

    Allow ``more operands'' to VOPs in define-vop.  You can't do much with the
    more operands: define-vop just fills in the cost information according to
    the loading costs for a SC you specify.  You can't restrict more operands,
    and you can't make local preferences.  In the generator, the named variable
    is bound to the TN-ref for the first extra operand.  This should be good
    enough to handle all the variable arg VOPs (primarily function call and
    return).  Usually more operands are used just to get TN lifetimes to work
    out; the generator actually ignores them.

    Variable-arg VOPs can't be used with the VOP macro.  You must use VOP*.
    VOP* doesn't do anything with these extra operand except stick them on the
    ends of the operand lists passed into the template.  VOP* is often useful
    within the convert functions for non-VOP templates, since it can emit a VOP
    using an already prepared TN-Ref list.
    

    It is pretty basic to the whole primitive-type idea that there is only one
    primitive-type for a given lisp type.  This is really the same as saying
    primitive types are disjoint.  A primitive type serves two somewhat
    unrelated purposes:
     -- It is an abstraction of a Lisp type used to select type specific
        operations.  Originally kind of an efficiency hack, but it lets a
        template's type signature be used both for selection and operand
        representation determination.
     -- It represents a set of possible representations for a value (SCs).  The
        primitive type is used to determine the legal SCs for a TN, and is also
        used to determine which type-coercion/move VOP to use.

]

There are basically three levels of target dependence:

 -- Code in the ``front end'' (before VMR conversion) deals only with Lisp
    semantics, and is totally target independent.

 -- Code after VMR conversion and before code generation depends on the VM,
    but should work with little modification across a wide range of
    ``conventional'' architectures.

 -- Code generation depends on the machine's instruction set and other
    implementation details, so it will have to be redone for each
    implementation.  Most of the work here is in defining the translation into
    assembly code of all the supported VOPs.



\chapter{Storage bases and classes}
New interface: instead of CURRENT-FRAME-SIZE, have CURRENT-SB-SIZE \verb+<name>+ which
returns the current element size of the named SB.

How can we have primitive types that overlap, i.e. (UNSIGNED-BYTE 32),
(SIGNED-BYTE 32), FIXNUM?
Primitive types are used for two things:
    Representation selection: which SCs can be used to represent this value?
	For this purpose, it isn't necessary that primitive types be disjoint,
	since any primitive type can choose an arbitrary set of
	representations.  For moves between the overlapping representations,
	the move/load operations can just be noops when the locations are the
	same (vanilla MOVE), since any bad moves should be caught out by type
	checking.
    VOP selection:
	Is this operand legal for this VOP?  When ptypes overlap in interesting
	ways, there is a problem with allowing just a simple ptype restriction,
	since we might want to allow multiple ptypes.  This could be handled
	by allowing ``union primitive types'', or by allowing multiple primitive
	types to be specified (only in the operand restriction.)  The latter
	would be along the lines of other more flexible VOP operand restriction
	mechanisms, (constant, etc.)



Ensure that load/save-operand never need to do representation conversion.

The PRIMITIVE-TYPE more/coerce info would be moved into the SC.  This could
perhaps go along with flushing the TN-COSTS.  We would annotate the TN with
best SC, which implies the representation (boxed or unboxed).  We would still
need to represent the legal SCs for restricted TNs somehow, and also would have to
come up with some other way for pack to keep track of which SCs we have already
tried.

An SC would have a list of ``alternate'' SCs and a boolean SAVE-P value that
indicates it needs to be saved across calls in some non-SAVE-P SC.  A TN is
initially given its ``best'' SC.  The SC is annotated with VOPs that are used for
moving between the SC and its alternate SCs (load/save operand, save/restore
register).  It is also annotated with the ``move'' VOPs used for moving between
this SC and all other SCs it is possible to move between.  We flush the idea
that there is only c-to-t and c-from-t.

But how does this mesh with the idea of putting operand load/save back into the
generator?  Maybe we should instead specify a load/save function?  The
load/save functions would also differ from the move VOPs in that they would
only be called when the TN is in fact in that particular alternate SC, whereas
the move VOPs will be associated with the primary SC, and will be emitted
before it is known whether the TN will be packed in the primary SC or an
alternate.

I guess a packed SC could also have immediate SCs as alternate SCs, and
constant loading functions could be associated with SCs using this mechanism.

So given a TN packed in SC X and an SC restriction for Y and Z, how do we know
which load function to call?  There would be ambiguity if X was an alternate
for both Y and Z and they specified different load functions.  This seems
unlikely to arise in practice, though, so we could just detect the ambiguity
and give an error at define-vop time.  If they are doing something totally
weird, they can always inhibit loading and roll their own.

Note that loading costs can be specified at the same time (same syntax) as
association of loading functions with SCs.  It seems that maybe we will be
rolling DEFINE-SAVE-SCS and DEFINE-MOVE-COSTS into DEFINE-STORAGE-CLASS.

Fortunately, these changes will affect most VOP definitions very little.


A Storage Base represents a physical storage resource such as a register set or
stack frame.  Storage bases for non-global resources such as the stack are
relativized by the environment that the TN is allocated in.  Packing conflict
information is kept in the storage base, but non-packed storage resources such
as closure environments also have storage bases.
Some storage bases:
\begin{verbatim}
    General purpose registers
    Floating point registers
    Boxed (control) stack environment
    Unboxed (number) stack environment
    Closure environment
\end{verbatim}

A storage class is a potentially arbitrary set of the elements in a storage
base.  Although conceptually there may be a hierarchy of storage classes such
as ``all registers'', ``boxed registers'', ``boxed scratch registers'', this doesn't
exist at the implementation level.  Such things can be done by specifying
storage classes whose locations overlap.  A TN shouldn't have lots of
overlapping SC's as legal SC's, since time would be wasted repeatedly
attempting to pack in the same locations.

There will be some SC's whose locations overlap a great deal, since we get Pack
to do our representation analysis by having lots of SC's.  An SC is basically a
way of looking at a storage resource.  Although we could keep a fixnum and an
unboxed representation of the same number in the same register, they correspond
to different SC's since they are different representation choices.

TNs are annotated with the primitive type of the object that they hold:
    T: random boxed object with only one representation.
    Fixnum, Integer, XXX-Float: Object is always of the specified numeric type.
    String-Char: Object is always a string-char.

When a TN is packed, it is annotated with the SC it was packed into.  The code
generator for a VOP must be able to uniquely determine the representation of
its operands from the SC. (debugger also...)

Some SCs:
    Reg: any register (immediate objects)
    Save-Reg: a boxed register near r15 (registers easily saved in a call)
    Boxed-Reg: any boxed register (any boxed object)
    Unboxed-Reg: any unboxed register (any unboxed object)
    Float-Reg, Double-Float-Reg: float in FP register.
    Stack: boxed object on the stack (on cstack)
    Word: any 32bit unboxed object on nstack.
    Double: any 64bit unboxed object on nstack.

We have a number of non-packed storage classes which serve to represent access
costs associated with values that are not allocated using conflicts
information.  Non-packed TNs appear to already be packed in the appropriate
storage base so that Pack doesn't get confused.  Costs for relevant non-packed
SC's appear in the TN-Ref cost information, but need not ever be summed into
the TN cost vectors, since TNs cannot be packed into them.

There are SCs for non-immediate constants and for each significant kind of
immediate operand in the architecture.  On the RT, 4, 8 and 20 bit integer SCs
are probably worth having.

\begin{verbatim}
Non-packed SCs:
    Constant
    Immediate constant SCs:
        Signed-Byte-<N>, Unsigned-Byte-<N>, for various architecture dependent
	    values of <N>
	String-Char
	XXX-Float
	Magic values: T, NIL, 0.
\end{verbatim}

\chapter{Type system parameterization}

The main aspect of the VM that is likely to vary for good reason is the type
system:

 -- Different systems will have different ways of representing dynamic type
    information.  The primary effect this has on the compiler is causing VMR
    conversion of type tests and checks to be implementation dependent.
    Rewriting this code for each implementation shouldn't be a big problem,
    since the portable semantics of types has already been dealt with.

 -- Different systems will have different specialized number and array types,
    and different VOPs specialized for these types.  It is easy to add this kind
    of knowledge without affecting the rest of the compiler.  All you have to
    do is define the VOPs and translations.

 -- Different systems will offer different specialized storage resources
    such as floating-point registers, and will have additional kinds of
    primitive-types.  The storage class mechanism handles a large part of this,
    but there may be some problem in getting VMR conversion to realize the
    possibly large hidden costs in implicit moves to and from these specialized
    storage resources.  Probably the answer is to have some sort of general
    mechanism for determining the primitive-type for a TN given the Lisp type,
    and then to have some sort of mechanism for automatically using specialized
    Move VOPs when the source or destination has some particular primitive-type.

\#|
How to deal with list/null(symbol)/cons in primitive-type structure?  Since
cons and symbol aren't used for type-specific template selection, it isn't
really all that critical.  Probably Primitive-Type should return the List
primitive type for all of Cons, List and Null (indicating when it is exact).
This would allow type-dispatch for simple sequence functions (such as length)
to be done using the standard template-selection mechanism.  [Not a wired
assumption] 
|\#



\chapter{VOP Definition}

Before the operand TN-refs are passed to the emit function, the following
stuff is done:
 -- The refs in the operand and result lists are linked together in order using
    the Across slot.  This list is properly NIL terminated.
 -- The TN slot in each ref is set, and the ref is linked into that TN's refs
    using the Next slot.
 -- The Write-P slot is set depending on whether the ref is an argument or
    result.
 -- The other slots have the default values.

The template emit function fills in the Vop, Costs, Cost-Function,
SC-Restriction and Preference slots, and links together the Next-Ref chain as
appropriate.


\section{Lifetime model}

\#|
Note in doc that the same TN may not be used as both a more operand and as any
other operand to the same VOP, to simplify more operand LTN number coalescing.
|\#

It seems we need a fairly elaborate model for intra-VOP conflicts in order to
allocate temporaries without introducing spurious conflicts.  Consider the
important case of a VOP such as a miscop that must have operands in certain
registers.  We allocate a wired temporary, create a local preference for the
corresponding operand, and move to (or from) the temporary.  If all temporaries
conflict with all arguments, the result will be correct, but arguments could
never be packed in the actual passing register.  If temporaries didn't conflict
with any arguments, then the temporary for an earlier argument might get packed
in the same location as the operand for a later argument; loading would then
destroy an argument before it was read.

A temporary's intra-VOP lifetime is represented by the times at which its life
starts and ends.  There are various instants during the evaluation that start
and end VOP lifetimes.  Two TNs conflict if the live intervals overlap.
Lifetimes are open intervals: if one TN's lifetime begins at a point where
another's ends, then the TNs don't conflict.

The times within a VOP are the following:

:Load
    This is the beginning of the argument's lives, as far as intra-vop
    conflicts are concerned.  If load-TNs are allocated, then this is the
    beginning of their lives.

(:Argument $<$n$>$)
    The point at which the N'th argument is read for the last time (by this
    VOP).  If the argument is dead after this VOP, then the argument becomes
    dead at this time, and may be reused as a temporary or result load-TN.

(:Eval $<$n$>$)
    The N'th evaluation step.  There may be any number of evaluation steps, but
    it is unlikely that more than two are needed.

(:Result $<$n$>$) 
    The point at which the N'th result is first written into.  This is the
    point at which that result becomes live.

:Save
    Similar to :Load, but marks the end of time.  This is the point at which result
    load-TNs are stored back to the actual location.

In any of the list-style time specifications, the keyword by itself stands for
the first such time, i.e.
\begin{verbatim}
    :argument  <==>  (:argument 0)
\end{verbatim}

Note that argument/result read/write times don't actually have to be in the
order specified, but they must *appear* to happen in that order as far as
conflict analysis is concerned.  For example, the arguments can be read in any
order as long as no TN is written that has a life beginning at or after
(:Argument $<$n$>$), where N is the number of an argument whose reading was
postponed.

[\#\#\# (???)

We probably also want some syntactic sugar in Define-VOP for automatically
moving operands to/from explicitly allocated temporaries so that this kind of
thing is somewhat easy.  There isn't really any reason to consider the
temporary to be a load-TN, but we want to compute costs as though it was and
want to use the same operand loading routines.

We also might consider allowing the lifetime of an argument/result to be
extended forward/backward.  This would in many cases eliminate the need for
temporaries when operands are read/written out of order.
]


\section{VOP Cost model}

Note that in this model, if an operand has no restrictions, it has no cost.
This makes sense, since the purpose of the cost is to indicate the
relative value of packing in different SCs.  If the operand isn't required to
be in a good SC (i.e. a register), then we might as well leave it in memory.
The SC restriction mechanism can be used even when doing a move into the SC is
too complex to be generated automatically (perhaps requiring temporary
registers), since Define-VOP allows operand loading to be done explicitly.


\section{Efficiency notes}

  In addition to
being used to tell whether a particular unsafe template might get emitted, we
can also use it to give better efficiency notes:
 -- We can say what is wrong with the call types, rather than just saying we
    failed to open-code.
 -- We can tell whether any of the ``better'' templates could possibly apply,
    i.e. is the inapplicability of a template because of inadequate type
    information or because the type is just plain wrong.  We don't want to
    flame people when a template that couldn't possibly match doesn't match,
    e.g. complaining that we can't use fixnum+ when the arguments are known to
    be floats.


This is how we give better efficiency notes:

The Template-Note is a short noun-like string without capitalization or
punctuation that describes what the template ``does'', i.e. we say
``Unable to do ~A, doing ~A instead.''

The Cost is moved from the Vop-Info to the Template structure, and is used to
determine the ``goodness'' of possibly applicable templates.  [Could flush
Template/Vop-Info distinction]  The cost is used to choose the best applicable
template to emit, and also to determine what better templates we might have
been able to use.

A template is possibly applicable if there is an intersection between all of
the arg/result types and the corresponding arg/result restrictions, i.e. the
template is not clearly impossible: more declarations might allow it to be
emitted.


\chapter{Assembler Retargeting}


\chapter{Writing Assembly Code}

VOP writers expect:
\begin{Lentry}
\item[MOVE]
      You write when you port the assembler.)
\item[EMIT-LABEL]
      Assembler interface like INST.  Takes a label you made and says ``stick it
      here.''
   \item[GEN-LABEL]
      Returns a new label suitable for use with EMIT-LABEL exactly once and
      for referencing as often as necessary.
   \item[INST]
      Recognizes and dispatches to instructions you defined for assembler.
   \item[ALIGN]
      This takes the number of zero bits you want in the low end of the address
      of the next instruction.
   \item[ASSEMBLE]
   \item[ASSEMBLE-ELSEWHERE]
      Get ready for assembling stuff.  Takes a VOP and arbitrary PROGN-style
      body.  Wrap these around instruction emission code announcing the first
      pass of our assembler.
   \item[CURRENT-NFP-TN]
      This returns a TN for the NFP if the caller uses the number stack, or
      nil.
   \item[SB-ALLOCATED-SIZE]
      This returns the size of some storage base used by the currently
      compiling component.
   \item[...]
\end{Lentry}
;;;
;;; VOP idioms
;;;

\begin{Lentry}
\item[STORE-STACK-TN]
\item[LOAD-STACK-TN]
   These move a value from a register to the control stack, or from the
   control stack to a register.  They take care of checking the TN types,
   modifying offsets according to the address units per word, etc.
\end{Lentry}

\chapter{Required VOPS}


Note: the move VOP cannot have any wired temps.  (Move-Argument also?)  This is
so we can move stuff into wired TNs without stepping on our toes.


We create set closure variables using the Value-Cell VOP, which takes a value
and returns a value cell containing the value.  We can basically use this
instead of a Move VOP when initializing the variable.  Value-Cell-Set and
Value-Cell-Ref are used to access the value cell.  We can have a special effect
for value cells so that value cells references can be discovered to be common
subexpressions or loop invariants.




Represent unknown-values continuations as (start, count).  Unknown values
continuations are always outside of the current frame (on stack top).  Within a
function, we always set up and receive values in the standard passing
locations.  If we receive stack values, then we must BLT them down to the start
of our frame, filling in any unsupplied values.  If we generate unknown values
(i.e. PUSH-VALUES), then we set the values up in the standard locations, then
BLT them to stack top.  When doing a tail-return of MVs, we just set them up in
the standard locations and decrement SP: no BLT is necessary.

Unknown argument call (MV-CALL) takes its arguments on stack top (is given a
base pointer).  If not a tail call, then we just set the arg pointer to the
base pointer and call.  If a tail call, we must BLT the arguments down to the
beginning of the current frame.


Implement more args by BLT'ing the more args *on top* of the current frame.
This solves two problems:
\begin{itemize}
\item Any register more arguments can be made uniformly accessibly by copying
    them into memory.  [We can't store the registers in place, since the
    beginning of the frame gets double use for storing the old-cont, return-pc
    and env.]
\item It solves the deallocation problem: the arguments will be deallocated when
    the frame is returned from or a tail full call is done out of it.  So
    keyword args will be properly tail-recursive without any special mechanism
    for squeezing out the more arg once the parsing is done.  Note that a tail
    local call won't blast the more arg, since in local call the callee just
    takes the frame it is given (in this case containing the more arg).
\end{itemize}

More args in local call???  Perhaps we should not attempt local call conversion
in this case.  We already special-case keyword args in local call.  It seems
that the main importance of more args is primarily related to full call: it is
used for defining various kinds of frobs that need to take arbitrary arguments:
\begin{itemize}
\item Keyword arguments
\item Interpreter stubs
\item ``Pass through'' applications such as dispatch functions
\end{itemize}
Given the marginal importance of more args in local call, it seems unworth
going to any implementation difficulty.  In fact, it seems that it would cause
complications both at the VMR level and also in the VM definition.  This being
the case, we should flush it.


\section{Function Call}



\subsection{Registers and frame format}

These registers are used in function call and return:

A0..A{\it n}
    In full call, the first three arguments.  In unknown values return, the
    first three return values.

CFP
    The current frame pointer.  In full call, this initially points to a
    partial frame large enough to hold the passed stack arguments (zero-length
    if none).

CSP
    The current control stack top pointer. 

OCFP
    In full call, the passing location for the frame to return to.

    In unknown-values return of other than one value, the pointer to returned
    stack values.  In such a return, OCFP is always initialized to point to
    the frame returned from, even when no stack values are returned.  This
    allows OCFP to be used to restore CSP.

LRA
    In full call, the passing location for the return PC.

NARGS
    In full call, the number of arguments passed.  In unknown-values return of
    other than one value, the number of values returned.


\subsection{Full call}

What is our usage of CFP, OCFP and CSP?  

It is an invariant that CSP always points after any useful information so that
at any time an interrupt can come and allocate stuff in the stack.

TR call is also a constraint: we can't deallocate the caller's frame before the
call, since it holds the stack arguments for the call.  

What we do is have the caller set up CFP, and have the callee set CSP to CFP
plus the frame size.  The caller leaves CSP alone: the callee is the one who
does any necessary stack deallocation.

In a TR call, we don't do anything: CFP is left as CFP, and CSP points to the
end of the frame, keeping the stack arguments from being trashed.

In a normal call, CFP is set to CSP, causing the callee's frame to be allocated
after the current frame.


\subsection{Unknown values return}

The unknown values return convention is always used in full call, and is used
in local call when the compiler either can't prove that a fixed number of
values are returned, or decides not to use the fixed values convention to allow
tail-recursive XEP calls.

The unknown-values return convention has variants: single value and variable
values.  We make this distinction to optimize the important case of a returner
who knows exactly one value is being returned.  Note that it is possible to
return a single value using the variable-values convention, but it is less
efficient.

We indicate single-value return by returning at the return-pc+4; variable value
return is indicated by returning at the return PC.

Single-value return makes only the following guarantees:
    A0 holds the value returned.
    CSP has been reset: there is no garbage on the stack.

In variable value return, more information is passed back:
    A0..A2 hold the first three return values.  If fewer than three values are
    returned, then the unused registers are initialized to NIL.

    OCFP points to the frame returned from.  Note that because of our
    tail-recursive implementation of call, the frame receiving the values is
    always immediately under the frame returning the values.  This means that
    we can use OCFP to index the values when we access them, and to restore
    CSP when we want to discard them.

    NARGS holds the number of values returned.

    CSP is always (+ OCFP (* NARGS 4)), i.e. there is room on the stack
    allocated for all returned values, even if they are all actually passed in
    registers.


\subsection{External Entry Points}

Things that need to be done on XEP entry:
 1] Allocate frame
 2] Move more arg above the frame, saving context
 3] Set up env, saving closure pointer if closure
 4] Move arguments from closure to local home
    Move old-cont and return-pc to the save locations
 5] Argument count checking and dispatching

XEP VOPs:
\begin{verbatim}
Allocate-Frame
Copy-More-Arg <nargs-tn> 'fixed {in a3} => <context>, <count>
Setup-Environment
Setup-Closure-Environment => <closure>
Verify-Argument-Count <nargs-tn> 'count {for fixed-arg lambdas}
Argument-Count-Error <nargs-tn> {Drop-thru on hairy arg dispatching}
Use fast-if-=/fixnum and fast-if-</fixnum for dispatching.
\end{verbatim}

Closure vops:
\begin{verbatim}
make-closure <fun entry> <slot count> => <closure>
closure-init <closure> <values> 'slot
\end{verbatim}

Things that need to be done on all function entry:
\begin{itemize}
\item Move arguments to the variable home (consing value cells as necessary)
\item Move environment values to the local home
\item Move old-cont and return-pc to the save locations
\end{itemize}

\section{Calls}

Calling VOP's are a cross product of the following sets (with some members
missing):
   Return values
      multiple (all values)
      fixed (calling with unknown values conventions, wanting a certain
             number.)
      known (only in local call where caller/callee agree on number of
      	     values.)
      tail (doesn't return but does tail call)
   What function
      local
      named (going through symbol, like full but stash fun name for error sys)
      full (have a function)
   Args
      fixed (number of args are known at compile-time)
      variable (MULTIPLE-VALUE-CALL and APPLY)

Note on all jumps for calls and returns that we want to put some instruction
in the jump's delay slot(s).

Register usage at the time of the call:

LEXENV
   This holds the lexical environment to use during the call if it's a closure,
   and it is undefined otherwise.

CNAME
   This holds the symbol for a named call and garbage otherwise.

OCFP
   This holds the frame pointer, which the system restores upon return.  The
   callee saves this if necessary; this is passed as a pseudo-argument.

A0 ... An
   These holds the first n+1 arguments.

NARGS
   This holds the number of arguments, as a fixnum.

LRA
   This holds the lisp-return-address object which indicates where to return.
   For a tail call, this retains its current value.  The callee saves this
   if necessary; this is passed as a pseudo-argument.

CODE
   This holds the function object being called.

CSP
   The caller ignores this.  The callee sets it as necessary based on CFP.

CFP
   This holds the callee's frame pointer.  Caller sets this to the new frame
   pointer, which it remembered when it started computing arguments; this is
   CSP if there were no stack arguments.  For a tail call CFP retains its
   current value.

NSP
   The system uses this within a single function.  A function using NSP must
   allocate and deallocate before returning or making a tail call.

Register usage at the time of the return for single value return, which
goes with the unknown-values convention the caller used.

A0
   This holds the value.

CODE
   This holds the lisp-return-address at which the system continues executing.

CSP
   This holds the CFP.  That is, the stack is guaranteed to be clean, and there
   is no code at the return site to adjust the CSP.

CFP
   This holds the OCFP.

Additional register usage for multiple value return:

NARGS
   This holds the number of values returned.

A0 ... An
   These holds the first n+1 values, or NIL if there are less than n+1 values.

CSP
   Returner stores CSP to hold its CFP + NARGS * \verb+<address units per word>+

OCFP
   Returner stores this as its CFP, so the returnee has a handle on either
   the start of the returned values on the stack.


ALLOCATE FULL CALL FRAME.

If the number of call arguments (passed to the VOP as an info argument)
indicates that there are stack arguments, then it makes some callee frame for
arguments:
\begin{verbatim}
   VOP-result <- CSP
   CSP <- CSP + value of VOP info arg times address units per word.
\end{verbatim}

In a call sequence, move some arguments to the right places.

There's a variety of MOVE-ARGUMENT VOP's.

FULL CALL VOP'S
(variations determined by whether it's named, it's a tail call, there
is a variable arg count, etc.)
\begin{verbatim}
  if variable number of arguments
    NARGS <- (CSP - value of VOP argument) shift right by address units per word.
    A0...An <- values off of VOP argument (just fill them all)
  else
    NARGS <- value of VOP info argument (always a constant)

  if tail call
    OCFP <- value from VOP argument
    LRA <- value from VOP argument
    CFP stays the same since we reuse the frame
    NSP <- NFP
  else
    OCFP <- CFP
    LRA <- compute LRA by adding an assemble-time determined constant to
    	   CODE.
    CFP <- new frame pointer (remembered when starting to compute args)
           This is CSP if no stack args.
    when (current-nfp-tn VOP-self-pointer)
      stack-temp <- NFP

  if named
    CNAME <- function symbol name
    the-fun <- function object out of symbol

  LEXENV <- the-fun (from previous line or VOP argument)
  CODE <- function-entry (the first word after the-fun)
  LIP <- calc first instruction addr (CODE + constant-offset)
  jump and run off temp

  <emit Lisp return address data-block>
  <default and move return values OR receive return values>
  when (current-nfp-tn VOP-self-pointer)
    NFP <- stack-temp
\end{verbatim}
Callee:

\begin{verbatim}
XEP-ALLOCATE-FRAME
  emit function header (maybe initializes offset back to component start,
  			but other pointers are set up at load-time.  Pads
			to dual-word boundary.)
  CSP <- CFP + compile-time determined constant (frame size)
  if the function uses the number stack
    NFP <- NSP
    NSP <- NSP + compile-time determined constant (number stack frame size)
\end{verbatim}

\begin{verbatim}
SETUP-ENVIRONMENT
(either use this or the next one)

CODE <- CODE - assembler-time determined offset from function-entry back to
	       the code data-block address.
\end{verbatim}

\begin{verbatim}
SETUP-CLOSURE-ENVIRONMENT
(either use this or the previous one)
After this the CLOSURE-REF VOP can reference closure variables.

VOP-result <- LEXENV
CODE <- CODE - assembler-time determined offset from function-entry back to
	       the code data-block address.
\end{verbatim}

Return VOP's
RETURN and RETURN-MULTIPLE are for the unknown-values return convention.
For some previous caller this is either it wants n values (and it doesn't
know how many are coming), or it wants all the values returned (and it 
doesn't know how many are coming).


RETURN
(known fixed number of values, used with the unknown-values convention
 in the caller.)
When compiler invokes VOP, all values are already where they should be;
just get back to caller.

\begin{verbatim}
when (current-nfp-tn VOP-self-pointer)
  ;; The number stack grows down in memory.
  NSP <- NFP + number stack frame size for calls within the currently
                  compiling component
	       times address units per word
CODE <- value of VOP argument with LRA
if VOP info arg is 1 (number of values we know we're returning)
  CSP <- CFP
  LIP <- calc target addr
          (CODE + skip over LRA header word + skip over address units per branch)
	  (The branch is in the caller to skip down to the MV code.)
else
  NARGS <- value of VOP info arg
  nil out unused arg regs
  OCFP <- CFP  (This indicates the start of return values on the stack,
  		but you leave space for those in registers for convenience.)
  CSP <- CFP + NARGS * address-units-per-word
  LIP <- calc target addr (CODE + skip over LRA header word)
CFP <- value of VOP argument with OCFP
jump and run off LIP
\end{verbatim}

RETURN-MULTIPLE
(unknown number of values, used with the unknown-values convention in
 the caller.)
When compiler invokes VOP, it gets TN's representing a pointer to the
values on the stack and how many values were computed.

\begin{verbatim}
when (current-nfp-tn VOP-self-pointer)
  ;; The number stack grows down in memory.
  NSP <- NFP + number stack frame size for calls within the currently
                  compiling component
	       times address units per word
NARGS <- value of VOP argument
copy the args to the beginning of the current (returner's) frame.
   Actually some go into the argument registers.  When putting the rest at
   the beginning of the frame, leave room for those in the argument registers.
CSP <- CFP + NARGS * address-units-per-word
nil out unused arg regs
OCFP <- CFP  (This indicates the start of return values on the stack,
	      but you leave space for those in registers for convenience.)
CFP <- value of VOP argument with OCFP
CODE <- value of VOP argument with LRA
LIP <- calc target addr (CODE + skip over LRA header word)
jump and run off LIP
\end{verbatim}

Returnee
The call VOP's call DEFAULT-UNKNOWN-VALUES or RECEIVE-UNKNOWN-VALUES after
spitting out transfer control to get stuff from the returner.

DEFAULT-UNKNOWN-VALUES
(We know what we want and we got something.)
If returnee wants one value, it never does anything to deal with a shortage
of return values.  However, if start at PC, then it has to adjust the stack
pointer to dump extra values (move OCFP into CSP).  If it starts at PC+N,
then it just goes along with the ``want one value, got it'' case.
If the returnee wants multiple values, and there's a shortage of return
values, there are two cases to handle.  One, if the returnee wants fewer
values than there are return registers, and we start at PC+N, then it fills
in return registers \verb|A1..A<desired values necessary>|; if we start at PC,
then the returnee is fine since the returning conventions have filled in
the unused return registers with nil, but the returnee must adjust the
stack pointer to dump possible stack return values (move OCFP to CSP).
Two, if the returnee wants more values than the number of return registers,
and it starts at PC+N (got one value), then it sets up returnee state as if
an unknown number of values came back:
\begin{verbatim}
   A0 has the one value
   A1..An get nil
   NARGS gets 1
   OCFP gets CSP, so general code described below can move OCFP into CSP
If we start at PC, then branch down to the general ``got k values, wanted n''
code which takes care of the following issues:
   If k < n, fill in stack return values of nil for shortage of return
      values and move OCFP into CSP
   If k >= n, move OCFP into CSP
This also restores CODE from LRA by subtracting an assemble-time constant.
\end{verbatim}

RECEIVE-UKNOWN-VALUES
(I want whatever I get.)
We want these at the end of our frame.  When the returnee starts at
PC, it moves the return value registers to OCFP..OCFP[An] ignoring where
the end of the stack is and whether all the return value registers had
values.  The returner left room on the stack before the stack return values
for the register return values.  When the returnee starts at PC+N, bump CSP
by 1 and copy A0 there.
This also restores CODE from LRA by subtracting an assemble-time constant.


Local call

There are three flavors:
   1] KNOWN-CALL-LOCAL
      Uses known call convention where caller and callee agree where all
      the values are, and there's a fixed number of return values.
   2] CALL-LOCAL
      Uses the unknown-values convention, but we expect a particular
      number of values in return.
   3] MULTIPLE-CALL-LOCAL
      Uses the unknown-values convention, but we want all values returned.

ALLOCATE-FRAME

If the number of call arguments (passed to the VOP as an info argument)
indicates that there are stack arguments, then it makes some callee frame for
arguments:
\begin{verbatim}
   VOP-result1 <- CSP
   CSP <- CSP + control stack frame size for calls within the currently
   		   compiling component
   		times address units per word.
   when (callee-nfp-tn <VOP info arg holding callee>)
     ;; The number stack grows down.
     ;; May have to round to dual-word boundary if machines C calling
     ;;    conventions demand this.
     NSP <- NSP - number stack frame size for calls within the currently
     		     compiling component
		  times address units per word
     VOP-result2 <- NSP
\end{verbatim}
KNOWN-CALL-LOCAL, CALL-LOCAL, MULTIPLE-CALL-LOCAL
KNOWN-CALL-LOCAL has no need to affect CODE since CODE is the same for the
caller/returnee and the returner.  This uses KNOWN-RETURN.
With CALL-LOCAL and MULTIPLE-CALL-LOCAL, the caller/returnee must fixup
CODE since the callee may do a tail full call.  This happens in the code
emitted by DEFAULT-UNKNOWN-VALUES and RECEIVE-UNKNOWN-VALUES.  We use these
return conventions since we don't know what kind of values the returner
will give us.  This could happen due to a tail full call to an unknown
function, or because the callee had different return points that returned
various numbers of values.

\begin{verbatim}
when (current-nfp-tn VOP-self-pointer)   ;Get VOP self-pointer with
					 ;DEFINE-VOP switch :vop-var.
  stack-temp <- NFP
CFP <- value of VOP arg
when (callee-nfp-tn <VOP info arg holding callee>)
  <where-callee-wants-NFP-tn>  <-  value of VOP arg
<where-callee-wants-LRA-tn>  <-  compute LRA by adding an assemble-time
				 determined constant to CODE.
jump and run off VOP info arg holding start instruction for callee

<emit Lisp return address data-block>
<case call convention
  known: do nothing
  call: default and move return values
  multiple: receive return values
>
when (current-nfp-tn VOP-self-pointer)   
  NFP <- stack-temp
\end{verbatim}

KNOWN-RETURN
\begin{verbatim}
CSP <- CFP
when (current-nfp-tn VOP-self-pointer)
  ;; number stack grows down in memory.
  NSP <- NFP + number stack frame size for calls within the currently
                  compiling component
	       times address units per word
LIP <- calc target addr (value of VOP arg + skip over LRA header word)
CFP <- value of VOP arg
jump and run off LIP

\end{verbatim}


\chapter{Standard Primitives}


\chapter{Customizing VMR Conversion}

Another way in which different implementations differ is in the relative cost
of operations.  On machines without an integer multiply instruction, it may be
desirable to convert multiplication by a constant into shifts and adds, while
this is surely a bad idea on machines with hardware support for multiplication.
Part of the tuning process for an implementation will be adding implementation
dependent transforms and disabling undesirable standard transforms.

When practical, ICR transforms should be used instead of VMR generators, since
transforms are more portable and less error-prone.  Note that the Lisp code
need not be implementation independent: it may contain all sorts of
sub-primitives and similar stuff.  Generally a function should be implemented
using a transform instead of a VMR translator unless it cannot be implemented
as a transform due to being totally evil or it is just as easy to implement as
a translator because it is so simple.


\section{Constant Operands}

If the code emitted for a VOP when an argument is constant is very different
than the non-constant case, then it may be desirable to special-case the
operation in VMR conversion by emitting different VOPs.  An example would be if
SVREF is only open-coded when the index is a constant, and turns into a miscop
call otherwise.  We wouldn't want constant references to spuriously allocate
all the miscop linkage registers on the off chance that the offset might not be
constant.  See the :constant feature of VOP primitive type restrictions.


\section{Supporting Multiple Hardware Configurations}


A winning way to change emitted code depending on the hardware configuration,
i.e. what FPA is present is to do this using primitive types.  Note that the
Primitive-Type function is VM supplied, and can look at any appropriate
hardware configuration switches.  Short-Float can become 6881-Short-Float,
AFPA-Short-Float, etc.  There would be separate SBs and SCs for the registers
of each kind of FP hardware, with each hardware-specific primitive type
using the appropriate float register SC.  Then the hardware specific templates
would provide AFPA-Short-Float as the argument type restriction.

Primitive type changes:

The primitive-type structure is given a new \%Type slot, which is the CType
structure that is equivalent to this type.  There is also a Guard slot, which,
if true is a function that control whether this primitive type is allowed (due
to hardware configuration, etc.)  

We add new :Type and :Guard keywords to Def-Primitive-Type.  Type is the type
specifier that is equivalent (default to the primitive-type name), and Guard is
an expression evaluated in the null environment that controls whether this type
applies (default to none, i.e. constant T).

The Primitive-Type-Type function returns the Lisp CType corresponding to a
primitive type.  This is the \%Type unless there is a guard that returns false,
in which case it is the empty type (i.e. NIL).

[But this doesn't do what we want it to do, since we will compute the
function type for a template at load-time, so they will correspond to whatever
configuration was in effect then.  Maybe we don't want to dick with guards here
(if at all).  I guess we can defer this issue until we actually support
different FP configurations.  But it would seem pretty losing to separately
flame about all the different FP configurations that could be used to open-code
+ whenever we are forced to closed-code +.

If we separately report each better possibly applicable template that we
couldn't use, then it would be reasonable to report any conditional template
allowed by the configuration.  

But it would probably also be good to give some sort of hint that perhaps it
would be a good time to make sure you understand how to tell the compiler to
compile for a particular configuration.  Perhaps if there is a template that
applies *but for the guard*, then we could give a note.  This way, if someone
thinks they are being efficient by throwing in lots of declarations, we can let
them know that they may have to do more.

I guess the guard should be associated with the template rather than the
primitive type.  This would allow LTN and friends to easily tell whether a
template applies in this configuration.  It is also probably more natural for
some sorts of things: with some hardware variants, it may be that the SBs and
representations (SCs) are really the same, but there are some different allowed
operations.  In this case, we could easily conditionalize VOPs without the
increased complexity due to bogus SCs.  If there are different storage
resources, then we would conditionalize Primitive-Type as well.



\section{Special-case VMR convert methods}

    (defun continuation-tn (cont \&optional (check-p t))
      ...)
Return the TN which holds Continuation's first result value.  In general
this may emit code to load the value into a TN.  If Check-P is true, then
when policy indicates, code should be emitted to check that the value satisfies
the continuation asserted type.

    (defun result-tn (cont)
      ...)
Return the TN that Continuation's first value is delivered in.  In general,
may emit code to default any additional values to NIL.

    (defun result-tns (cont n)
      ...)
Similar to Result-TN, except that it returns a list of N result TNs, one
for each of the first N values.


Nearly all open-coded functions should be handled using standard template
selection.  Some (all?) exceptions:
\begin{itemize}
\item List, List* and Vector take arbitrary numbers of arguments.  Could
    implement Vector as a source transform.  Could even do List in a transform
    if we explicitly represent the stack args using \%More-Args or something.
\item \%Typep varies a lot depending on the type specifier.  We don't want to
    transform it, since we want \%Typep as a canonical form so that we can do
    type optimizations.
\item Apply is weird.
\item Funny functions emitted by the compiler: \%Listify-Rest-Args, Arg,
    \%More-Args, \%Special-Bind, \%Catch, \%Unknown-Values (?), \%Unwind-Protect,
    \%Unwind, \%\%Primitive.
\end{itemize}

\part{Run-Time System}

\chapter{The Type System}



\chapter{The Info Database}

The info database provides a functional interface to global
information about named things in \cmucl{}. Information is considered to
be global if it must persist between invocations of the compiler. The
use of a functional interface eliminates the need for the compiler to
worry about the details of the representation. The info database also
handles the need to multiple ``global'' environments, which makes it
possible to change something in the compiler without trashing the
running Lisp environment.

The info database contains arbitrary lisp values, addressed by a
combination of name, class and type. The Name is an EQUAL-thing which
is the name of the thing that we are recording information about.
Class is the kind of object involved: typical classes are Function,
Variable, Type. A type names a particular piece of information within
a given class. Class and Type are symbols, but are compared with
STRING=.


%  					-*- Dictionary: design; Package: C -*-

\chapter{The IR1 Interpreter}

May be worth having a byte-code representation for interpreted code.  This way,
an entire system could be compiled into byte-code for debugging (the
``check-out'' compiler?).

Given our current inclination for using a stack machine to interpret IR1, it
would be straightforward to layer a byte-code interpreter on top of this.


Instead of having no interpreter, or a more-or-less conventional interpreter,
or byte-code interpreter, how about directly executing IR1?

We run through the IR1 passes, possibly skipping optional ones, until we get
through environment analysis.  Then we run a post-pass that annotates IR1 with
information about where values are kept, i.e. the stack slot.

We can lazily convert functions by having FUNCTION make an interpreted function
object that holds the code (really a closure over the interpreter).  The first
time that we try to call the function, we do the conversion and processing.
Also, we can easily keep track of which interpreted functions we have expanded
macros in, so that macro redefinition automatically invalidates the old
expansion, causing lazy reconversion.

Probably the interpreter will want to represent MVs by a recognizable structure
that is always heap-allocated.  This way, we can punt the stack issues involved
in trying to spread MVs.  So a continuation value can always be kept in a
single cell.

The compiler can have some special frobs for making the interpreter efficient,
such as a call operation that extracts arguments from the stack
slots designated by a continuation list.  Perhaps 

\begin{verbatim}
    (values-mapcar fun . lists)
<==>
    (values-list (mapcar fun . lists))
\end{verbatim}

This would be used with MV-CALL.


This scheme seems to provide nearly all of the advantages of both the compiler
and conventional interpretation.  The only significant disadvantage with
respect to a conventional interpreter is that there is the one-time overhead of
conversion, but doing this lazily should make this quite acceptable.

With respect to a conventional interpreter, we have major advantages:
 + Full syntax checking: safety comparable to compiled code.
 + Semantics similar to compiled code due to code sharing.  Similar diagnostic
   messages, etc.  Reduction of error-prone code duplication.
 + Potential for full type checking according to declarations (would require
   running IR1 optimize?)
 + Simplifies debugger interface, since interpreted code can look more like
   compiled code: source paths, edit definition, etc.

For all non-run-time symbol annotations (anything other than SYMBOL-FUNCTION
and SYMBOL-VALUE), we use the compiler's global database.  MACRO-FUNCTION will
use INFO, rather than vice-versa.

When doing the IR1 phases for the interpreter, we probably want to suppress
optimizations that change user-visible function calls:
 -- Don't do local call conversion of any named functions (even lexical ones).
    This is so that a call will appear on the stack that looks like the call in
    the original source.  The keyword and optional argument transformations
    done by local call mangle things quite a bit.  Also, note local-call
    converting prevents unreferenced arguments from being deleted, which is
    another non-obvious transformation.
 -- Don't run source-transforms, IR1 transforms and IR1 optimizers.  This way,
    TRACE and BACKTRACE will show calls with the original arguments, rather
    than the ``optimized'' form, etc.  Also, for the interpreter it will
    actually be faster to call the original function (which is compiled) than
    to ``inline expand'' it.  Also, this allows implementation-dependent
    transforms to expand into %PRIMITIVE uses.

There are some problems with stepping, due to our non-syntactic IR1
representation.  The source path information is the key that makes this
conceivable.  We can skip over the stepping of a subform by quietly evaluating
nodes whose source path lies within the form being skipped.

One problem with determining what value has been returned by a form.  With a
function call, it is theoretically possible to precisely determine this, since
if we complete evaluation of the arguments, then we arrive at the Combination
node whose value is synonymous with the value of the form.  We can even detect
this case, since the Node-Source will be EQ to the form.  And we can also
detect when we unwind out of the evaluation, since we will leave the form
without having ever reached this node.

But with macros and special-forms, there is no node whose value is the value of
the form, and no node whose source is the macro call or special form.  We can
still detect when we leave the form, but we can't be sure whether this was a
normal evaluation result or an explicit RETURN-FROM.  

But does this really matter?  It seems that we can print the value returned (if
any), then just print the next form to step.  In the rare case where we did
unwind, the user should be able to figure it out.  

[We can look at this as a side-effect of CPS: there isn't any difference
between a ``normal'' return and a non-local one.]

[Note that in any control transfer (normal or otherwise), the stepper may need
to unwind out of an arbitrary number of levels of stepping.  This is because a
form in a TR position may yield its to a node arbitrarily far out.]

Another problem is with deciding what form is being stepped.  When we start
evaluating a node, we dive into code that is nested somewhere down inside that
form.  So we actually have to do a loop of asking questions before we do any
evaluation.  But what do we ask about?

If we ask about the outermost enclosing form that is a subform of the last
form that the user said to execute, then we might offer a form that isn't
really evaluated, such as a LET binding list.  

But once again, is this really a problem?  It is certainly different from a
conventional stepper, but a pretty good argument could be made that it is
superior.  Haven't you ever wanted to skip the evaluation of all the
LET bindings, but not the body?  Wouldn't it be useful to be able to skip the
DO step forms?

All of this assumes that nobody ever wants to step through the guts of a
macroexpansion.  This seems reasonable, since steppers are for weenies, and
weenies don't define macros (hence don't debug them).  But there are probably
some weenies who don't know that they shouldn't be writing macros.

We could handle this by finding the ``source paths'' in the expansion of each
macro by sticking some special frob in the source path marking the place where
the expansion happened.  When we hit code again that is in the source, then we
revert to the normal source path.  Something along these lines might be a good
idea anyway (for compiler error messages, for example).  

The source path hack isn't guaranteed to work quite so well in generated code,
though, since macros return stuff that isn't freshly consed.  But we could
probably arrange to win as long as any given expansion doesn't return two EQ
forms.

It might be nice to have a command that skipped stepping of the form, but
printed the results of each outermost enclosed evaluated subform, i.e. if you
used this on the DO step-list, it would print the result of each new-value
form.  I think this is implementable.  I guess what you would do is print each
value delivered to a DEST whose source form is the current or an enclosing
form.  Along with the value, you would print the source form for the node that
is computing the value.

The stepper can also have a ``back'' command that ``unskips'' or ``unsteps''.  This
would allow the evaluation of forms that are pure (modulo lexical variable
setting) to be undone.  This is useful, since in stepping it is common that you
skip a form that you shouldn't have, or get confused and want to restart at
some earlier point.

What we would do is remember the current node and the values of all local
variables.  heap before doing each step or skip action.  We can then back up
the state of all lexical variables and the ``program counter''.  To make this
work right with set closure variables, we would copy the cell's value, rather
than the value cell itself.

[To be fair, note that this could easily be done with our current interpreter:
the stepper could copy the environment alists.]

We can't back up the ``program counter'' when a control transfer leaves the
current function, since this state is implicitly represented in the
interpreter's state, and is discarded when we exit.  We probably want to ask
for confirmation before leaving the function to give users a chance to ``unskip''
the forms in a TR position.

Another question is whether the conventional stepper is really a good thing to
imitate...  How about an editor-based mouse-driven interface?  Instead of
``skipping'' and ``stepping'', you would just designate the next form that you
wanted to stop at.  Instead of displaying return values, you replace the source
text with the printed representation of the value.

It would show the ``program counter'' by highlighting the *innermost* form that
we are about to evaluate, i.e. the source form for the node that we are stopped
at.  It would probably also be useful to display the start of the form that was
used to designate the next stopping point, although I guess this could be
implied by the mouse position.


Such an interface would be a little harder to implement than a dumb stepper,
but it would be much easier to use.  [It would be impossible for an evalhook
stepper to do this.]


\section{Use of \%PRIMITIVE}

Note: \verb|%PRIMITIVE| can only be used in compiled code. It is a
trapdoor into the compiler, not a general syntax for accessing
``sub-primitives''. It's main use is in implementation-dependent
compiler transforms. It saves us the effort of defining a ``phony
function'' (that is not really defined), and also allows direct
communication with the code generator through codegen-info arguments.

Some primitives may be exported from the VM so that \verb|%PRIMITIVE|
can be used to make it explicit that an escape routine or interpreter
stub is assuming an operation is implemented by the compiler.

\chapter{The Debugger}
\cindex{debugger}
\label{debugger}

\credits{by Robert MacLachlan}


\section{Debugger Introduction}

The \cmucl{} debugger is unique in its level of support for source-level
debugging of compiled code.  Although some other debuggers allow access of
variables by name, this seems to be the first \llisp{} debugger that:
\begin{itemize}

\item
Tells you when a variable doesn't have a value because it hasn't been
initialized yet or has already been deallocated, or

\item
Can display the precise source location corresponding to a code
location in the debugged program.
\end{itemize}
These features allow the debugging of compiled code to be made almost
indistinguishable from interpreted code debugging.

The debugger is an interactive command loop that allows a user to examine
the function call stack.  The debugger is invoked when:
\begin{itemize}

\item
A \tindexed{serious-condition} is signaled, and it is not handled, or

\item
\findexed{error} is called, and the condition it signals is not handled, or

\item
The debugger is explicitly invoked with the \clisp{} \findexed{break}
or \findexed{debug} functions.
\end{itemize}

{\it Note: there are two debugger interfaces in \cmucl{}: the TTY
debugger (described below) and the Motif debugger. Since the
difference is only in the user interface, much of this chapter also
applies to the Motif version. \xlref{motif-interface} for a very brief
discussion of the graphical interface.}

When you enter the TTY debugger, it looks something like this:

\begin{example}
Error in function CAR.
Wrong type argument, 3, should have been of type LIST.

Restarts:
  0: Return to Top-Level.

Debug  (type H for help)

(CAR 3)
0]
\end{example}

The first group of lines describe what the error was that put us in the
debugger.  In this case \code{car} was called on \code{3}.  After \code{Restarts:}
is a list of all the ways that we can restart execution after this error.  In
this case, the only option is to return to top-level.  After printing its
banner, the debugger prints the current frame and the debugger prompt.


\section{The Command Loop}

The debugger is an interactive read-eval-print loop much like the normal
top-level, but some symbols are interpreted as debugger commands instead
of being evaluated.  A debugger command starts with the symbol name of
the command, possibly followed by some arguments on the same line.  Some
commands prompt for additional input.  Debugger commands can be
abbreviated by any unambiguous prefix: \code{help} can be typed as
\code{h}, \code{he}, etc.  For convenience, some commands have
ambiguous one-letter abbreviations: \code{f} for \code{frame}.

The package is not significant in debugger commands; any symbol with the
name of a debugger command will work.  If you want to show the value of
a variable that happens also to be the name of a debugger command, you
can use the \code{list-locals} command or the \code{debug:var}
function, or you can wrap the variable in a \code{progn} to hide it from
the command loop.

The debugger prompt is ``\var{frame}\code{]}'', where \var{frame} is the number
of the current frame.  Frames are numbered starting from zero at the top (most
recent call), increasing down to the bottom.  The current frame is the frame
that commands refer to.  The current frame also provides the lexical
environment for evaluation of non-command forms.

\cpsubindex{evaluation}{debugger} The debugger evaluates forms in the lexical
environment of the functions being debugged.  The debugger can only
access variables.  You can't \code{go} or \code{return-from} into a
function, and you can't call local functions.  Special variable
references are evaluated with their current value (the innermost binding
around the debugger invocation)\dash{}you don't get the value that the
special had in the current frame.  \xlref{debug-vars} for more
information on debugger variable access.


\section{Stack Frames}
\cindex{stack frames} \cpsubindex{frames}{stack}

A stack frame is the run-time representation of a call to a function;
the frame stores the state that a function needs to remember what it is
doing.  Frames have:
\begin{itemize}

\item
Variables (\pxlref{debug-vars}), which are the values being operated
on, and

\item
Arguments to the call (which are really just particularly interesting
variables), and

\item
A current location (\pxlref{source-locations}), which is the place in
the program where the function was running when it stopped to call another
function, or because of an interrupt or error.
\end{itemize}


\subsection{Stack Motion}

These commands move to a new stack frame and print the name of the function
and the values of its arguments in the style of a Lisp function call:
\begin{Lentry}

\item[\code{up}]
Move up to the next higher frame.  More recent function calls are considered
to be higher on the stack.

\item[\code{down}]
Move down to the next lower frame.

\item[\code{top}]
Move to the highest frame.

\item[\code{bottom}]
Move to the lowest frame.

\item[\code{frame} [\textit{n}]]
Move to the frame with the specified number.  Prompts for the number if not
supplied.

% \key{S} [\var{function-name} [\var{n}]]
% 
% \item
% Search down the stack for function.  Prompts for the function name if not
% supplied.  Searches an optional number of times, but doesn't prompt for
% this number; enter it following the function.
% 
% \item[\key{R} [\var{function-name} [\var{n}]]]
% Search up the stack for function.  Prompts for the function name if not
% supplied.  Searches an optional number of times, but doesn't prompt for
% this number; enter it following the function.
\end{Lentry}


\subsection{How Arguments are Printed}

A frame is printed to look like a function call, but with the actual argument
values in the argument positions.  So the frame for this call in the source:

\begin{lisp}
(myfun (+ 3 4) 'a)
\end{lisp}

would look like this:

\begin{example}
(MYFUN 7 A)
\end{example}

All keyword and optional arguments are displayed with their actual
values; if the corresponding argument was not supplied, the value will
be the default.  So this call:

\begin{lisp}
(subseq "foo" 1)
\end{lisp}

would look like this:

\begin{example}
(SUBSEQ "foo" 1 3)
\end{example}

And this call:

\begin{lisp}
(string-upcase "test case")
\end{lisp}

would look like this:

\begin{example}
(STRING-UPCASE "test case" :START 0 :END NIL)
\end{example}

The arguments to a function call are displayed by accessing the argument
variables.  Although those variables are initialized to the actual argument
values, they can be set inside the function; in this case the new value will be
displayed.

\code{\amprest} arguments are handled somewhat differently.  The value of
the rest argument variable is displayed as the spread-out arguments to
the call, so:

\begin{lisp}
(format t "~A is a ~A." "This" 'test)
\end{lisp}

would look like this:

\begin{example}
(FORMAT T "~A is a ~A." "This" 'TEST)
\end{example}

Rest arguments cause an exception to the normal display of keyword
arguments in functions that have both \code{\amprest} and \code{\&key}
arguments.  In this case, the keyword argument variables are not
displayed at all; the rest arg is displayed instead.  So for these
functions, only the keywords actually supplied will be shown, and the
values displayed will be the argument values, not values of the
(possibly modified) variables.

If the variable for an argument is never referenced by the function, it will be
deleted.  The variable value is then unavailable, so the debugger prints
\code{\#\textless unused-arg\textgreater} instead of the value.  Similarly, if for any of a number of
reasons (described in more detail in section \ref{debug-vars}) the value of the
variable is unavailable or not known to be available, then
\code{\#\textless unavailable-arg\textgreater} will be printed instead of the argument value.

Printing of argument values is controlled by \code{*debug-print-level*} and
\varref{debug-print-length}.

\subsection{Function Names}
\cpsubindex{function}{names}
\cpsubindex{names}{function}

If a function is defined by \code{defun}, \code{labels}, or \code{flet}, then the
debugger will print the actual function name after the open parenthesis, like:

\begin{example}
(STRING-UPCASE "test case" :START 0 :END NIL)
((SETF AREF) \#\back{a} "for" 1)
\end{example}

Otherwise, the function name is a string, and will be printed in quotes:

\begin{example}
("DEFUN MYFUN" BAR)
("DEFMACRO DO" (DO ((I 0 (1+ I))) ((= I 13))) NIL)
("SETQ *GC-NOTIFY-BEFORE*")
\end{example}

This string name is derived from the \w{\code{def}\var{mumble}} form
that encloses or expanded into the lambda, or the outermost enclosing
form if there is no \w{\code{def}\var{mumble}}.

\subsection{Funny Frames}
\cindex{external entry points}
\cpsubindex{entry points}{external}
\cpsubindex{block compilation}{debugger implications}
\cpsubindex{external}{stack frame kind}
\cpsubindex{optional}{stack frame kind}
\cpsubindex{cleanup}{stack frame kind}

Sometimes the evaluator introduces new functions that are used to implement a
user function, but are not directly specified in the source.  The main place
this is done is for checking argument type and syntax.  Usually these functions
do their thing and then go away, and thus are not seen on the stack in the
debugger.  But when you get some sort of error during lambda-list processing,
you end up in the debugger on one of these funny frames.

These funny frames are flagged by printing ``\code{[}\var{keyword}\code{]}'' after the
parentheses.  For example, this call:

\begin{lisp}
(car 'a 'b)
\end{lisp}

will look like this:

\begin{example}
(CAR 2 A) [:EXTERNAL]
\end{example}

And this call:

\begin{lisp}
(string-upcase "test case" :end)
\end{lisp}

would look like this:

\begin{example}
("DEFUN STRING-UPCASE" "test case" 335544424 1) [:OPTIONAL]
\end{example}

As you can see, these frames have only a vague resemblance to the original
call.  Fortunately, the error message displayed when you enter the debugger
will usually tell you what problem is (in these cases, too many arguments
and odd keyword arguments.)  Also, if you go down the stack to the frame for
the calling function, you can display the original source (\pxlref{source-locations}.)

With recursive or block compiled functions
(\pxlref{block-compilation}), an \kwd{EXTERNAL} frame may appear
before the frame representing the first call to the recursive function
or entry to the compiled block. This is a consequence of the way the
compiler does block compilation: there is nothing odd with your
program. You will also see \kwd{CLEANUP} frames during the execution
of \code{unwind-protect} cleanup code. Note that inline expansion and
open-coding affect what frames are present in the debugger, see
sections \ref{debugger-policy} and \ref{open-coding}.


\subsection{Debug Tail Recursion}
\label{debug-tail-recursion}
\cindex{tail recursion}
\cpsubindex{recursion}{tail}

Both the compiler and the interpreter are ``properly tail recursive.''  If a
function call is in a tail-recursive position, the stack frame will be
deallocated {\em at the time of the call}, rather than after the call returns.
Consider this backtrace:
\begin{example}
(BAR ...) 
(FOO ...)
\end{example}
Because of tail recursion, it is not necessarily the case that
\code{FOO} directly called \code{BAR}.  It may be that \code{FOO} called
some other function \code{FOO2} which then called \code{BAR}
tail-recursively, as in this example:
\begin{example}
(defun foo ()
  ...
  (foo2 ...)
  ...)

(defun foo2 (...)
  ...
  (bar ...))

(defun bar (...)
  ...)
\end{example}

Usually the elimination of tail-recursive frames makes debugging more
pleasant, since theses frames are mostly uninformative.  If there is any
doubt about how one function called another, it can usually be
eliminated by finding the source location in the calling frame (section
\ref{source-locations}.)

The elimination of tail-recursive frames can be prevented by disabling
tail-recursion optimization, which happens when the \code{debug}
optimization quality is greater than \code{2}
(\pxlref{debugger-policy}.)

For a more thorough discussion of tail recursion, \pxlref{tail-recursion}.


\subsection{Unknown Locations and Interrupts}
\label{unknown-locations}
\cindex{unknown code locations}
\cpsubindex{locations}{unknown}
\cindex{interrupts}
\cpsubindex{errors}{run-time}

The debugger operates using special debugging information attached to
the compiled code.  This debug information tells the debugger what it
needs to know about the locations in the code where the debugger can be
invoked.  If the debugger somehow encounters a location not described in
the debug information, then it is said to be \var{unknown}.  If the code
location for a frame is unknown, then some variables may be
inaccessible, and the source location cannot be precisely displayed.

There are three reasons why a code location could be unknown:
\begin{itemize}

\item
There is inadequate debug information due to the value of the \code{debug}
optimization quality.  \xlref{debugger-policy}.

\item
The debugger was entered because of an interrupt such as \code{$\hat{ }C$}.

\item
A hardware error such as ``\code{bus error}'' occurred in code that was
compiled unsafely due to the value of the \code{safety} optimization
quality.  \xlref{optimize-declaration}.
\end{itemize}

In the last two cases, the values of argument variables are accessible,
but may be incorrect.  \xlref{debug-var-validity} for more details on
when variable values are accessible.

It is possible for an interrupt to happen when a function call or return is in
progress.  The debugger may then flame out with some obscure error or insist
that the bottom of the stack has been reached, when the real problem is that
the current stack frame can't be located.  If this happens, return from the
interrupt and try again.

When running interpreted code, all locations should be known.  However,
an interrupt might catch some subfunction of the interpreter at an
unknown location.  In this case, you should be able to go up the stack a
frame or two and reach an interpreted frame which can be debugged.


\section{Variable Access}
\label{debug-vars}
\cpsubindex{variables}{debugger access}
\cindex{debug variables}

There are three ways to access the current frame's local variables in the
debugger.  The simplest is to type the variable's name into the debugger's
read-eval-print loop.  The debugger will evaluate the variable reference as
though it had appeared inside that frame.

The debugger doesn't really understand lexical scoping; it has just one
namespace for all the variables in a function.  If a symbol is the name of
multiple variables in the same function, then the reference appears ambiguous,
even though lexical scoping specifies which value is visible at any given
source location.  If the scopes of the two variables are not nested, then the
debugger can resolve the ambiguity by observing that only one variable is
accessible.

When there are ambiguous variables, the evaluator assigns each one a
small integer identifier.  The \code{debug:var} function and the
\code{list-locals} command use this identifier to distinguish between
ambiguous variables:
\begin{Lentry}

\item[\code{list-locals} \mopt{\var{prefix}}]%%\hfill\\
This command prints the name and value of all variables in the current
frame whose name has the specified \var{prefix}.  \var{prefix} may be a
string or a symbol.  If no \var{prefix} is given, then all available
variables are printed.  If a variable has a potentially ambiguous name,
then the name is printed with a ``\code{\#}\var{identifier}'' suffix, where
\var{identifier} is the small integer used to make the name unique.
\end{Lentry}

\begin{defun}{debug:}{var}{\args{\var{name} \ampoptional{} \var{identifier}}}
  
  This function returns the value of the variable in the current frame
  with the specified \var{name}.  If supplied, \var{identifier}
  determines which value to return when there are ambiguous variables.
  
  When \var{name} is a symbol, it is interpreted as the symbol name of
  the variable, i.e. the package is significant.  If \var{name} is an
  uninterned symbol (gensym), then return the value of the uninterned
  variable with the same name.  If \var{name} is a string,
  \code{debug:var} interprets it as the prefix of a variable name, and
  must unambiguously complete to the name of a valid variable.
  
  This function is useful mainly for accessing the value of uninterned
  or ambiguous variables, since most variables can be evaluated
  directly.
\end{defun}


\subsection{Variable Value Availability}
\label{debug-var-validity}
\cindex{availability of debug variables}
\cindex{validity of debug variables}
\cindex{debug optimization quality}

The value of a variable may be unavailable to the debugger in portions of the
program where \clisp{} says that the variable is defined.  If a variable value is
not available, the debugger will not let you read or write that variable.  With
one exception, the debugger will never display an incorrect value for a
variable.  Rather than displaying incorrect values, the debugger tells you the
value is unavailable.

The one exception is this: if you interrupt (e.g., with \code{$\hat{ }C$}) or if there is
an unexpected hardware error such as ``\code{bus error}'' (which should only happen
in unsafe code), then the values displayed for arguments to the interrupted
frame might be incorrect.\footnote{Since the location of an interrupt or hardware
error will always be an unknown location (\pxlref{unknown-locations}),
non-argument variable values will never be available in the interrupted frame.}
This exception applies only to the interrupted frame: any frame farther down
the stack will be fine.

The value of a variable may be unavailable for these reasons:
\begin{itemize}

\item
The value of the \code{debug} optimization quality may have omitted debug
information needed to determine whether the variable is available.
Unless a variable is an argument, its value will only be available when
\code{debug} is at least \code{2}.

\item
The compiler did lifetime analysis and determined that the value was no longer
needed, even though its scope had not been exited.  Lifetime analysis is
inhibited when the \code{debug} optimization quality is \code{3}.

\item
The variable's name is an uninterned symbol (gensym).  To save space, the
compiler only dumps debug information about uninterned variables when the
\code{debug} optimization quality is \code{3}.

\item
The frame's location is unknown (\pxlref{unknown-locations}) because
the debugger was entered due to an interrupt or unexpected hardware error.
Under these conditions the values of arguments will be available, but might be
incorrect.  This is the exception above.

\item
The variable was optimized out of existence.  Variables with no reads are
always optimized away, even in the interpreter.  The degree to which the
compiler deletes variables will depend on the value of the \code{compile-speed}
optimization quality, but most source-level optimizations are done under all
compilation policies.
\end{itemize}


Since it is especially useful to be able to get the arguments to a function,
argument variables are treated specially when the \code{speed} optimization
quality is less than \code{3} and the \code{debug} quality is at least \code{1}.
With this compilation policy, the values of argument variables are almost
always available everywhere in the function, even at unknown locations.  For
non-argument variables, \code{debug} must be at least \code{2} for values to be
available, and even then, values are only available at known locations.


\subsection{Note On Lexical Variable Access}
\cpsubindex{evaluation}{debugger}
 
When the debugger command loop establishes variable bindings for available
variables, these variable bindings have lexical scope and dynamic
extent.\footnote{The variable bindings are actually created using the \clisp{}
\code{symbol-macrolet} special form.}  You can close over them, but such closures
can't be used as upward funargs.

You can also set local variables using \code{setq}, but if the variable was closed
over in the original source and never set, then setting the variable in the
debugger may not change the value in all the functions the variable is defined
in.  Another risk of setting variables is that you may assign a value of a type
that the compiler proved the variable could never take on.  This may result in
bad things happening.


\section{Source Location Printing}
\label{source-locations}
\cpsubindex{source location printing}{debugger}

One of \cmucl{}'s unique capabilities is source level debugging of compiled
code.  These commands display the source location for the current frame:
\begin{Lentry}

\item[\code{source} \mopt{\var{context}}]%%\hfill\\
This command displays the file that the current frame's function was defined
from (if it was defined from a file), and then the source form responsible for
generating the code that the current frame was executing.  If \var{context} is
specified, then it is an integer specifying the number of enclosing levels of
list structure to print.

\item[\code{vsource} \mopt{\var{context}}]%%\hfill\\
This command is identical to \code{source}, except that it uses the
global values of \code{*print-level*} and \code{*print-length*} instead
of the debugger printing control variables \code{*debug-print-level*}
and \code{*debug-print-length*}.
\end{Lentry}

The source form for a location in the code is the innermost list present
in the original source that encloses the form responsible for generating
that code.  If the actual source form is not a list, then some enclosing
list will be printed.  For example, if the source form was a reference
to the variable \code{*some-random-special*}, then the innermost
enclosing evaluated form will be printed.  Here are some possible
enclosing forms:
\begin{example}
(let ((a *some-random-special*))
  ...)

(+ *some-random-special* ...)
\end{example}

If the code at a location was generated from the expansion of a macro or a
source-level compiler optimization, then the form in the original source that
expanded into that code will be printed.  Suppose the file
\file{/usr/me/mystuff.lisp} looked like this:
\begin{example}
(defmacro mymac ()
  '(myfun))

(defun foo ()
  (mymac)
  ...)
\end{example}
If \code{foo} has called \code{myfun}, and is waiting for it to return, then the
\code{source} command would print:
\begin{example}
; File: /usr/me/mystuff.lisp

(MYMAC)
\end{example}
Note that the macro use was printed, not the actual function call form,
\code{(myfun)}.

If enclosing source is printed by giving an argument to \code{source} or
\code{vsource}, then the actual source form is marked by wrapping it in a list
whose first element is \code{\#:***HERE***}.  In the previous example, 
\w{\code{source 1}} would print:
\begin{example}
; File: /usr/me/mystuff.lisp

(DEFUN FOO ()
  (#:***HERE***
   (MYMAC))
  ...)
\end{example}


\subsection{How the Source is Found}

If the code was defined from \llisp{} by \code{compile} or
\code{eval}, then the source can always be reliably located.  If the
code was defined from a \code{fasl} file created by
\findexed{compile-file}, then the debugger gets the source forms it
prints by reading them from the original source file.  This is a
potential problem, since the source file might have moved or changed
since the time it was compiled.

The source file is opened using the \code{truename} of the source file
pathname originally given to the compiler.  This is an absolute pathname
with all logical names and symbolic links expanded.  If the file can't
be located using this name, then the debugger gives up and signals an
error.

If the source file can be found, but has been modified since the time it was
compiled, the debugger prints this warning:
\begin{example}
; File has been modified since compilation:
;   \var{filename}
; Using form offset instead of character position.
\end{example}
where \var{filename} is the name of the source file.  It then proceeds using a
robust but not foolproof heuristic for locating the source.  This heuristic
works if:
\begin{itemize}

\item
No top-level forms before the top-level form containing the source have been
added or deleted, and

\item
The top-level form containing the source has not been modified much.  (More
precisely, none of the list forms beginning before the source form have been
added or deleted.)
\end{itemize}

If the heuristic doesn't work, the displayed source will be wrong, but will
probably be near the actual source.  If the ``shape'' of the top-level form in
the source file is too different from the original form, then an error will be
signaled.  When the heuristic is used, the the source location commands are
noticeably slowed.

Source location printing can also be confused if (after the source was
compiled) a read-macro you used in the code was redefined to expand into
something different, or if a read-macro ever returns the same \code{eq}
list twice.  If you don't define read macros and don't use \code{\#\#} in
perverted ways, you don't need to worry about this.


\subsection{Source Location Availability}

\cindex{debug optimization quality}
Source location information is only available when the \code{debug}
optimization quality is at least \code{2}.  If source location information is
unavailable, the source commands will give an error message.

If source location information is available, but the source location is
unknown because of an interrupt or unexpected hardware error
(\pxlref{unknown-locations}), then the command will print:

\begin{example}
Unknown location: using block start.
\end{example}

and then proceed to print the source location for the start of the
{\em basic block} enclosing the code location.
\cpsubindex{block}{basic} \cpsubindex{block}{start location} 
It's a bit complicated to explain exactly what a basic block is, but
here are some properties of the block start location:

\begin{itemize}
  
\item The block start location may be the same as the true location.
  
\item The block start location will never be later in the the
  program's flow of control than the true location.
  
\item No conditional control structures (such as \code{if},
  \code{cond}, \code{or}) will intervene between the block start and
  the true location (but note that some conditionals present in the
  original source could be optimized away.)  Function calls {\em do not}
  end basic blocks.
  
\item The head of a loop will be the start of a block.
  
\item The programming language concept of ``block structure'' and the
  \clisp{} \code{block} special form are totally unrelated to the
  compiler's basic block.
\end{itemize}

In other words, the true location lies between the printed location and the
next conditional (but watch out because the compiler may have changed the
program on you.)


\section{Compiler Policy Control}
\label{debugger-policy}
\cpsubindex{policy}{debugger}
\cindex{debug optimization quality}
\cindex{optimize declaration}

The compilation policy specified by \code{optimize} declarations affects the
behavior seen in the debugger.  The \code{debug} quality directly affects the
debugger by controlling the amount of debugger information dumped.  Other
optimization qualities have indirect but observable effects due to changes in
the way compilation is done.

Unlike the other optimization qualities (which are compared in relative value
to evaluate tradeoffs), the \code{debug} optimization quality is directly
translated to a level of debug information.  This absolute interpretation
allows the user to count on a particular amount of debug information being
available even when the values of the other qualities are changed during
compilation.  These are the levels of debug information that correspond to the
values of the \code{debug} quality:
\begin{Lentry}

\item[\code{0}]
Only the function name and enough information to allow the stack to
be parsed.

\item[\code{\w{$>$ 0}}]
Any level greater than \code{0} gives level \code{0} plus all
argument variables.  Values will only be accessible if the argument
variable is never set and
\code{speed} is not \code{3}.  \cmucl{} allows any real value for optimization
qualities.  It may be useful to specify \code{0.5} to get backtrace argument
display without argument documentation.

\item[\code{1}] Level \code{1} provides argument documentation
(printed arglists) and derived argument/result type information.
This makes \findexed{describe} more informative, and allows the
compiler to do compile-time argument count and type checking for any
calls compiled at run-time.

\item[\code{2}]
Level \code{1} plus all interned local variables, source location
information, and lifetime information that tells the debugger when arguments
are available (even when \code{speed} is \code{3} or the argument is set.)  This is
the default.

\item[\code{\w{$>$ 2}}]
Any level greater than \code{2} gives level \code{2} and in addition
disables tail-call optimization, so that the backtrace will contain
frames for all invoked functions, even those in tail positions.

\item[\code{3}]
Level \code{2} plus all uninterned variables.  In addition, lifetime
analysis is disabled (even when \code{speed} is \code{3}), ensuring
that all variable values are available at any known location within
the scope of the binding.  This has a speed penalty in addition to the
obvious space penalty. 
\end{Lentry}

As you can see, if the \code{speed} quality is \code{3}, debugger performance is
degraded.  This effect comes from the elimination of argument variable
special-casing (\pxlref{debug-var-validity}.)  Some degree of
speed/debuggability tradeoff is unavoidable, but the effect is not too drastic
when \code{debug} is at least \code{2}.

\cindex{inline expansion}
\cindex{semi-inline expansion}
In addition to \code{inline} and \code{notinline} declarations, the relative values
of the \code{speed} and \code{space} qualities also change whether functions are
inline expanded (\pxlref{inline-expansion}.)  If a function is inline
expanded, then there will be no frame to represent the call, and the arguments
will be treated like any other local variable.  Functions may also be
``semi-inline'', in which case there is a frame to represent the call, but the
call is to an optimized local version of the function, not to the original
function.


\section{Exiting Commands}

These commands get you out of the debugger.

\begin{Lentry}

\item[\code{quit}]
Throw to top level.

\item[\code{restart} \mopt{\var{n}}]%%\hfill\\
Invokes the \var{n}th restart case as displayed by the \code{error}
command.  If \var{n} is not specified, the available restart cases are
reported.

\item[\code{go}]
Calls \code{continue} on the condition given to \code{debug}.  If there is no
restart case named \var{continue}, then an error is signaled.

\item[\code{abort}]
Calls \code{abort} on the condition given to \code{debug}.  This is
useful for popping debug command loop levels or aborting to top level,
as the case may be.

% (\code{debug:debug-return} \var{expression} \mopt{\var{frame}})
% 
% \item
% From the current or specified frame, return the result of evaluating
% expression.  If multiple values are expected, then this function should be
% called for multiple values.
\end{Lentry}


\section{Information Commands}

Most of these commands print information about the current frame or
function, but a few show general information.

\begin{Lentry}

\item[\code{help}, \code{?}]
Displays a synopsis of debugger commands.

\item[\code{describe}]
Calls \code{describe} on the current function, displays number of local
variables, and indicates whether the function is compiled or interpreted.

\item[\code{print}]
Displays the current function call as it would be displayed by moving to
this frame.

\item[\code{vprint} (or \code{pp}) \mopt{\var{verbosity}}]%%\hfill\\
Displays the current function call using \code{*print-level*} and
\code{*print-length*} instead of \code{*debug-print-level*} and
\code{*debug-print-length*}.  \var{verbosity} is a small integer
(default 2) that controls other dimensions of verbosity.

\item[\code{error}]
Prints the condition given to \code{invoke-debugger} and the active
proceed cases.

\item[\code{backtrace} \mopt{\var{n}}]\hfill\\
Displays all the frames from the current to the bottom.  Only shows
\var{n} frames if specified.  The printing is controlled by
\code{*debug-print-level*} and \code{*debug-print-length*}.

% (\code{debug:debug-function} \mopt{\var{n}})
% 
% \item
% Returns the function from the current or specified frame.
% 
% \item[(\code{debug:function-name} \mopt{\var{n}])]
% Returns the function name from the current or specified frame.
% 
% \item[(\code{debug:pc} \mopt{\var{frame}})]
% Returns the index of the instruction for the function in the current or
% specified frame.  This is useful in conjunction with \code{disassemble}.
% The pc returned points to the instruction after the one that was fatal.
\end{Lentry}


\section{Breakpoint Commands}\cindex{breakpoints}

\cmucl{} supports setting of breakpoints inside compiled functions and
stepping of compiled code.  Breakpoints can only be set at at known
locations (\pxlref{unknown-locations}), so these commands are largely
useless unless the \code{debug} optimize quality is at least \code{2}
(\pxlref{debugger-policy}).  These commands manipulate breakpoints:
\begin{Lentry}
\item[\code{breakpoint} \var{location} \mstar{\var{option} \var{value}}]
%%\hfill\\
Set a breakpoint in some function.  \var{location} may be an integer
code location number (as displayed by \code{list-locations}) or a
keyword.  The keyword can be used to indicate setting a breakpoint at
the function start (\kwd{start}, \kwd{s}) or function end
(\kwd{end}, \kwd{e}).  The \code{breakpoint} command has
\kwd{condition}, \kwd{break}, \kwd{print} and \kwd{function}
options which work similarly to the \code{trace} options.

\item[\code{list-locations} (or \code{ll}) \mopt{\var{function}}]%%\hfill\\
List all the code locations in the current frame's function, or in
\var{function} if it is supplied.  The display format is the code
location number, a colon and then the source form for that location:
\begin{example}
3: (1- N)
\end{example}
If consecutive locations have the same source, then a numeric range like
\code{3-5:} will be printed.  For example, a default function call has a
known location both immediately before and after the call, which would
result in two code locations with the same source.  The listed function
becomes the new default function for breakpoint setting (via the
\code{breakpoint}) command.

\item[\code{list-breakpoints} (or \code{lb})]%%\hfill\\
List all currently active breakpoints with their breakpoint number.

\item[\code{delete-breakpoint} (or \code{db}) \mopt{\var{number}}]%%\hfill\\
Delete a breakpoint specified by its breakpoint number.  If no number is
specified, delete all breakpoints.

\item[\code{step}]%%\hfill\\
Step to the next possible breakpoint location in the current function.
This always steps over function calls, instead of stepping into them
\end{Lentry}


\subsection{Breakpoint Example}

Consider this definition of the factorial function:

\begin{lisp}
(defun ! (n)
  (if (zerop n)
      1
      (* n (! (1- n)))))
\end{lisp}

This debugger session demonstrates the use of breakpoints:

\begin{example}
common-lisp-user> (break) ; Invoke debugger

Break

Restarts:
  0: [CONTINUE] Return from BREAK.
  1: [ABORT   ] Return to Top-Level.

Debug  (type H for help)

(INTERACTIVE-EVAL (BREAK))
0] ll #'!
0: #'(LAMBDA (N) (BLOCK ! (IF # 1 #)))
1: (ZEROP N)
2: (* N (! (1- N)))
3: (1- N)
4: (! (1- N))
5: (* N (! (1- N)))
6: #'(LAMBDA (N) (BLOCK ! (IF # 1 #)))
0] br 2
(* N (! (1- N)))
1: 2 in !
Added.
0] q

common-lisp-user> (! 10) ; Call the function

*Breakpoint hit*

Restarts:
  0: [CONTINUE] Return from BREAK.
  1: [ABORT   ] Return to Top-Level.

Debug  (type H for help)

(! 10) ; We are now in first call (arg 10) before the multiply
Source: (* N (! (1- N)))
3] st

*Step*

(! 10) ; We have finished evaluation of (1- n)
Source: (1- N)
3] st

*Breakpoint hit*

Restarts:
  0: [CONTINUE] Return from BREAK.
  1: [ABORT   ] Return to Top-Level.

Debug  (type H for help)

(! 9) ; We hit the breakpoint in the recursive call
Source: (* N (! (1- N)))
3] 
\end{example}


\section{Function Tracing}
\cindex{tracing}
\cpsubindex{function}{tracing}

The tracer causes selected functions to print their arguments and
their results whenever they are called.  Options allow conditional
printing of the trace information and conditional breakpoints on
function entry or exit.

\begin{defmac}{}{trace}{%
    \args{\mstar{option global-value} \mstar{name \mstar{option
          value}}}}
  
  \code{trace} is a debugging tool that prints information when
  specified functions are called.  In its simplest form:
  \begin{example}
    (trace \var{name-1} \var{name-2} ...)
  \end{example}
  \code{trace} causes a printout on \vindexed{trace-output} each time
  that one of the named functions is entered or returns (the
  \var{names} are not evaluated.)  Trace output is indented according
  to the number of pending traced calls, and this trace depth is
  printed at the beginning of each line of output.  Printing verbosity
  of arguments and return values is controlled by
  \vindexed{debug-print-level} and \vindexed{debug-print-length}.

  Local functions defined by \code{flet} and \code{labels} can be
  traced using the syntax \code{(flet f f1 f2 ...)} or \code{(labels f
    f1 f2 ...)} where \code{f} is the \code{flet} or \code{labels}
  function we want to trace and \code{f1}, \code{f2}, are the
  functions containing the local function \code{f}.
  Invidiual methods can also be traced using the syntax \code{(method
    \var{name} \var{qualifiers} \var{specializers})}.
  See~\ref{sec:method-tracing} for more information.

  If no \var{names} or \var{options} are are given, \code{trace}
  returns the list of all currently traced functions,
  \code{*traced-function-list*}.
  
  Trace options can cause the normal printout to be suppressed, or
  cause extra information to be printed.  Each option is a pair of an
  option keyword and a value form.  Options may be interspersed with
  function names.  Options only affect tracing of the function whose
  name they appear immediately after.  Global options are specified
  before the first name, and affect all functions traced by a given
  use of \code{trace}.  If an already traced function is traced again,
  any new options replace the old options.  The following options are
  defined:
  \begin{Lentry}
  \item[\kwd{condition} \var{form}, \kwd{condition-after} \var{form},
    \kwd{condition-all} \var{form}] If \kwd{condition} is specified,
    then \code{trace} does nothing unless \var{form} evaluates to true
    at the time of the call.  \kwd{condition-after} is similar, but
    suppresses the initial printout, and is tested when the function
    returns.  \kwd{condition-all} tries both before and after.
    
  \item[\kwd{wherein} \var{names}] If specified, \var{names} is a
    function name or list of names.  \code{trace} does nothing unless
    a call to one of those functions encloses the call to this
    function (i.e. it would appear in a backtrace.)  Anonymous
    functions have string names like \code{"DEFUN FOO"}.  Individual
    methods can also be traced.  See section~\ref{sec:method-tracing}.

  \item[\kwd{wherein-only} \var{names}] If specified, this is just
    like \kwd{wherein}, but trace produces output only if the
    immediate caller of the traced function is one of the functions
    listed in \var{names}.
    
  \item[\kwd{break} \var{form}, \kwd{break-after} \var{form},
    \kwd{break-all} \var{form}] If specified, and \var{form} evaluates
    to true, then the debugger is invoked at the start of the
    function, at the end of the function, or both, according to the
    respective option.
    
  \item[\kwd{print} \var{form}, \kwd{print-after} \var{form},
    \kwd{print-all} \var{form}] In addition to the usual printout, the
    result of evaluating \var{form} is printed at the start of the
    function, at the end of the function, or both, according to the
    respective option.  Multiple print options cause multiple values
    to be printed.
    
  \item[\kwd{function} \var{function-form}] This is a not really an
    option, but rather another way of specifying what function to
    trace.  The \var{function-form} is evaluated immediately, and the
    resulting function is traced.
    
  \item[\kwd{encapsulate \mgroup{:default | t | nil}}] In \cmucl,
    tracing can be done either by temporarily redefining the function
    name (encapsulation), or using breakpoints.  When breakpoints are
    used, the function object itself is destructively modified to
    cause the tracing action.  The advantage of using breakpoints is
    that tracing works even when the function is anonymously called
    via \code{funcall}.
  
    When \kwd{encapsulate} is true, tracing is done via encapsulation.
    \kwd{default} is the default, and means to use encapsulation for
    interpreted functions and funcallable instances, breakpoints
    otherwise.  When encapsulation is used, forms are {\it not}
    evaluated in the function's lexical environment, but
    \code{debug:arg} can still be used.

    Note that if you trace using \kwd{encapsulate}, you will
    only get a trace or breakpoint at the outermost call to the traced
    function, not on recursive calls.

  \end{Lentry}

  In the case of functions where the known return convention is used
  to optimize, encapsulation may be necessary in order to make
  tracing work at all.  The symptom of this occurring is an error
  stating
  \begin{example}
    Error in function \var{foo}: :FUNCTION-END breakpoints are
    currently unsupported for the known return convention.
  \end{example}
  in such cases we recommend using \code{(trace \var{foo} :encapsulate
    t)}
  
  \cpsubindex{tracing}{errors}
  \cpsubindex{breakpoints}{errors}
  \cpsubindex{errors}{breakpoints}
  \cindex{function-end breakpoints}
  \cpsubindex{breakpoints}{function-end}
  
  \kwd{condition}, \kwd{break} and \kwd{print} forms are evaluated in
  the lexical environment of the called function; \code{debug:var} and
  \code{debug:arg} can be used.  The \code{-after} and \code{-all}
  forms are evaluated in the null environment.
\end{defmac}

\begin{defmac}{}{untrace}{ \args{\amprest{} \var{function-names}}}
  
  This macro turns off tracing for the specified functions, and
  removes their names from \code{*traced-function-list*}.  If no
  \var{function-names} are given, then all currently traced functions
  are untraced.
\end{defmac}

\begin{defvar}{extensions:}{traced-function-list}
  
  A list of function names maintained and used by \code{trace},
  \code{untrace}, and \code{untrace-all}.  This list should contain
  the names of all functions currently being traced.
\end{defvar}

\begin{defvar}{extensions:}{max-trace-indentation}
  
  The maximum number of spaces which should be used to indent trace
  printout.  This variable is initially set to 40.
\end{defvar}

\begin{defvar}{debug:}{trace-encapsulate-package-names}
  
  A list of package names.  Functions from these packages are traced
  using encapsulation instead of function-end breakpoints.  This list
  should at least include those packages containing functions used
  directly or indirectly in the implementation of \code{trace}.
\end{defvar}


\subsection{Encapsulation Functions}
\cindex{encapsulation}
\cindex{advising}

The encapsulation functions provide a mechanism for intercepting the
arguments and results of a function.  \code{encapsulate} changes the
function definition of a symbol, and saves it so that it can be
restored later.  The new definition normally calls the original
definition.  The \clisp{} \findexed{fdefinition} function always returns
the original definition, stripping off any encapsulation.

The original definition of the symbol can be restored at any time by
the \code{unencapsulate} function.  \code{encapsulate} and \code{unencapsulate}
allow a symbol to be multiply encapsulated in such a way that different
encapsulations can be completely transparent to each other.

Each encapsulation has a type which may be an arbitrary lisp object.
If a symbol has several encapsulations of different types, then any
one of them can be removed without affecting more recent ones.
A symbol may have more than one encapsulation of the same type, but
only the most recent one can be undone.

\begin{defun}{extensions:}{encapsulate}{%
    \args{\var{symbol} \var{type} \var{body}}}
  
  Saves the current definition of \var{symbol}, and replaces it with a
  function which returns the result of evaluating the form,
  \var{body}.  \var{Type} is an arbitrary lisp object which is the
  type of encapsulation.
  
  When the new function is called, the following variables are bound
  for the evaluation of \var{body}:
  \begin{Lentry}
    
  \item[\code{extensions:argument-list}] A list of the arguments to
    the function.
    
  \item[\code{extensions:basic-definition}] The unencapsulated
    definition of the function.
  \end{Lentry}
  The unencapsulated definition may be called with the original
  arguments by including the form
  \begin{lisp}
    (apply extensions:basic-definition extensions:argument-list)
  \end{lisp}

  \code{encapsulate} always returns \var{symbol}.
\end{defun}

\begin{defun}{extensions:}{unencapsulate}{\args{\var{symbol} \var{type}}}
  
  Undoes \var{symbol}'s most recent encapsulation of type \var{type}.
  \var{Type} is compared with \code{eq}.  Encapsulations of other
  types are left in place.
\end{defun}

\begin{defun}{extensions:}{encapsulated-p}{%
    \args{\var{symbol} \var{type}}}
  
  Returns \true{} if \var{symbol} has an encapsulation of type
  \var{type}.  Returns \nil{} otherwise.  \var{type} is compared with
  \code{eq}.
\end{defun}

\subsection{Tracing Examples}
  Here is an example of tracing with some of the possible options.
  For simplicity, this is the function:
  \begin{example}
    (defun fact (n)
      (declare (double-float n) (optimize speed))
      (if (zerop n)
          1d0
          (* n (fact (1- n)))))
    (compile 'fact)
  \end{example}

  This example shows how to use the :condition option:
  \begin{example}
    (trace fact :condition (= 4d0 (debug:arg 0)))
    (fact 10d0) ->
      0: (FACT 4.0d0)
      0: FACT returned 24.0d0
    3628800.0d0
  \end{example}
  As we can see, we produced output when the condition was satisfied.

  Here's another example:
  \begin{example}
    (untrace)
    (trace fact :break (= 4d0 (debug:arg 0)))
    (fact 10d0) ->
      0: (FACT 5.0d0)
        1: (FACT 4.0d0)


    Breaking before traced call to FACT:
       [Condition of type SIMPLE-CONDITION]

    Restarts:
      0: [CONTINUE] Return from BREAK.
      1: [ABORT   ] Return to Top-Level.

    Debug  (type H for help)
  \end{example}
  In this example, we see that normal tracing occurs until we the
  argument reaches 4d0, at which point, we break into the debugger.


% section{The Single Stepper}
% 
% \begin{defmac}{}{step}{ \args{\var{form}}}
%   
%   Evaluates form with single stepping enabled or if \var{form} is
%   \code{T}, enables stepping until explicitly disabled.  Stepping can
%   be disabled by quitting to the lisp top level, or by evaluating the
%   form \w{\code{(step ())}}.
%   
%   While stepping is enabled, every call to eval will prompt the user
%   for a single character command.  The prompt is the form which is
%   about to be \code{eval}ed.  It is printed with \code{*print-level*}
%   and \code{*print-length*} bound to \code{*step-print-level*} and
%   \code{*step-print-length*}.  All interaction is done through the
%   stream \code{*query-io*}.  Because of this, the stepper can not be
%   used in Hemlock eval mode.  When connected to a slave Lisp, the
%   stepper can be used from Hemlock.
%   
%   The commands are:
%   \begin{Lentry}
%   
%   \item[\key{n} (next)] Evaluate the expression with stepping still
%     enabled.
%   
%   \item[\key{s} (skip)] Evaluate the expression with stepping
%     disabled.
%   
%   \item[\key{q} (quit)] Evaluate the expression, but disable all
%     further stepping inside the current call to \code{step}.
%   
%   \item[\key{p} (print)] Print current form.  (does not use
%     \code{*step-print-level*} or \code{*step-print-length*}.)
%   
%   \item[\key{b} (break)] Enter break loop, and then prompt for the
%     command again when the break loop returns.
%   
%   \item[\key{e} (eval)] Prompt for and evaluate an arbitrary
%     expression.  The expression is evaluated with stepping disabled.
%   
%   \item[\key{?} (help)] Prints a brief list of the commands.
%   
%   \item[\key{r} (return)] Prompt for an arbitrary value to return as
%     result of the current call to eval.
%   
%   \item[\key{g}] Throw to top level.
%   \end{Lentry}
% \end{defmac}
% 
% \begin{defvar}{extensions:}{step-print-level}
%   \defvarx[extensions:]{step-print-length}
%   
%   \code{*print-level*} and \code{*print-length*} are bound to these
%   values while printing the current form.  \code{*step-print-level*}
%   and \code{*step-print-length*} are initially bound to 4 and 5,
%   respectively.
% \end{defvar}
% 
% \begin{defvar}{extensions:}{max-step-indentation}
%   
%   Step indents the prompts to highlight the nesting of the evaluation.
%   This variable contains the maximum number of spaces to use for
%   indenting.  Initially set to 40.
% \end{defvar}


\section{Specials}
These are the special variables that control the debugger action.

\begin{defvar}{debug:}{debug-print-level}
  \defvarx[debug:]{debug-print-length}
  
  \code{*print-level*} and \code{*print-length*} are bound to these
  values during the execution of some debug commands.  When evaluating
  arbitrary expressions in the debugger, the normal values of
  \code{*print-level*} and \code{*print-length*} are in effect.  These
  variables are initially set to 3 and 5, respectively.
\end{defvar}

\chapter{Object Format}



\label{sec:tagging}

\section{Tagging}

The following is a key of the three bit low-tagging scheme:
\begin{description}
   \item[000] even fixnum
   \item[001] function pointer
   \item[010] even other-immediate (header-words, characters, symbol-value trap value, etc.)
   \item[011] list pointer
   \item[100] odd fixnum
   \item[101] structure pointer
   \item[110] odd other immediate
  \item[111] other-pointer to data-blocks (other than conses, structures,
                                     and functions)
\end{description}

This tagging scheme forces a dual-word alignment of data-blocks on the heap,
but this can be pretty negligible: 
\begin{itemize}
\item   RATIOS and COMPLEX must have a header-word anyway since they are not a
      major type.  This wastes one word for these infrequent data-blocks since
      they require two words for the data.

\item BIGNUMS must have a header-word and probably contain only one other word
      anyway, so we probably don't waste any words here.  Most bignums just
      barely overflow fixnums, that is by a bit or two.

\item   Single and double FLOATS?
      no waste, or
      one word wasted

\item   SYMBOLS have a pad slot (current called the setf function, but unused.)
\end{itemize}
Everything else is vector-like including code, so these probably take up
so many words that one extra one doesn't matter.



\section{GC Comments}

Data-Blocks comprise only descriptors, or they contain immediate data and raw
bits interpreted by the system.  GC must skip the latter when scanning the
heap, so it does not look at a word of raw bits and interpret it as a pointer
descriptor.  These data-blocks require headers for GC as well as for operations
that need to know how to interpret the raw bits.  When GC is scanning, and it
sees a header-word, then it can determine how to skip that data-block if
necessary.  Header-Words are tagged as other-immediates.  See 
``Other-Immediates'', section~\ref{sec:other-immediates} and
``Data-Blocks and Header-Words'', section~\ref{sec:data-blocks-and-header} for comments on
distinguishing header-words from other-immediate data.  This distinction is
necessary since we scan through data-blocks containing only descriptors just as
we scan through the heap looking for header-words introducing data-blocks.

Data-Blocks containing only descriptors do not require header-words for GC
since the entire data-block can be scanned by GC a word at a time, taking
whatever action is necessary or appropriate for the data in that slot.  For
example, a cons is referenced by a descriptor with a specific tag, and the
system always knows the size of this data-block.  When GC encounters a pointer
to a cons, it can transport it into the new space, and when scanning, it can
simply scan the two words manifesting the cons interpreting each word as a
descriptor.  Actually there is no cons tag, but a list tag, so we make sure the
cons is not nil when appropriate.  A header may still be desired if the pointer
to the data-block does not contain enough information to adequately maintain
the data-block.  An example of this is a simple-vector containing only
descriptor slots, and we attach a header-word because the descriptor pointing
to the vector lacks necessary information -- the type of the vector's elements,
its length, etc.

There is no need for a major tag for GC forwarding pointers.  Since the tag
bits are in the low end of the word, a range check on the start and end of old
space tells you if you need to move the thing.  This is all GC overhead.



\section{Structures}

A structure descriptor has the structure lowtag type code, making 
{\tt structurep} a fast operation.  A structure
data-block has the following format:
\begin{verbatim}
    -------------------------------------------------------
    |   length (24 bits) | Structure header type (8 bits) |
    -------------------------------------------------------
    |   structure type name (a symbol)                    |
    -------------------------------------------------------
    |   structure slot 0                                  |
    -------------------------------------------------------
    |   ... structure slot length - 2                     |
    -------------------------------------------------------
\end{verbatim}

The header word contains the structure length, which is the number of words
(other than the header word.)  The length is always at least one, since the
first word of the structure data is the structure type name.


\section{Fixnums}

A fixnum has one of the following formats in 32 bits:
\begin{verbatim}
    -------------------------------------------------------
    |        30 bit 2's complement even integer   | 0 0 0 |
    -------------------------------------------------------
\end{verbatim}
or
\begin{verbatim}
    -------------------------------------------------------
    |        30 bit 2's complement odd integer    | 1 0 0 |
    -------------------------------------------------------
\end{verbatim}

Effectively, there is one tag for immediate integers, two zeros.  This buys one
more bit for fixnums, and now when these numbers index into simple-vectors or
offset into memory, they point to word boundaries on 32-bit, byte-addressable
machines.  That is, no shifting need occur to use the number directly as an
offset.

This format has another advantage on byte-addressable machines when fixnums are
offsets into vector-like data-blocks, including structures.  Even though we
previously mentioned data-blocks are dual-word aligned, most indexing and slot
accessing is word aligned, and so are fixnums with effectively two tag bits.

Two tags also allow better usage of special instructions on some machines that
can deal with two low-tag bits but not three.

Since the two bits are zeros, we avoid having to mask them off before using the
words for arithmetic, but division and multiplication require special shifting.



\section{Other-immediates}
\label{sec:other-immediates}



As for fixnums, there are two different three-bit lowtag codes for
other-immediate, allowing 64 other-immediate types:
\begin{verbatim}
----------------------------------------------------------------
|   Data (24 bits)        | Type (8 bits with low-tag)   | 1 0 |
----------------------------------------------------------------
\end{verbatim}

The type-code for an other-immediate type is considered to include the two
lowtag bits.  This supports the concept of a single ``type code'' namespace for
all descriptors, since the normal lowtag codes are disjoint from the
other-immediate codes.

For other-pointer objects, the full eight bits of the header type code are used
as the type code for that kind of object.  This is why we use two lowtag codes
for other-immediate types: each other-pointer object needs a distinct
other-immediate type to mark its header.

The system uses the other-immediate format for characters, 
the {\tt symbol-value} unbound trap value, and header-words for data-blocks on
the heap.  The type codes are laid out to facilitate range checks for common
subtypes; for example, all numbers will have contiguous type codes which are
distinct from the contiguous array type codes.  See
section~\ref{sec:data-blocks-and-o-i}
for details.


\section{Data-Blocks and Header-Word Format}
\label{sec:data-blocks-and-header}

Pointers to data-blocks have the following format:
\begin{verbatim}
----------------------------------------------------------------
|      Dual-word address of data-block (29 bits)       | 1 1 1 |
----------------------------------------------------------------
\end{verbatim}

The word pointed to by the above descriptor is a header-word, and it has the
same format as an other-immediate:
\begin{verbatim}
----------------------------------------------------------------
|   Data (24 bits)        | Type (8 bits with low-tag) | 0 1 0 |
----------------------------------------------------------------
\end{verbatim}
This is convenient for scanning the heap when GC'ing, but it does mean that
whenever GC encounters an other-immediate word, it has to do a range check on
the low byte to see if it is a header-word or just a character (for example).
This is easily acceptable performance hit for scanning.

The system interprets the data portion of the header-word for non-vector
data-blocks as the word length excluding the header-word.  For example, the
data field of the header for ratio and complex numbers is two, one word each
for the numerator and denominator or for the real and imaginary parts.

For vectors and data-blocks representing Lisp objects stored like vectors, the
system (usually) ignores the data portion of the header-word:
\begin{verbatim}
----------------------------------------------------------------
| Unused Data (24 bits)   | Type (8 bits with low-tag) | 0 1 0 |
----------------------------------------------------------------
|           Element Length of Vector (30 bits)           | 0 0 | 
----------------------------------------------------------------
\end{verbatim}

Using a separate word allows for much larger vectors, and it allows {\tt
length} to simply access a single word without masking or shifting.  Similarly,
the header for complex arrays and vectors has a second word, following the
header-word, the system uses for the fill pointer, so computing the length of
any array is the same code sequence.

For normal Lisp vectors, the data portion MUST be zero.  For hash
tables, a vector is used to store information about the hash key and
value, and the data portion is non-zero to indicate to GC that this is
the key/value vector for the hash table.  GENCGC uses this to
determine scavenge the key/value pairs correctly.  Cheney GC also uses
this to determine if rehashing (for EQ hash tables) is needed.


\section{Data-Blocks and Other-immediates Typing}

\label{sec:data-blocks-and-o-i}
These are the other-immediate types.  We specify them including all low eight
bits, including the other-immediate tag, so we can think of the type bits as
one type -- not an other-immediate major type and a subtype.  Also, fetching a
byte and comparing it against a constant is more efficient than wasting even a
small amount of time shifting out the other-immediate tag to compare against a
five bit constant.  (The current values can be obtained from the
generated \code{internals.h} file.)
\begin{verbatim}
                                                         HEX
Number   (< 36)
  bignum                                           10     0A
    ratio                                          14     0E
    single-float                                   18     12
    double-float                                   22     16
    double-double-float                            26     1A
    complex                                        30     1E
    (complex single-float)                         34     22
    (complex double-float)                         38     26
    (complex double-double-float)                  42     2A

Array   (<= 46 code 118)
   Simple-Array   (<= 46 code 118)
         simple-array                              46     2E
      Vector  (<= 50 code 118)
         simple-string                             50     32
         simple-bit-vector                         54     36
         simple-vector                             58     3A
         (simple-array (unsigned-byte 2) (*))      62     3E
         (simple-array (unsigned-byte 4) (*))      66     42
         (simple-array (unsigned-byte 8) (*))      70     46
         (simple-array (unsigned-byte 16) (*))     74     4A
         (simple-array (unsigned-byte 32) (*))     78     4E
         (simple-array (signed-byte 8) (*))        82     52
         (simple-array (signed-byte 16) (*))       86     56
         (simple-array (signed-byte 30) (*))       90     5A
         (simple-array (signed-byte 32) (*))       94     5E
         (simple-array single-float (*))           98     62
         (simple-array double-float (*))           102    66
         (simple-array double-double-float (*))    106    6A
         (simple-array (complex single-float) (*)  110    6E
         (simple-array (complex double-float) (*)  114    72
         (simple-array (complex double-double) (*) 118    76
      complex-string                               122    7A
      complex-bit-vector                           126    7E
      (array * (*))   -- general complex vector.   130    82
   complex-array                                   134    86

code-header-type                                   138    8A
function-header-type                               142    8E
closure-header-type                                146    92
funcallable-instance-header-type                   150    96
byte-code-function-header-type                     154    9A
byte-code-closure-header-type                      158    9E
closure-function-header-type                       162    A2
return-pc-header-type (a.k.a LRA)                  166    A6
value-cell-header-type                             170    AA
symbol-header-type                                 174    AE
base-character-type                                178    B2
system-area-pointer-type (header type)             182    B6
unbound-marker                                     186    BA
weak-pointer-type                                  190    BE
instance-header-type                               194    C2
fdefn-type                                         198    C6
scavenger-hook-type                                202    CA
\end{verbatim}

\section{Strings}

All strings in the system are C-null terminated.  This saves copying the bytes
when calling out to C.  The only time this wastes memory is when the string
contains a multiple of eight characters, and then the system allocates two more
words (since Lisp objects are dual-word aligned) to hold the C-null byte.
Since the system will make heavy use of C routines for systems calls and
libraries that save reimplementation of higher level operating system
functionality (such as pathname resolution or current directory computation),
saving on copying strings for C should make C call out more efficient.

The length word in a string header, see ``Data-Blocks and Header-Word
Format'', section~\ref{sec:data-blocks-and-header}, counts only the characters truly in the Common Lisp string.
Allocation and GC will have to know to handle the extra C-null byte, and GC
already has to deal with rounding up various objects to dual-word alignment.



\section{Symbols and NIL}

Symbol data-block has the following format:
\begin{verbatim}
-------------------------------------------------------
|     5 (data-block words)     | Symbol Type (8 bits) |
-------------------------------------------------------
|                       Value Descriptor              |
-------------------------------------------------------
|  Hash Value (x86/amd64/sparc) Unused (other arch.)  |
-------------------------------------------------------
|                        Property List                |
-------------------------------------------------------
|                          Print Name                 |
-------------------------------------------------------
|                           Package                   |
-------------------------------------------------------
\end{verbatim}

All of these slots are self-explanatory given what symbols must do in Common
Lisp.

The issues with nil are that we want it to act like a symbol, and we need list
operations such as CAR and CDR to be fast on it.  CMU Common Lisp solves this
by putting nil as the first object in static space, where other global values
reside, so it has a known address in the system:
\begin{verbatim}
-------------------------------------------------------  <-- space
|     6 (data-block words)     |         0            |      start
-------------------------------------------------------
|     0 (data-block words)     | Symbol Type (8 bits) |
-------------------------------------------------------  <-- nil
|                           Value/CAR                 |
-------------------------------------------------------
|                         Hash Value/CDR              |
-------------------------------------------------------
|                         Property List               |
-------------------------------------------------------
|                           Print Name                |
-------------------------------------------------------
|                            Package                  |
-------------------------------------------------------
|                              ...                    |
-------------------------------------------------------
\end{verbatim}
In addition, we make the list typed pointer to nil actually point past the
header word of the nil symbol data-block.  This has usefulness explained below.
The value and hash-value of nil are nil.  Therefore, any reference to nil used
as a list has quick list type checking, and CAR and CDR can go right through
the first and second words as if nil were a cons object.

When there is a reference to nil used as a symbol, the system adds offsets to
the address the same as it does for any symbol.  This works due to a
combination of nil pointing past the symbol header-word and the chosen list and
other-pointer type tags.  The list type tag is four less than the other-pointer
type tag, but nil points four additional bytes into its symbol data-block.



\section{Array Headers}

The array-header data-block has the following format:
\begin{verbatim}
----------------------------------------------------------------
| Header Len (24 bits) = Array Rank +6   | Array Type (8 bits) |
----------------------------------------------------------------
|               Fill Pointer (30 bits)                   | 0 0 | 
----------------------------------------------------------------
|               Fill Pointer p (29 bits) -- t or nil   | 1 1 1 |
----------------------------------------------------------------
|               Available Elements (30 bits)             | 0 0 | 
----------------------------------------------------------------
|               Data Vector (29 bits)                  | 1 1 1 | 
----------------------------------------------------------------
|               Displacement (30 bits)                   | 0 0 | 
----------------------------------------------------------------
|               Displacedp (29 bits) -- t or nil       | 1 1 1 | 
----------------------------------------------------------------
|               Range of First Index (30 bits)           | 0 0 | 
----------------------------------------------------------------
                              .
                              .
                              .

\end{verbatim}
The array type in the header-word is one of the eight-bit patterns from 
``Data-Blocks and Other-immediates Typing'', section~\ref{sec:data-blocks-and-header}, indicating that this is a complex
string, complex vector, complex bit-vector, or a multi-dimensional array.  The
data portion of the other-immediate word is the length of the array header
data-block.  Due to its format, its length is always six greater than the
array's number of dimensions.  The following words have the following
interpretations and types:
\begin{description}
   \item[Fill Pointer:]
      This is a fixnum indicating the number of elements in the data vector
      actually in use.  This is the logical length of the array, and it is
      typically the same value as the next slot.  This is the second word, so
      LENGTH of any array, with or without an array header, is just four bytes
      off the pointer to it.
   \item[Fill Pointer P:]
      This is either T or NIL and indicates whether the array uses the
      fill-pointer or not.
   \item[Available Elements:]
      This is a fixnum indicating the number of elements for which there is
      space in the data vector.  This is greater than or equal to the logical
      length of the array when it is a vector having a fill pointer.
   \item[Data Vector:]
      This is a pointer descriptor referencing the actual data of the array.
      This a data-block whose first word is a header-word with an array type as
      described in ``Data-Blocks and Header-Word Format'', section~\ref{sec:data-blocks-and-header} and
      ``Data-Blocks and Other-immediates Typing'', section~\ref{sec:data-blocks-and-o-i}
   \item[Displacement:]
      This is a fixnum added to the computed row-major index for any array.
      This is typically zero.
   \item[Displacedp:]
      This is either t or nil.  This is separate from the displacement slot, so
      most array accesses can simply add in the displacement slot.  The rare
      need to know if an array is displaced costs one extra word in array
      headers which probably aren't very frequent anyway.
   \item[Range of First Index:]
      This is a fixnum indicating the number of elements in the first dimension
      of the array.  Legal index values are zero to one less than this number
      inclusively.  IF the array is zero-dimensional, this slot is
      non-existent.
   \item[... (remaining slots):]
      There is an additional slot in the header for each dimension of the
      array.  These are the same as the Range of First Index slot.
\end{description}


\section{Bignums}

Bignum data-blocks have the following format:
\begin{verbatim}
-------------------------------------------------------
|      Length (24 bits)        | Bignum Type (8 bits) |
-------------------------------------------------------
|             least significant bits                  |
-------------------------------------------------------
                            .
                            .
                            .
\end{verbatim}
The elements contain the two's complement representation of the integer with
the least significant bits in the first element or closer to the header.  The
sign information is in the high end of the last element.




\section{Code Data-Blocks}

A code data-block is the run-time representation of a ``component''.  A component
is a connected portion of a program's flow graph that is compiled as a single
unit, and it contains code for many functions.  Some of these functions are
callable from outside of the component, and these are termed ``entry points''.

Each entry point has an associated user-visible function data-block (of type
{\tt function}).  The full call convention provides for calling an entry point
specified by a function object.

Although all of the function data-blocks for a component's entry points appear
to the user as distinct objects, the system keeps all of the code in a single
code data-block.  The user-visible function object is actually a pointer into
the middle of a code data-block.  This allows any control transfer within a
component to be done using a relative branch.

Besides a function object, there are other kinds of references into the middle
of a code data-block.  Control transfer into a function also occurs at the
return-PC for a call.  The system represents a return-PC somewhat similarly to
a function, so GC can also recognize a return-PC as a reference to a code
data-block.  This representation is known as a Lisp Return Address (LRA).

It is incorrect to think of a code data-block as a concatenation of ``function
data-blocks''.  Code for a function is not emitted in any particular order with
respect to that function's function-header (if any).  The code following a
function-header may only be a branch to some other location where the
function's ``real'' definition is.


The following are the three kinds of pointers to code data-blocks:
\begin{description}
   \item[Code pointer (labeled A below):]
      A code pointer is a descriptor, with other-pointer low-tag bits, pointing
      to the beginning of the code data-block.  The code pointer for the
      currently running function is always kept in a register (CODE).  In
      addition to allowing loading of non-immediate constants, this also serves
      to represent the currently running function to the debugger.
   \item[LRA (labeled B below):]
      The LRA is a descriptor, with other-pointer low-tag bits, pointing
      to a location for a function call.  Note that this location contains no
      descriptors other than the one word of immediate data, so GC can treat
      LRA locations the same as instructions.
   \item[Function (labeled C below):]
      A function is a descriptor, with function low-tag bits, that is user
      callable.  When a function header is referenced from a closure or from
      the function header's self-pointer, the pointer has other-pointer low-tag
      bits, instead of function low-tag bits.  This ensures that the internal
      function data-block associated with a closure appears to be uncallable
      (although users should never see such an object anyway).

      Information about functions that is only useful for entry points is kept
      in some descriptors following the function's self-pointer descriptor.
      All of these together with the function's header-word are known as the
      ``function header''.  GC must be able to locate the function header.  We
      provide for this by chaining together the function headers in a NIL
      terminated list kept in a known slot in the code data-block.
\end{description}

A code data-block has the following format:
\begin{verbatim}
A -->
****************************************************************
|  Header-Word count (24 bits)    |   Code-Type (8 bits)       |
----------------------------------------------------------------
|  Number of code words (fixnum tag)                           |
----------------------------------------------------------------
|  Pointer to first function header (other-pointer tag)        |
----------------------------------------------------------------
|  Debug information (structure tag)                           |
----------------------------------------------------------------
|  First constant (a descriptor)                               |
----------------------------------------------------------------
|  ...                                                         |
----------------------------------------------------------------
|  Last constant (and last word of code header)                |
----------------------------------------------------------------
|  Some instructions (non-descriptor)                          |
----------------------------------------------------------------
|     (pad to dual-word boundary if necessary)                 |

B -->
****************************************************************
|  Word offset from code header (24)   |   Return-PC-Type (8)  |
----------------------------------------------------------------
|  First instruction after return                              |
----------------------------------------------------------------
|  ... more code and LRA header-words                          |
----------------------------------------------------------------
|     (pad to dual-word boundary if necessary)                 |

C -->
****************************************************************
|  Offset from code header (24)  |   Function-Header-Type (8)  |
----------------------------------------------------------------
|  x86/amd64/sparc: Address of start of instructions for       |
|  function (non-descriptor)                                   |
|  other architectures:                                        |
|  Self-pointer back to previous word (with other-pointer tag) |
----------------------------------------------------------------
|  Pointer to next function (other-pointer low-tag) or NIL     |
----------------------------------------------------------------
|  Function name (a string or a symbol)                        |
----------------------------------------------------------------
|  Function debug arglist (a string)                           |
----------------------------------------------------------------
|  Function type (a list-style function type specifier)        |
----------------------------------------------------------------
|  Start of instructions for function (non-descriptor)         |
----------------------------------------------------------------
|  More function headers and instructions and return PCs,      |
|  until we reach the total size of header-words + code        |
|  words.                                                      |
----------------------------------------------------------------
\end{verbatim}

The following are detailed slot descriptions:
\begin{description}
   \item[Code data-block header-word:]
      The immediate data in the code data-block's header-word is the number of
      leading descriptors in the code data-block, the fixed overhead words plus
      the number of constants.  The first non-descriptor word, some code,
      appears at this word offset from the header.
   \item[Number of code words:]
      The total number of non-header-words in the code data-block.  The total
      word size of the code data-block is the sum of this slot and the
      immediate header-word data of the previous slot.
      header-word.
   \item[Pointer to first function header:]
      A NIL-terminated list of the function headers for all entry points to
      this component.
   \item[Debug information:]
      The DEBUG-INFO structure describing this component.  All information that
      the debugger wants to get from a running function is kept in this
      structure.  Since there are many functions, the current PC is used to
      locate the appropriate debug information.  The system keeps the debug
      information separate from the function data-block, since the currently
      running function may not be an entry point.  There is no way to recover
      the function object for the currently running function, since this
      data-block may not exist.
   \item[First constant ... last constant:]
      These are the constants referenced by the component, if there are any.
\vspace{1ex}
   \item[LRA header word:]
      The immediate header-word data is the word offset from the enclosing code
      data-block's header-word to this word.  This allows GC and the debugger
      to easily recover the code data-block from an LRA.  The code at the
      return point restores the current code pointer using a subtract immediate
      of the offset, which is known at compile time.
\vspace{1ex}
   \item[Function entry point header-word:]
      The immediate header-word data is the word offset from the enclosing code
      data-block's header-word to this word.  This is the same as for the
      return-PC header-word.
   \item[Address of start of instructions for function:] This is
     implemented on x86, amd64, and sparc only. In a non-closure
     function, this address allows the call sequence to always
     indirect through the second word in a user callable function.
     See section ``Closure Format''.  With a closure, indirecting
     through the second word also gets you the start of instructions
     of a function.  This pointer is a raw address, not a descriptor.
   \item[Self-pointer back to header-word:]
      In a non-closure function, this self-pointer to the previous header-word
      allows the call sequence to always indirect through the second word in a
      user callable function.  See section ``Closure Format''.  With a closure,
      indirecting through the second word gets you a function header-word.  The
      system ignores this slot in the function header for a closure, since it
      has already indirected once, and this slot could be some random thing
      that causes an error if you jump to it.  This pointer has an
      other-pointer tag instead of a function pointer tag, indicating it is not
      a user callable Lisp object.
   \item[Pointer to next function:]
      This is the next link in the thread of entry point functions found in
      this component.  This value is NIL when the current header is the last
      entry point in the component.
   \item[Function name:]
      This function's name (for printing).  If the user defined this function
      with DEFUN, then this is the defined symbol, otherwise it is a
      descriptive string.
   \item[Function debug arglist:]
      A printed string representing the function's argument list, for human
      readability.  If it is a macroexpansion function, then this is the
      original DEFMACRO arglist, not the actual expander function arglist.
   \item[Function type:]
      A list-style function type specifier representing the argument signature
      and return types for this function.  For example,
      \begin{verbatim}
(function (fixnum fixnum fixnum) fixnum)
      \end{verbatim}
      or
      \begin{verbatim}
(function (string &key (:start unsigned-byte)) string)
      \end{verbatim}
      This information is intended for machine readablilty, such as by the
      compiler.
\end{description}


\section{Closure Format}

A closure data-block has the following format:
\begin{verbatim}
----------------------------------------------------------------
|  Word size (24 bits)           |  Closure-Type (8 bits)      |
----------------------------------------------------------------
|  Pointer to function header (other-pointer low-tag)          |
----------------------------------------------------------------
|                                 .                            |
|                      Environment information                 |
|                                 .                            |
----------------------------------------------------------------
\end{verbatim}

A closure descriptor has function low-tag bits.  This means that a descriptor
with function low-tag bits may point to either a function header or to a
closure.  The idea is that any callable Lisp object has function low-tag bits.
Insofar as call is concerned, we make the format of closures and non-closure
functions compatible.  This is the reason for the self-pointer in a function
header.  Whenever you have a callable object, you just jump through the second
word, offset some bytes, and go.



\section{Function call}

Due to alignment requirements and low-tag codes, it is not possible to use a
hardware call instruction to compute the LRA.  Instead the LRA
for a call is computed by doing an add-immediate to the start of the code
data-block.

An advantage of using a single data-block to represent both the descriptor and
non-descriptor parts of a function is that both can be represented by a
single pointer.  This reduces the number of memory accesses that have to be
done in a full call.  For example, since the constant pool is implicit in an
LRA, a call need only save the LRA, rather than saving both the
return PC and the constant pool.



\section{Memory Layout}

\cmucl{} has four spaces, read-only, static, dynamic-0, and dynamic-1.
Read-only contains objects that the system never modifies, moves, or reclaims.
Static space contains some global objects necessary for the system's runtime or
performance (since they are located at a known offset at a known address), and
the system never moves or reclaims these.  However, GC does need to scan static
space for references to moved objects.  Dynamic-0 and dynamic-1 are the two
heap areas for stop-and-copy GC algorithms.

What global objects are at the head of static space???
\begin{verbatim}
   NIL
   eval::*top-of-stack*
   lisp::*current-catch-block*
   lisp::*current-unwind-protect*
   FLAGS (RT only)
   BSP (RT only)
   HEAP (RT only)
\end{verbatim}

In addition to the above spaces, the system has a control stack, binding stack,
and a number stack.  The binding stack contains pairs of descriptors, a symbol
and its previous value.  The number stack is the same as the C stack, and the
system uses it for non-Lisp objects such as raw system pointers, saving
non-Lisp registers, parts of bignum computations, etc.



\section{System Pointers}

The system pointers reference raw allocated memory, data returned by foreign
function calls, etc.  The system uses these when you need a pointer to a
non-Lisp block of memory, using an other-pointer.  This provides the greatest
flexibility by relieving contraints placed by having more direct references
that require descriptor type tags.

A system area pointer data-block has the following format:
\begin{verbatim}
-------------------------------------------------------
|     1 (data-block words)        | SAP Type (8 bits) |
-------------------------------------------------------
|             system area pointer                     |
-------------------------------------------------------
\end{verbatim}

``SAP'' means ``system area pointer'', and much of our code contains this naming
scheme.  We don't currently restrict system pointers to one area of memory, but
if they do point onto the heap, it is up to the user to prevent being screwed
by GC or whatever.

\section{Weak Pointers}
\label{sec:weak-pointers}

A weak-pointer data-block has the following format:
\begin{verbatim}
-------------------------------------------------------
|  4 (data-block words) |  Weak pointer Type (8 bits) |
-------------------------------------------------------
|                 weak-pointer-value                  |
-------------------------------------------------------
|                 weak-pointer-broken                 |
-------------------------------------------------------
|                 mark-bit (T or NIL)                 |
-------------------------------------------------------
|                   next                              |
-------------------------------------------------------
\end{verbatim}

The mark-bit is used when gencgc is available.  It's used to note if
this weak pointer has been visited before so that scavenging
weak-pointers isn't an $O(n^2)$ process.

The last slot is an internal slot used by the C runtime to chain all
the weak pointers together for GC.


\chapter{Memory Management}

\section{Stacks and Globals}

\section{Heap Layout}

\section{Garbage Collection}

\chapter{Interface to C and Assembler}


\section{Linkage Table}

The linkage table feature is based on how dynamic libraries dispatch.
A table of functions is used which is filled in with the appropriate
code to jump to the correct address.

For \cmucl{}, this table is stored at
\code{target-foreign-linkage-space-start}. Each entry is
\code{target-foreign-linkage-entry-size} bytes long.

At startup, the table is initialized with default values in
\code{os\_foreign\_linkage\_init}. On x86 platforms, the first entry is
code to call the routine \code{resolve\_linkage\_tramp}. All other
entries jump to the first entry. The function
\code{resolve\_linkage\_tramp} looks at where it was called from to
figure out which entry in the table was used. It calls
\code{lazy\_resolve\_linkage} with the address of the linkage entry.
This routine then fills in the appropriate linkage entry with code to
jump to where the real routine is located, and returns the address of
the entry. On return, \code{resolve\_linkage\_tramp} then just jumps to
the returned address to call the desired function. On all subsequent
calls, the entry no longer points to \code{resolve\_linkage\_tramp} but
to the real function.

This describes how function calls are made. For foreign data,
\code{lazy\_resolve\_linkage} stuffs the address of the actual foreign
data into the linkage table. The lisp code then just loads the address
from there to get the actual address of the foreign data.

For sparc, the linkage table is slightly different. The first entry is
the entry for \code{call\_into\_c} so we never have to look this up. All
other entries are for \code{resolve\_linkage\_tramp}. This has the
advantage that \code{resolve\_linkage\_tramp} can be much simpler since
all calls to foreign code go through \code{call\_into\_c} anyway, and
that means all live Lisp registers have already been saved. Also, to
make life simpler, we lie about \code{closure\_tramp} and
\code{undefined\_tramp} in the Lisp code. These are really functions,
but we treat them as foreign data since these two routines are only
used as addresses in the Lisp code to stuff into a lisp function
header.

On the Lisp side, there are two supporting data structures for the
linkage table: \code{*linkage-table-data*} and
\code{*foreign-linkage-symbols*}. The latter is a hash table whose key
is the foreign symbol (a string) and whose value is an index into
\code{*linkage-table-data*}.

\code{*linkage-table-data*} is a vector with an unlispy layout. Each
entry has 3 parts:

\begin{itemize}
\item symbol name
\item type, a fixnum, 1 = code, 2 = data
\item library list - the library list at the time the symbol is registered.
\end{itemize}

Whenever a new foreign symbol is defined, a new
\code{*linkage-table-data*} entry is created.
\code{*foreign-linkage-symbols*} is updated with the symbol and the
entry number into \code{*linkage-table-data*}.

The \code{*linkage-table-data*} is accessed from C (hence the unlispy
layout), to figure out the symbol name and the type so that the
address of the symbol can be determined.  The type tells the C code
how to fill in the entry in the linkage-table itself.

% (Should say something about genesis too, but I don't know how that
% works other than the initial table is setup with the appropriate first
% entry.)


\chapter{Low-level debugging}

\chapter{Core File Format}

\chapter{Fasload File Format}% -*- Dictionary: design -*-
\section{General}

The purpose of Fasload files is to allow concise storage and rapid
loading of Lisp data, particularly function definitions.  The intent
is that loading a Fasload file has the same effect as loading the
source file from which the Fasload file was compiled, but accomplishes
the tasks more efficiently.  One noticeable difference, of course, is
that function definitions may be in compiled form rather than
S-expression form.  Another is that Fasload files may specify in what
parts of memory the Lisp data should be allocated.  For example,
constant lists used by compiled code may be regarded as read-only.

In some Lisp implementations, Fasload file formats are designed to
allow sharing of code parts of the file, possibly by direct mapping
of pages of the file into the address space of a process.  This
technique produces great performance improvements in a paged
time-sharing system.  Since the Mach project is to produce a
distributed personal-computer network system rather than a
time-sharing system, efficiencies of this type are explicitly {\it not}
a goal for the CMU Common Lisp Fasload file format.

On the other hand, CMU Common Lisp is intended to be portable, as it will
eventually run on a variety of machines.  Therefore an explicit goal
is that Fasload files shall be transportable among various
implementations, to permit efficient distribution of programs in
compiled form.  The representations of data objects in Fasload files
shall be relatively independent of such considerations as word
length, number of type bits, and so on.  If two implementations
interpret the same macrocode (compiled code format), then Fasload
files should be completely compatible.  If they do not, then files
not containing compiled code (so-called ``Fasdump'' data files) should
still be compatible.  While this may lead to a format which is not
maximally efficient for a particular implementation, the sacrifice of
a small amount of performance is deemed a worthwhile price to pay to
achieve portability.

The primary assumption about data format compatibility is that all
implementations can support I/O on finite streams of eight-bit bytes.
By ``finite'' we mean that a definite end-of-file point can be detected
irrespective of the content of the data stream.  A Fasload file will
be regarded as such a byte stream.

\section{Strategy}

A Fasload file may be regarded as a human-readable prefix followed by
code in a funny little language.  When interpreted, this code will
cause the construction of the encoded data structures.  The virtual
machine which interprets this code has a {\it stack} and a {\it table},
both initially empty.  The table may be thought of as an expandable
register file; it is used to remember quantities which are needed
more than once.  The elements of both the stack and the table are
Lisp data objects.  Operators of the funny language may take as
operands following bytes of the data stream, or items popped from the
stack.  Results may be pushed back onto the stack or pushed onto the
table.  The table is an indexable stack that is never popped; it is
indexed relative to the base, not the top, so that an item once
pushed always has the same index.

More precisely, a Fasload file has the following macroscopic
organization.  It is a sequence of zero or more groups concatenated
together.  End-of-file must occur at the end of the last group.  Each
group begins with a series of seven-bit ASCII characters terminated
by one or more bytes of all ones \verb|#xFF|; this is called the
{\it header}.  Following the bytes which terminate the header is the
{\it body}, a stream of bytes in the funny binary language.  The body
of necessity begins with a byte other than \verb|#xFF|.  The body is
terminated by the operation {\tt FOP-END-GROUP}.

The first nine characters of the header must be \verb|FASL FILE| in
upper-case letters.  The rest may be any ASCII text, but by
convention it is formatted in a certain way.  The header is divided
into lines, which are grouped into paragraphs.  A paragraph begins
with a line which does {\it not} begin with a space or tab character,
and contains all lines up to, but not including, the next such line.
The first word of a paragraph, defined to be all characters up to but
not including the first space, tab, or end-of-line character, is the
{\it name} of the paragraph.  A Fasload file header might look something like
this:
\begin{verbatim}
FASL FILE >SteelesPerq>User>Guy>IoHacks>Pretty-Print.Slisp
Package Pretty-Print
Compiled 31-Mar-1988 09:01:32 by some random luser
Compiler Version 1.6, Lisp Version 3.0.
Functions: INITIALIZE DRIVER HACK HACK1 MUNGE MUNGE1 GAZORCH
	   MINGLE MUDDLE PERTURB OVERDRIVE GOBBLE-KEYBOARD
	   FRY-USER DROP-DEAD HELP CLEAR-MICROCODE
	    %AOS-TRIANGLE %HARASS-READTABLE-MAYBE
Macros:    PUSH POP FROB TWIDDLE
\end{verbatim}
{\it one or more bytes of \verb|#xFF|}

The particular paragraph names and contents shown here are only intended as
suggestions.

\section{Fasload Language}

Each operation in the binary Fasload language is an eight-bit
(one-byte) opcode.  Each has a name beginning with ``{\tt FOP-}''.  In	
the following descriptions, the name is followed by operand
descriptors.  Each descriptor denotes operands that follow the opcode
in the input stream.  A quantity in parentheses indicates the number
of bytes of data from the stream making up the operand.  Operands
which implicitly come from the stack are noted in the text.  The
notation ``$\Rightarrow$ stack'' means that the result is pushed onto the
stack; ``$\Rightarrow$ table'' similarly means that the result is added to the
table.  A construction like ``{\it n}(1) {\it value}({\it n})'' means that
first a single byte {\it n} is read from the input stream, and this
byte specifies how many bytes to read as the operand named {\it value}.
All numeric values are unsigned binary integers unless otherwise
specified.  Values described as ``signed'' are in two's-complement form
unless otherwise specified.  When an integer read from the stream
occupies more than one byte, the first byte read is the least
significant byte, and the last byte read is the most significant (and
contains the sign bit as its high-order bit if the entire integer is
signed).

Some of the operations are not necessary, but are rather special
cases of or combinations of others.  These are included to reduce the
size of the file or to speed up important cases.  As an example,
nearly all strings are less than 256 bytes long, and so a special
form of string operation might take a one-byte length rather than a
four-byte length.  As another example, some implementations may
choose to store bits in an array in a left-to-right format within
each word, rather than right-to-left.  The Fasload file format may
support both formats, with one being significantly more efficient
than the other for a given implementation.  The compiler for any
implementation may generate the more efficient form for that
implementation, and yet compatibility can be maintained by requiring
all implementations to support both formats in Fasload files.

Measurements are to be made to determine which operation codes are
worthwhile; little-used operations may be discarded and new ones
added.  After a point the definition will be ``frozen'', meaning that
existing operations may not be deleted (though new ones may be added;
some operations codes will be reserved for that purpose).

\begin{description}
\item[0:] \hspace{2em} {\tt FOP-NOP} \\
No operation.  (This is included because it is recognized
that some implementations may benefit from alignment of operands to some
operations, for example to 32-bit boundaries.  This operation can be used
to pad the instruction stream to a desired boundary.)

\item[1:] \hspace{2em} {\tt FOP-POP} \hspace{2em} $\Rightarrow$ \hspace{2em} table \\
One item is popped from the stack and added to the table.

\item[2:] \hspace{2em} {\tt FOP-PUSH} \hspace{2em} {\it index}(4) \hspace{2em} $\Rightarrow$ \hspace{2em} stack \\
Item number {\it index} of the table is pushed onto the stack.
The first element of the table is item number zero.

\item[3:] \hspace{2em} {\tt FOP-BYTE-PUSH} \hspace{2em} {\it index}(1) \hspace{2em} $\Rightarrow$ \hspace{2em} stack \\
Item number {\it index} of the table is pushed onto the stack.
The first element of the table is item number zero.

\item[4:] \hspace{2em} {\tt FOP-EMPTY-LIST} \hspace{2em} $\Rightarrow$ \hspace{2em} stack \\
The empty list ({\tt ()}) is pushed onto the stack.

\item[5:] \hspace{2em} {\tt FOP-TRUTH} \hspace{2em} $\Rightarrow$ \hspace{2em} stack \\
The standard truth value ({\tt T}) is pushed onto the stack.

\item[6:] \hspace{2em} {\tt FOP-SYMBOL-SAVE} \hspace{2em} {\it n}(4) \hspace{2em} {\it name}({\it n})
\hspace{2em} $\Rightarrow$ \hspace{2em} stack \& table\\
The four-byte operand {\it n} specifies the length of the print name
of a symbol.  The name follows, one character per byte,
with the first byte of the print name being the first read.
The name is interned in the default package,
and the resulting symbol is both pushed onto the stack and added to the table.

\item[7:] \hspace{2em} {\tt FOP-SMALL-SYMBOL-SAVE} \hspace{2em} {\it n}(1) \hspace{2em} {\it name}({\it n}) \hspace{2em} $\Rightarrow$ \hspace{2em} stack \& table\\
The one-byte operand {\it n} specifies the length of the print name
of a symbol.  The name follows, one character per byte,
with the first byte of the print name being the first read.
The name is interned in the default package,
and the resulting symbol is both pushed onto the stack and added to the table.

\item[8:] \hspace{2em} {\tt FOP-SYMBOL-IN-PACKAGE-SAVE} \hspace{2em} {\it index}(4)
\hspace{2em} {\it n}(4) \hspace{2em} {\it name}({\it n})
\hspace{2em} $\Rightarrow$ \hspace{2em} stack \& table\\
The four-byte {\it index} specifies a package stored in the table.
The four-byte operand {\it n} specifies the length of the print name
of a symbol.  The name follows, one character per byte,
with the first byte of the print name being the first read.
The name is interned in the specified package,
and the resulting symbol is both pushed onto the stack and added to the table.

\item[9:] \hspace{2em} {\tt FOP-SMALL-SYMBOL-IN-PACKAGE-SAVE}  \hspace{2em} {\it index}(4)
\hspace{2em} {\it n}(1) \hspace{2em} {\it name}({\it n}) \hspace{2em}
$\Rightarrow$ \hspace{2em} stack \& table\\
The four-byte {\it index} specifies a package stored in the table.
The one-byte operand {\it n} specifies the length of the print name
of a symbol.  The name follows, one character per byte,
with the first byte of the print name being the first read.
The name is interned in the specified package,
and the resulting symbol is both pushed onto the stack and added to the table.

\item[10:] \hspace{2em} {\tt FOP-SYMBOL-IN-BYTE-PACKAGE-SAVE} \hspace{2em} {\it index}(1)
\hspace{2em} {\it n}(4) \hspace{2em} {\it name}({\it n})
\hspace{2em} $\Rightarrow$ \hspace{2em} stack \& table\\
The one-byte {\it index} specifies a package stored in the table.
The four-byte operand {\it n} specifies the length of the print name
of a symbol.  The name follows, one character per byte,
with the first byte of the print name being the first read.
The name is interned in the specified package,
and the resulting symbol is both pushed onto the stack and added to the table.

\item[11:]\hspace{2em} {\tt FOP-SMALL-SYMBOL-IN-BYTE-PACKAGE-SAVE} \hspace{2em} {\it index}(1)
\hspace{2em} {\it n}(1) \hspace{2em} {\it name}({\it n}) \hspace{2em}
$\Rightarrow$ \hspace{2em} stack \& table\\
The one-byte {\it index} specifies a package stored in the table.
The one-byte operand {\it n} specifies the length of the print name
of a symbol.  The name follows, one character per byte,
with the first byte of the print name being the first read.
The name is interned in the specified package,
and the resulting symbol is both pushed onto the stack and added to the table.

\item[12:] \hspace{2em} {\tt FOP-UNINTERNED-SYMBOL-SAVE} \hspace{2em} {\it n}(4) \hspace{2em} {\it name}({\it n})
\hspace{2em} $\Rightarrow$ \hspace{2em} stack \& table\\
Like {\tt FOP-SYMBOL-SAVE}, except that it creates an uninterned symbol.

\item[13:] \hspace{2em} {\tt FOP-UNINTERNED-SMALL-SYMBOL-SAVE} \hspace{2em} {\it n}(1)
\hspace{2em} {\it name}({\it n}) \hspace{2em} $\Rightarrow$ \hspace{2em} stack
\& table\\
Like {\tt FOP-SMALL-SYMBOL-SAVE}, except that it creates an uninterned symbol.

\item[14:] \hspace{2em} {\tt FOP-PACKAGE} \hspace{2em} $\Rightarrow$ \hspace{2em} table \\
An item is popped from the stack; it must be a symbol.	The package of
that name is located and pushed onto the table.

\item[15:] \hspace{2em} {\tt FOP-LIST} \hspace{2em} {\it length}(1) \hspace{2em} $\Rightarrow$ \hspace{2em} stack \\
The unsigned operand {\it length} specifies a number of
operands to be popped from the stack.  These are made into a list
of that length, and the list is pushed onto the stack.
The first item popped from the stack becomes the last element of
the list, and so on.  Hence an iterative loop can start with
the empty list and perform ``pop an item and cons it onto the list''
{\it length} times.
(Lists of length greater than 255 can be made by using {\tt FOP-LIST*}
repeatedly.)

\item[16:] \hspace{2em} {\tt FOP-LIST*} \hspace{2em} {\it length}(1) \hspace{2em} $\Rightarrow$ \hspace{2em} stack \\
This is like {\tt FOP-LIST} except that the constructed list is terminated
not by {\tt ()} (the empty list), but by an item popped from the stack
before any others are.	Therefore {\it length}+1 items are popped in all.
Hence an iterative loop can start with
a popped item and perform ``pop an item and cons it onto the list''
{\it length}+1 times.

\item[17-24:] \hspace{2em} {\tt FOP-LIST-1}, {\tt FOP-LIST-2}, ..., {\tt FOP-LIST-8} \\
{\tt FOP-LIST-{\it k}} is like {\tt FOP-LIST} with a byte containing {\it k}
following it.  These exist purely to reduce the size of Fasload files.
Measurements need to be made to determine the useful values of {\it k}.

\item[25-32:] \hspace{2em} {\tt FOP-LIST*-1}, {\tt FOP-LIST*-2}, ..., {\tt FOP-LIST*-8} \\
{\tt FOP-LIST*-{\it k}} is like {\tt FOP-LIST*} with a byte containing {\it k}
following it.  These exist purely to reduce the size of Fasload files.
Measurements need to be made to determine the useful values of {\it k}.

\item[33:] \hspace{2em} {\tt FOP-INTEGER} \hspace{2em} {\it n}(4) \hspace{2em} {\it value}({\it n}) \hspace{2em}
$\Rightarrow$ \hspace{2em} stack \\
A four-byte unsigned operand specifies the number of following
bytes.	These bytes define the value of a signed integer in two's-complement
form.  The first byte of the value is the least significant byte.

\item[34:] \hspace{2em} {\tt FOP-SMALL-INTEGER} \hspace{2em} {\it n}(1) \hspace{2em} {\it value}({\it n})
\hspace{2em} $\Rightarrow$ \hspace{2em} stack \\
A one-byte unsigned operand specifies the number of following
bytes.	These bytes define the value of a signed integer in two's-complement
form.  The first byte of the value is the least significant byte.

\item[35:] \hspace{2em} {\tt FOP-WORD-INTEGER} \hspace{2em} {\it value}(4) \hspace{2em} $\Rightarrow$ \hspace{2em} stack \\
A four-byte signed integer (in the range $-2^{31}$ to $2^{31}-1$) follows the
operation code.  A LISP integer (fixnum or bignum) with that value
is constructed and pushed onto the stack.

\item[36:] \hspace{2em} {\tt FOP-BYTE-INTEGER} \hspace{2em} {\it value}(1) \hspace{2em} $\Rightarrow$ \hspace{2em} stack \\
A one-byte signed integer (in the range -128 to 127) follows the
operation code.  A LISP integer (fixnum or bignum) with that value
is constructed and pushed onto the stack.

\item[37:] \hspace{2em} {\tt FOP-STRING} \hspace{2em} {\it n}(4) \hspace{2em} {\it name}({\it n})
\hspace{2em} $\Rightarrow$ \hspace{2em} stack \\
The four-byte operand {\it n} specifies the length of a string to
construct.  The characters of the string follow, one per byte.
The constructed string is pushed onto the stack.

\item[38:] \hspace{2em} {\tt FOP-SMALL-STRING} \hspace{2em} {\it n}(1) \hspace{2em} {\it name}({\it n}) \hspace{2em} $\Rightarrow$ \hspace{2em} stack \\
The one-byte operand {\it n} specifies the length of a string to
construct.  The characters of the string follow, one per byte.
The constructed string is pushed onto the stack.

\item[39:] \hspace{2em} {\tt FOP-VECTOR} \hspace{2em} {\it n}(4) \hspace{2em} $\Rightarrow$ \hspace{2em} stack \\
The four-byte operand {\it n} specifies the length of a vector of LISP objects
to construct.  The elements of the vector are popped off the stack;
the first one popped becomes the last element of the vector.
The constructed vector is pushed onto the stack.

\item[40:] \hspace{2em} {\tt FOP-SMALL-VECTOR} \hspace{2em} {\it n}(1) \hspace{2em} $\Rightarrow$ \hspace{2em} stack \\
The one-byte operand {\it n} specifies the length of a vector of LISP objects
to construct.  The elements of the vector are popped off the stack;
the first one popped becomes the last element of the vector.
The constructed vector is pushed onto the stack.

\item[41:] \hspace{2em} {\tt FOP-UNIFORM-VECTOR} \hspace{2em} {\it n}(4) \hspace{2em} $\Rightarrow$ \hspace{2em} stack \\
The four-byte operand {\it n} specifies the length of a vector of LISP objects
to construct.  A single item is popped from the stack and used to initialize
all elements of the vector.  The constructed vector is pushed onto the stack.

\item[42:] \hspace{2em} {\tt FOP-SMALL-UNIFORM-VECTOR} \hspace{2em} {\it n}(1) \hspace{2em} $\Rightarrow$ \hspace{2em} stack \\
The one-byte operand {\it n} specifies the length of a vector of LISP objects
to construct.  A single item is popped from the stack and used to initialize
all elements of the vector.  The constructed vector is pushed onto the stack.

\item[43:] \hspace{2em} {\tt FOP-INT-VECTOR} \hspace{2em} {\it len}(4) \hspace{2em}
{\it size}(1) \hspace{2em} {\it data}($\left\lceil len*count/8\right\rceil$)
\hspace{2em} $\Rightarrow$ \hspace{2em} stack \\
The four-byte operand {\it n} specifies the length of a vector of
unsigned integers to be constructed.   Each integer is {\it size}
bits long, and is packed according to the machine's native byte ordering.
{\it size} must be a directly supported i-vector element size.  Currently
supported values are 1,2,4,8,16 and 32.

\item[44:] \hspace{2em} {\tt FOP-UNIFORM-INT-VECTOR} \hspace{2em} {\it n}(4) \hspace{2em} {\it size}(1) \hspace{2em}
{\it value}(@ceiling$<${\it size}/8$>$) \hspace{2em} $\Rightarrow$ \hspace{2em} stack \\
The four-byte operand {\it n} specifies the length of a vector of unsigned
integers to construct.
Each integer is {\it size} bits big, and is initialized to the value
of the operand {\it value}.
The constructed vector is pushed onto the stack.

\item[45:] \hspace{2em} {\tt FOP-LAYOUT} \hspace{2em} \\
Pops the stack four times to get the name, length, inheritance and depth for a layout object. 

\item[46:] \hspace{2em} {\tt FOP-SINGLE-FLOAT} \hspace{2em} {\it data}(4) \hspace{2em}
$\Rightarrow$ \hspace{2em} stack \\
The {\it data} bytes are read as an integer, then turned into an IEEE single
float (as though by {\tt make-single-float}).

\item[47:] \hspace{2em} {\tt FOP-DOUBLE-FLOAT} \hspace{2em} {\it data}(8) \hspace{2em}
$\Rightarrow$ \hspace{2em} stack \\
The {\it data} bytes are read as an integer, then turned into an IEEE double
float (as though by {\tt make-double-float}).

\item[48:] \hspace{2em} {\tt FOP-STRUCT} \hspace{2em} {\it n}(4) \hspace{2em} $\Rightarrow$ \hspace{2em} stack \\
The four-byte operand {\it n} specifies the length structure to construct.  The
elements of the vector are popped off the stack; the first one popped becomes
the last element of the structure.  The constructed vector is pushed onto the
stack.

\item[49:] \hspace{2em} {\tt FOP-SMALL-STRUCT} \hspace{2em} {\it n}(1) \hspace{2em} $\Rightarrow$ \hspace{2em} stack \\
The one-byte operand {\it n} specifies the length structure to construct.  The
elements of the vector are popped off the stack; the first one popped becomes
the last element of the structure.  The constructed vector is pushed onto the
stack.

\item[50-52:] Unused

\item[53:] \hspace{2em} {\tt FOP-EVAL} \hspace{2em} $\Rightarrow$ \hspace{2em} stack \\
Pop an item from the stack and evaluate it (give it to {\tt EVAL}).
Push the result back onto the stack.

\item[54:] \hspace{2em} {\tt FOP-EVAL-FOR-EFFECT} \\
Pop an item from the stack and evaluate it (give it to {\tt EVAL}).
The result is ignored.

\item[55:] \hspace{2em} {\tt FOP-FUNCALL} \hspace{2em} {\it nargs}(1) \hspace{2em} $\Rightarrow$ \hspace{2em} stack \\
Pop {\it nargs}+1 items from the stack and apply the last one popped
as a function to
all the rest as arguments (the first one popped being the last argument).
Push the result back onto the stack.

\item[56:] \hspace{2em} {\tt FOP-FUNCALL-FOR-EFFECT} \hspace{2em} {\it nargs}(1) \\
Pop {\it nargs}+1 items from the stack and apply the last one popped
as a function to
all the rest as arguments (the first one popped being the last argument).
The result is ignored.

\item[57:] \hspace{2em} {\tt FOP-CODE-FORMAT} \hspace{2em} {\it implementation}(1)
\hspace{2em} {\it version}(1) \\
This FOP specifiers the code format for following code objects.  The operations
{\tt FOP-CODE} and its relatives may not occur in a group until after {\tt
FOP-CODE-FORMAT} has appeared; there is no default format.  The {\it
implementation} is an integer indicating the target hardware and environment.
See {\tt compiler/generic/vm-macs.lisp} for the currently defined
implementations.  {\it version} for an implementation is increased whenever
there is a change that renders old fasl files unusable.

\item[58:] \hspace{2em} {\tt FOP-CODE} \hspace{2em} {\it nitems}(4) \hspace{2em} {\it size}(4) \hspace{2em}
{\it code}({\it size}) \hspace{2em} $\Rightarrow$ \hspace{2em} stack \\
A compiled function is constructed and pushed onto the stack.
This object is in the format specified by the most recent
occurrence of {\tt FOP-CODE-FORMAT}.
The operand {\it nitems} specifies a number of items to pop off
the stack to use in the ``boxed storage'' section.  The operand {\it code}
is a string of bytes constituting the compiled executable code.

\item[59:] \hspace{2em} {\tt FOP-SMALL-CODE} \hspace{2em} {\it nitems}(1) \hspace{2em} {\it size}(2) \hspace{2em}
{\it code}({\it size}) \hspace{2em} $\Rightarrow$ \hspace{2em} stack \\
A compiled function is constructed and pushed onto the stack.
This object is in the format specified by the most recent
occurrence of {\tt FOP-CODE-FORMAT}.
The operand {\it nitems} specifies a number of items to pop off
the stack to use in the ``boxed storage'' section.  The operand {\it code}
is a string of bytes constituting the compiled executable code.

\item[60] \hspace{2em} {\tt FOP-FDEFINITION} \hspace{2em} \\
Pops the stack to get an fdefinition.

\item[61] \hspace{2em} {\tt FOP-SANCTIFY-FOR-EXECUTION} \hspace{2em} \\
A code component is popped from the stack, and the necessary magic is applied 
to the code so that it can be executed.

\item[62:] \hspace{2em} {\tt FOP-VERIFY-TABLE-SIZE} \hspace{2em} {\it size}(4) \\
If the current size of the table is not equal to {\it size},
then an inconsistency has been detected.  This operation
is inserted into a Fasload file purely for error-checking purposes.
It is good practice for a compiler to output this at least at the
end of every group, if not more often.

\item[63:] \hspace{2em} {\tt FOP-VERIFY-EMPTY-STACK} \\
If the stack is not currently empty,
then an inconsistency has been detected.  This operation
is inserted into a Fasload file purely for error-checking purposes.
It is good practice for a compiler to output this at least at the
end of every group, if not more often.

\item[64:] \hspace{2em} {\tt FOP-END-GROUP} \\
This is the last operation of a group.	If this is not the
last byte of the file, then a new group follows; the next
nine bytes must be ``{\tt FASL FILE}''.

\item[65:] \hspace{2em} {\tt FOP-POP-FOR-EFFECT} \hspace{2em} stack \hspace{2em} $\Rightarrow$ \hspace{2em} \\
One item is popped from the stack.

\item[66:] \hspace{2em} {\tt FOP-MISC-TRAP} \hspace{2em} $\Rightarrow$ \hspace{2em} stack \\
A trap object is pushed onto the stack.

\item[67:] \hspace{2em} {\tt FOP-DOUBLE-DOUBLE-FLOAT} \hspace{2em} {\it double-double-float}(8) \hspace{2em} $\Rightarrow$ \hspace{2em} stack \\
The next 8 bytes are read, and a double-double-float number is constructed.

\item[68:] \hspace{2em} {\tt FOP-CHARACTER} \hspace{2em} {\it character}(3) \hspace{2em} $\Rightarrow$ \hspace{2em} stack \\
The three bytes are read as an integer then converted to a character.  This FOP
is currently rather useless, as extended characters are not supported.

\item[69:] \hspace{2em} {\tt FOP-SHORT-CHARACTER} \hspace{2em} {\it character}(1) \hspace{2em}
$\Rightarrow$ \hspace{2em} stack \\
The one byte specifies the code of a Common Lisp character object.  A character
is constructed and pushed onto the stack.

\item[70:] \hspace{2em} {\tt FOP-RATIO} \hspace{2em} $\Rightarrow$ \hspace{2em} stack \\
Creates a ratio from two integers popped from the stack.
The denominator is popped first, the numerator second.

\item[71:] \hspace{2em} {\tt FOP-COMPLEX} \hspace{2em} $\Rightarrow$ \hspace{2em} stack \\
Creates a complex number from two numbers popped from the stack.
The imaginary part is popped first, the real part second.

\item[72] \hspace{2em} {\tt FOP-COMPLEX-SINGLE-FLOAT} {\it real(4)} {\it imag(4)}\hspace{2em} $\Rightarrow$ \hspace{2em} stack \\
Creates a complex single-float number from the following 8 bytes.

\item[73] \hspace{2em} {\tt FOP-COMPLEX-DOUBLE-FLOAT} {\it real(8)} {\it imag(8)}\hspace{2em} $\Rightarrow$ \hspace{2em} stack \\
Creates a complex double-float number from the following 16 bytes.


\item[74:] \hspace{2em} {\tt FOP-FSET} \hspace{2em} \\
Except in the cold loader (Genesis), this is a no-op with two stack arguments.
In the initial core this is used to make DEFUN functions defined at cold-load
time so that global functions can be called before top-level forms are run
(which normally installs definitions.)  Genesis pops the top two things off of
the stack and effectively does (SETF SYMBOL-FUNCTION).

\item[75:] \hspace{2em} {\tt FOP-LISP-SYMBOL-SAVE} \hspace{2em} {\it n}(4) \hspace{2em} {\it name}({\it n})
\hspace{2em} $\Rightarrow$ \hspace{2em} stack \& table\\
Like {\tt FOP-SYMBOL-SAVE}, except that it creates a symbol in the LISP
package.

\item[76:] \hspace{2em} {\tt FOP-LISP-SMALL-SYMBOL-SAVE} \hspace{2em} {\it n}(1)
\hspace{2em} {\it name}({\it n}) \hspace{2em} $\Rightarrow$ \hspace{2em} stack
\& table\\
Like {\tt FOP-SMALL-SYMBOL-SAVE}, except that it creates a symbol in the LISP
package.

\item[77:] \hspace{2em} {\tt FOP-KEYWORD-SYMBOL-SAVE} \hspace{2em} {\it n}(4) \hspace{2em} {\it name}({\it n})
\hspace{2em} $\Rightarrow$ \hspace{2em} stack \& table\\
Like {\tt FOP-SYMBOL-SAVE}, except that it creates a symbol in the
KEYWORD package.

\item[78:] \hspace{2em} {\tt FOP-KEYWORD-SMALL-SYMBOL-SAVE} \hspace{2em} {\it n}(1)
\hspace{2em} {\it name}({\it n}) \hspace{2em} $\Rightarrow$ \hspace{2em} stack
\& table\\
Like {\tt FOP-SMALL-SYMBOL-SAVE}, except that it creates a symbol in the
KEYWORD package.

\item[79-80:] Unused

\item[81:] \hspace{2em} {\tt FOP-NORMAL-LOAD}\\
This FOP is used in conjunction with the cold loader (Genesis) to read
top-level package manipulation forms.  These forms are to be read as though by
the normal loaded, so that they can be evaluated at cold load time, instead of
being dumped into the initial core image.  A no-op in normal loading.

\item[82:] \hspace{2em} {\tt FOP-MAYBE-COLD-LOAD}\\
Undoes the effect of {\tt FOP-NORMAL-LOAD}. 

\item[83:] \hspace{2em} {\tt FOP-ARRAY} \hspace{2em} {\it rank}(4)
\hspace{2em} $\Rightarrow$ \hspace{2em} stack\\
This operation creates a simple array header (used for simple-arrays with rank
/= 1).  The data vector is popped off of the stack, and then {\it rank}
dimensions are popped off of the stack (the highest dimensions is on top.)

\item[84:] \hspace{2em} {\tt FOP-SINGLE-FLOAT-VECTOR} \hspace{2em} {\it length}(4) {\it data}(n)
 \hspace{2em} $\Rightarrow$ \hspace{2em} stack\\
Creates a {\it (simple-array single-float (*))} object.  The number of single-floats is {\it length}.

\item[85:] \hspace{2em} {\tt FOP-DOUBLE-FLOAT-VECTOR} \hspace{2em} {\it length}(4) {\it data}(n)
 \hspace{2em} $\Rightarrow$ \hspace{2em} stack\\
Creates a {\it (simple-array double-float (*))} object.  The number of double-floats is {\it length}.

\item[86:] \hspace{2em} {\tt FOP-COMPLEX-SINGLE-FLOAT-VECTOR} \hspace{2em} {\it length}(4) {\it data}(n)
 \hspace{2em} $\Rightarrow$ \hspace{2em} stack\\
Creates a {\it (simple-array (complex single-float) (*))} object.  The number of complex single-floats is {\it length}.

\item[87:] \hspace{2em} {\tt FOP-COMPLEX-DOUBLE-FLOAT-VECTOR} \hspace{2em} {\it length}(4) {\it data}(n)
 \hspace{2em} $\Rightarrow$ \hspace{2em} stack\\
Creates a {\it (simple-array (complex double-float) (*))} object.  The number of complex double-floats is {\it length}.

\item[88:] \hspace{2em} {\tt FOP-DOUBLE-DOUBLE-FLOAT-VECTOR} \hspace{2em} {\it length}(4) {\it data}(n)
 \hspace{2em} $\Rightarrow$ \hspace{2em} stack\\
Creates a {\it (simple-array double-double-float (*))} object.  The number of double-double-floats is {\it length}.

\item[89:] \hspace{2em} {\tt FOP-COMPLEX-DOUBLE-DOUBLE-FLOAT} \hspace{2em} {\it data}(32)
 \hspace{2em} $\Rightarrow$ \hspace{2em} stack\\
Creates a {\it (complex double-double-float)} object from the following 32 bytes of data.

\item[90:] \hspace{2em} {\tt FOP-COMPLEX-DOUBLE-DOUBLE-FLOAT-VECTOR} \hspace{2em} {\it length}(4) {\it data}(n)
 \hspace{2em} $\Rightarrow$ \hspace{2em} stack\\
Creates a {\it (simple-arra (complex double-double-float) (*))} object.  The number of complex double-double-floats is {\it length}.

\item[91-139:] Unused

\item[140:] \hspace{2em} {\tt FOP-ALTER-CODE} \hspace{2em} {\it index}(4)\\
This operation modifies the constants part of a code object (necessary for
creating certain circular function references.)  It pops the new value and code
object are off of the stack, storing the new value at the specified index.

\item[141:] \hspace{2em} {\tt FOP-BYTE-ALTER-CODE} \hspace{2em} {\it index}(1)\\
Like {\tt FOP-ALTER-CODE}, but has only a one byte offset.

\item[142:] \hspace{2em} {\tt FOP-FUNCTION-ENTRY} \hspace{2em} {\it index}(4)
\hspace{2em} $\Rightarrow$ \hspace{2em} stack\\
Initializes a function-entry header inside of a pre-existing code object, and
returns the corresponding function descriptor.  {\it index} is the byte offset
inside of the code object where the header should be plunked down.  The stack
arguments to this operation are the code object, function name, function debug
arglist and function type.

\item[143:] \hspace{2em} {\tt FOP-MAKE-BYTE-COMPILED-FUNCTION} \hspace{2em} {\it size}(1) \hspace{2em} $\Rightarrow$ \hspace{2em} stack\\
Create a byte-compiled function.  {\it FIXME:} describe what's on the stack.

\item[144:] \hspace{2em} {\tt FOP-ASSEMBLER-CODE} \hspace{2em} {\it length}(4)
\hspace{2em} $\Rightarrow$ \hspace{2em} stack\\
This operation creates a code object holding assembly routines.  {\it length}
bytes of code are read and placed in the code object, and the code object
descriptor is pushed on the stack.  This FOP is only recognized by the cold
loader (Genesis.)

\item[145:] \hspace{2em} {\tt FOP-ASSEMBLER-ROUTINE} \hspace{2em} {\it offset}(4)
\hspace{2em} $\Rightarrow$ \hspace{2em} stack\\
This operation records an entry point into an assembler code object (for use
with {\tt FOP-ASSEMBLER-FIXUP}).  The routine name (a symbol) is on stack top.
The code object is underneath.  The entry point is defined at {\it offset}
bytes inside the code area of the code object, and the code object is left on
stack top (allowing multiple uses of this FOP to be chained.)  This FOP is only
recognized by the cold loader (Genesis.)

\item[146:] Unused

\item[147:] \hspace{2em} {\tt FOP-FOREIGN-FIXUP} \hspace{2em} {\it len}(1)
\hspace{2em} {\it name}({\it len})
\hspace{2em} {\it offset}(4) \hspace{2em} $\Rightarrow$ \hspace{2em} stack\\
This operation resolves a reference to a foreign (C) symbol.  {\it len} bytes
are read and interpreted as the symbol {\it name}.  First the {\it kind} and the
code-object to patch are popped from the stack.  The kind is a target-dependent
symbol indicating the instruction format of the patch target (at {\it offset}
bytes from the start of the code area.)  The code object is left on
stack top (allowing multiple uses of this FOP to be chained.)

\item[148:] \hspace{2em} {\tt FOP-ASSEMBLER-FIXUP} \hspace{2em} {\it offset}(4)
\hspace{2em} $\Rightarrow$ \hspace{2em} stack\\
This operation resolves a reference to an assembler routine.  The stack args
are ({\it routine-name}, {\it kind} and {\it code-object}).  The kind is a
target-dependent symbol indicating the instruction format of the patch target
(at {\it offset} bytes from the start of the code area.)  The code object is
left on stack top (allowing multiple uses of this FOP to be chained.)

\item[149:] \hspace{2em} {\tt FOP-CODE-OBJECT-FIXUP} 
\hspace{2em} $\Rightarrow$ \hspace{2em} stack\\
{\it FIXME:} Describe what this does!

\item[150:] \hspace{2em} {\tt FOP-FOREIGN-DATA-FIXUP} 
\hspace{2em} $\Rightarrow$ \hspace{2em} stack\\
{\it FIXME:} Describe what this does!

\item[151-156:] Unused

\item[157:] \hspace{2em} {\tt FOP-LONG-CODE-FORMAT} \hspace{2em} {\it implementation}(1)
\hspace{2em} {\it version}(4) \\
Like FOP-CODE-FORMAT, except that the version is 32 bits long.

\item[158-199:] Unused

\item[200:] \hspace{2em} {\tt FOP-RPLACA} \hspace{2em} {\it table-idx}(4)
\hspace{2em} {\it cdr-offset}(4)\\

\item[201:] \hspace{2em} {\tt FOP-RPLACD} \hspace{2em} {\it table-idx}(4)
\hspace{2em} {\it cdr-offset}(4)\\
These operations destructively modify a list entered in the table.  {\it
table-idx} is the table entry holding the list, and {\it cdr-offset} designates
the cons in the list to modify (like the argument to {\tt nthcdr}.)  The new
value is popped off of the stack, and stored in the {\tt car} or {\tt cdr},
respectively.

\item[202:] \hspace{2em} {\tt FOP-SVSET} \hspace{2em} {\it table-idx}(4)
\hspace{2em} {\it vector-idx}(4)\\
Destructively modifies a {\tt simple-vector} entered in the table.  Pops the
new value off of the stack, and stores it in the {\it vector-idx} element of
the contents of the table entry {\it table-idx.}

\item[203:] \hspace{2em} {\tt FOP-NTHCDR} \hspace{2em} {\it cdr-offset}(4)
\hspace{2em} $\Rightarrow$ \hspace{2em} stack\\
Does {\tt nthcdr} on the top-of stack, leaving the result there.

\item[204:] \hspace{2em} {\tt FOP-STRUCTSET} \hspace{2em} {\it table-idx}(4)
\hspace{2em} {\it vector-idx}(4)\\
Like {\tt FOP-SVSET}, except it alters structure slots.

\item[205-254:] Unused
\item[255:] \hspace{2em} {\tt FOP-END-HEADER} \\ Indicates the end of a group header,
as described above.
\end{description}


\appendix
\chapter{Glossary}% -*- Dictionary: int:design -*-

% Note: in an entry, any word that is also defined should be \it
% should entries have page references as well?

\begin{description}
\item[assert (a type)]
In Python, all type checking is done via a general type assertion
mechanism.  Explicit declarations and implicit assertions (e.g. the arg to
+ is a number) are recorded in the front-end (implicit continuation)
representation.  Type assertions (and thus type-checking) are ``unbundled''
from the operations that are affected by the assertion.  This has two major
advantages:
\begin{itemize}
\item Code that implements operations need not concern itself with checking
operand types.

\item Run-time type checks can be eliminated when the compiler can prove that
the assertion will always be satisfied.
\end{itemize}
See also {\it restrict}.

\item[back end] The back end is the part of the compiler that operates on the
{\it virtual machine} intermediate representation.  Also included are the
compiler phases involved in the conversion from the {\it front end}
representation (or {\it ICR}).

\item[bind node] This is a node type the that marks the start of a {\it lambda}
body in {\it ICR}.  This serves as a placeholder for environment manipulation
code.

\item[IR1] The first intermediate representation, also known as {\it ICR}, or
the Implicit Continuation Represenation.

\item[IR2] The second intermediate representation, also known as {\it VMR}, or
the Virtual Machine Representation.

\item[basic block] A basic block (or simply ``block'') has the pretty much the
usual meaning of representing a straight-line sequence of code.  However, the
code sequence ultimately generated for a block might contain internal branches
that were hidden inside the implementation of a particular operation.  The type
of a block is actually {\tt cblock}.  The {\tt block-info} slot holds an 
{\tt VMR-block} containing backend information.

\item[block compilation] Block compilation is a term commonly used to describe
the compile-time resolution of function names.  This enables many
optimizations.

\item[call graph]
Each node in the call graph is a function (represented by a {\it flow graph}.)
The arcs in the call graph represent a possible call from one function to
another.  See also {\it tail set}.

\item[cleanup]
A cleanup is the part of the implicit continuation representation that
retains information scoping relationships.  For indefinite extent bindings
(variables and functions), we can abandon scoping information after ICR
conversion, recovering the lifetime information using flow analysis.  But
dynamic bindings (special values, catch, unwind protect, etc.) must be
removed at a precise time (whenever the scope is exited.)  Cleanup
structures form a hierarchy that represents the static nesting of dynamic
binding structures.  When the compiler does a control transfer, it can use
the cleanup information to determine what cleanup code needs to be emitted.

\item[closure variable]
A closure variable is any lexical variable that has references outside of
its {\it home environment}.  See also {\it indirect value cell}.

\item[closed continuation] A closed continuation represents a {\tt tagbody} tag
or {\tt block} name that is closed over.  These two cases are mostly
indistinguishable in {\it ICR}.

\item[home] Home is a term used to describe various back-pointers.  A lambda
variable's ``home'' is the lambda that the variable belongs to.  A lambda's ``home
environment'' is the environment in which that lambda's variables are allocated.

\item[indirect value cell]
Any closure variable that has assignments ({\tt setq}s) will be allocated in an
indirect value cell.  This is necessary to ensure that all references to
the variable will see assigned values, since the compiler normally freely
copies values when creating a closure.

\item[set variable] Any variable that is assigned to is called a ``set
variable''.  Several optimizations must special-case set variables, and set
closure variables must have an {\it indirect value cell}.

\item[code generator] The code generator for a {\it VOP} is a potentially
arbitrary list code fragment which is responsible for emitting assembly code to
implement that VOP.

\item[constant pool] The part of a compiled code object that holds pointers to
non-immediate constants.

\item[constant TN]
A constant TN is the {\it VMR} of a compile-time constant value.  A
constant may be immediate, or may be allocated in the {\it constant pool}.

\item[constant leaf]
A constant {\it leaf} is the {\it ICR} of a compile-time constant value.

\item[combination]
A combination {\it node} is the {\it ICR} of any fixed-argument function
call (not {\tt apply} or {\tt multiple-value-call}.)  

\item[top-level component]
A top-level component is any component whose only entry points are top-level
lambdas.

\item[top-level lambda]
A top-level lambda represents the execution of the outermost form on which
the compiler was invoked.  In the case of {\tt compile-file}, this is often a
truly top-level form in the source file, but the compiler can recursively
descend into some forms ({\tt eval-when}, etc.) breaking them into separate
compilations.

\item[component] A component is basically a sequence of blocks.  Each component
is compiled into a separate code object.  With {\it block compilation} or {\it
local functions}, a component will contain the code for more than one function.
This is called a component because it represents a connected portion of the
call graph.  Normally the blocks are in depth-first order ({\it DFO}).

\item[component, initial] During ICR conversion, blocks are temporarily
assigned to initial components.  The ``flow graph canonicalization'' phase
determines the true component structure.

\item[component, head and tail]
The head and tail of a component are dummy blocks that mark the start and
end of the {\it DFO} sequence.  The component head and tail double as the root
and finish node of the component's flow graph.

\item[local function (call)]
A local function call is a call to a function known at compile time to be
in the same {\it component}.  Local call allows compile time resolution of the
target address and calling conventions.  See {\it block compilation}.

\item[conflict (of TNs, set)]
Register allocation terminology.  Two TNs conflict if they could ever be
live simultaneously.  The conflict set of a TN is all TNs that it conflicts
with.

\item[continuation]
The ICR data structure which represents both:
\begin{itemize}
\item The receiving of a value (or multiple values), and

\item A control location in the flow graph.
\end{itemize}
In the Implicit Continuation Representation, the environment is implicit in the
continuation's BLOCK (hence the name.)  The ICR continuation is very similar to
a CPS continuation in its use, but its representation doesn't much resemble (is
not interchangeable with) a lambda.

\item[cont] A slot in the {\it node} holding the {\it continuation} which
receives the node's value(s).  Unless the node ends a {\it block}, this also
implicitly indicates which node should be evaluated next.

\item[cost] Approximations of the run-time costs of operations are widely used
in the back end.  By convention, the unit is generally machine cycles, but the
values are only used for comparison between alternatives.  For example, the
VOP cost is used to determine the preferred order in which to try possible
implementations.
    
\item[CSP, CFP] See {\it control stack pointer} and {\it control frame
pointer}.

\item[Control stack] The main call stack, which holds function stack frames.
All words on the control stack are tagged {\it descriptors}.  In all ports done
so far, the control stack grows from low memory to high memory.  The most
recent call frames are considered to be ``on top'' of earlier call frames.

\item[Control stack pointer] The allocation pointer for the {\it control
stack}.  Generally this points to the first free word at the top of the stack.

\item[Control frame pointer] The pointer to the base of the {\it control stack}
frame for a particular function invocation.  The CFP for the running function
must be in a register.

\item[Number stack] The auxiliary stack used to hold any {\it non-descriptor}
(untagged) objects.  This is generally the same as the C call stack, and thus
typically grows down.

\item[Number stack pointer] The allocation pointer for the {\it number stack}.
This is typically the C stack pointer, and is thus kept in a register.

\item[NSP, NFP] See {\it number stack pointer}, {\it number frame pointer}.

\item[Number frame pointer] The pointer to the base of the {\it number stack}
frame for a particular function invocation.  Functions that don't use the
number stack won't have an NFP, but if an NFP is allocated, it is always
allocated in a particular register.  If there is no variable-size data on the
number stack, then the NFP will generally be identical to the NSP.

\item[Lisp return address] The name of the {\it descriptor} encoding the
``return pc'' for a function call.

\item[LRA] See {\it lisp return address}.  Also, the name of the register where
the LRA is passed.


\item[Code pointer] A pointer to the header of a code object.  The code pointer
for the currently running function is stored in the {\tt code} register.

\item[Interior pointer] A pointer into the inside of some heap-allocated
object.  Interior pointers confuse the garbage collector, so their use is
highly constrained.  Typically there is a single register dedicated to holding
interior pointers.

\item[dest]
A slot in the {\it continuation} which points the the node that receives this
value.  Null if this value is not received by anyone.

\item[DFN, DFO] See {\it Depth First Number}, {\it Depth First Order}.

\item[Depth first number] Blocks are numbered according to their appearance in
the depth-first ordering (the {\tt block-number} slot.)  The numbering actually
increases from the component tail, so earlier blocks have larger numbers.

\item[Depth first order] This is a linearization of the flow graph, obtained by
a depth-first walk.  Iterative flow analysis algorithms work better when blocks
are processed in DFO (or reverse DFO.)


\item[Object] In low-level design discussions, an object is one of the
following:
\begin{itemize}
\item a single word containing immediate data (characters, fixnums, etc)
\item a single word pointing to an object (structures, conses, etc.)
\end{itemize}
These are tagged with three low-tag bits as described in the section
\ref{sec:tagging} This is synonymous with {\it descriptor}.
In other parts of the documentation, may be used more loosely to refer to a
{\it lisp object}.

\item[Lisp object]
A Lisp object is a high-level object discussed as a data type in the Common
Lisp definition.

\item[Data-block]
A data-block is a dual-word aligned block of memory that either manifests a
Lisp object (vectors, code, symbols, etc.) or helps manage a Lisp object on
the heap (array header, function header, etc.).

\item[Descriptor]
A descriptor is a tagged, single-word object.  It either contains immediate
data or a pointer to data.  This is synonymous with {\it object}.  Storage
locations that must contain descriptors are referred to as descriptor
locations.

\item[Pointer descriptor]
A descriptor that points to a {\it data block} in memory (i.e. not an immediate
object.)

\item[Immediate descriptor]
A descriptor that encodes the object value in the descriptor itself; used for
characters, fixnums, etc.

\item[Word]
A word is a 32-bit quantity.

\item[Non-descriptor]
Any chunk of bits that isn't a valid tagged descriptor.  For example, a
double-float on the number stack.  Storage locations that are not scanned by
the garbage collector (and thus cannot contain {\it pointer descriptors}) are
called non-descriptor locations.  {\it Immediate descriptors} can be stored in
non-descriptor locations.


\item[Entry point] An entry point is a function that may be subject to
``unpredictable'' control transfers.  All entry points are linked to the root
of the flow graph (the component head.)  The only functions that aren't entry
points are {\it let} functions.  When complex lambda-list syntax is used,
multiple entry points may be created for a single lisp-level function.
See {\it external entry point}.

\item[External entry point] A function that serves as a ``trampoline'' to
intercept function calls coming in from outside of the component.  The XEP does
argument syntax and type checking, and may also translate the arguments and
return values for a locally specialized calling calling convention.

\item[XEP] An {\it external entry point}.

\item[lexical environment] A lexical environment is a structure that is used
during VMR conversion to represent all lexically scoped bindings (variables,
functions, declarations, etc.)  Each {\tt node} is annotated with its lexical
environment, primarily for use by the debugger and other user interfaces.  This
structure is also the environment object passed to {\tt macroexpand}.

\item[environment] The environment is part of the ICR, created during
environment analysis.  Environment analysis apportions code to disjoint
environments, with all code in the same environment sharing the same stack
frame.  Each environment has a ``{\it real}'' function that allocates it, and
some collection {\tt let} functions.   Although environment analysis is the
last ICR phase, in earlier phases, code is sometimes said to be ``in the
same/different environment(s)''.  This means that the code will definitely be
in the same environment (because it is in the same real function), or that is
might not be in the same environment, because it is not in the same function.

\item[fixup]  Some sort of back-patching annotation.  The main sort encountered
are load-time {\it assembler fixups}, which are a linkage annotation mechanism.

\item[flow graph] A flow graph is a directed graph of basic blocks, where each
arc represents a possible control transfer.  The flow graph is the basic data
structure used to represent code, and provides direct support for data flow
analysis.  See component and ICR.

\item[foldable] An attribute of {\it known functions}.  A function is foldable
if calls may be constant folded whenever the arguments are compile-time
constant.  Generally this means that it is a pure function with no side
effects.


\item[FSC]
\item[full call]
\item[function attribute]
function
        ``real'' (allocates environment)
        meaning function-entry
        more vague (any lambda?)
funny function
GEN (kill and...)
global TN, conflicts, preference
GTN (number)
IR ICR VMR  ICR conversion, VMR conversion (translation)
inline expansion, call
kill (to make dead)
known function
LAMBDA
leaf
let call
lifetime analysis, live (tn, variable)
load tn
LOCS (passing, return locations)
local call
local TN, conflicts, (or just used in one block)
location (selection)
LTN (number)
main entry
mess-up (for cleanup)
more arg (entry)
MV
non-local exit
non-packed SC, TN
non-set variable
operand (to vop)
optimizer (in icr optimize)
optional-dispatch
pack, packing, packed
pass (in a transform)
passing 
        locations (value)
        conventions (known, unknown)
policy (safe, fast, small, ...)
predecessor block
primitive-type
reaching definition
REF
representation
        selection
        for value
result continuation (for function)
result type assertion (for template) (or is it restriction)
restrict
        a TN to finite SBs
        a template operand to a primitive type (boxed...)
        a tn-ref to particular SCs

return (node, vops)
safe, safety
saving (of registers, costs)
SB
SC (restriction)
semi-inline
side-effect
        in ICR
        in VMR
sparse set
splitting (of VMR blocks)
SSET
SUBPRIMITIVE
successor block
tail recursion
        tail recursive
        tail recursive loop
        user tail recursion

template
TN
TNBIND
TN-REF
transform (source, ICR)
type
        assertion
        inference
                top-down, bottom-up
        assertion propagation
        derived, asserted
        descriptor, specifier, intersection, union, member type
        check
type-check (in continuation)
UNBOXED (boxed) descriptor
unknown values continuation
unset variable
unwind-block, unwinding
used value (dest)
value passing
VAR
VM
VOP
\item[XEP]

\end{description}

\end{document}
